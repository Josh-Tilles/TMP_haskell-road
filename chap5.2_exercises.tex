\section{Properties of Relations}

\newcommand{\R}{\mathrel{R}}

%unanswered
\begin{Exercise} [number=17]
  Show that a relation $R$ on $A$ is irreflexive iff $\Delta_A
  \intersect R = \emptyset$.
\end{Exercise}

\begin{Answer}
  \begin{structured_derivation}
    \task{Show that $\Delta_A \intersect R = \emptyset \iff R \textrm{
        is irreflexive}$ when:}
    \assumption{$R \subseteq A^2$}
    \isProvedBy{Derive each side of the equivalence and prove the
      other to prove the equivalence itself.}
    \begin{nested_derivation}
      \task{Show that $\Delta_A \intersect R = \emptyset$ when:}
      \assumption{$R$ is irreflexive}
      \observation{simple substitution based on definition of
        \term{irreflexive}}{$\forall{x \in A} \colon \neg(x \mathbin{R} x)$}
      \observation{reformulation, again based on
        definition}{$\nexists{x \in A} \suchThat (x,x) \in R$}
      
      
    \end{nested_derivation}

    \begin{nested_derivation}
      \task{Show that $R$ is irreflexive when:}
      \assumption{$\Delta_A \intersect R = \emptyset$}
      \observation{??}{
        \begin{nested_derivation}
          \task{$\Delta_A \intersect R = \emptyset$}
          \justification[\equiv]{Substitute by definition}
          \dijk{$\Set{(x,x)}{x \in A} \intersect R = \emptyset$}
          \justification[\equiv]{??}          
        \end{nested_derivation}
}
    \end{nested_derivation}
  \end{structured_derivation}
\end{Answer}

%unanswered
\begin{Exercise} [number=19]
  Show the following:
  \Question A relation $R$ on a set $A$ is symmetric iff $\forall{x,y
    \in A} \colon xRy \iff yRx$.
  \Question A relation $R$ is symmetric iff $R \subseteq R^{-1}$, iff
  $R = R^{-1}$.
\end{Exercise}

%unanswered
\begin{Exercise} [number=20]
  Show that every asymmetric relation is irreflexive.
\end{Exercise}

%unanswered
\begin{Exercise} [number=22]
  Show from the definitions that an asymmetric relation always is
  antisymmetric.
\end{Exercise}

%unanswered
\begin{Exercise} [number=23]
  Show that a relation $R$ on a set $A$ is transitive
  iff \[\forall{x,z \in A} \colon \left( \exists{y \in A} \suchThat
    a \R y \land y \R z \right) \implies x \R z\].
\end{Exercise}

%unanswered
\begin{Exercise} [number=28]
  Show that every strict partial order is asymmetric.
\end{Exercise}

%unanswered
\begin{Exercise} [number=29]
  Show that every relation which is transitive and asymmetric is a
  strict partial order.
\end{Exercise}

%unanswered
\begin{Exercise} [number=30]
  Show that if $R$ is a strict partial order on $A$, then $R \union
  \Delta_A$ is a partial order on $A$. (So every strict partial order
  is contained in a partial order.)
\end{Exercise}

%unanswered
\begin{Exercise} [number=31]
  Show that the inverse of a partial order is again a partial order.
\end{Exercise}

%unanswered
\begin{Exercise} [number=32]
  
  Let $S$ be a reflexive and symmetric relation on a set $A$.

  A path is a finite sequence $a_1, \dots, a_n$ of elements of $A$
  such that for every $i$, $1 \le i < n$, we have that
  $a_{i}\mathrel{S}a_{a+1}$. Such a path connects $a_1$ with $a_n$.
  
  Assume that for all $a$, $b \in A$ there is exactly one path
  connecting $a$ with $b$.

  Fix $r \in A$. Define the relation $\le$ on $A$ by: $a \le b$ iff
  $a$ is one of the elements in the path connecting $r$ with $b$.

  Show the following:
  \Question $\le$ is reflexive
  \Question $\le$ is antisymmetric
  \Question $\le$ is transitive
  \Question for all $a \in A$, $r \le a$
  \Question for every $a \in A$, the set $X_a = \Set{x \in A}{x \le
    a}$ is finite and if $b,c \in X_a$ then $b \le c$ or $c \le b$.

  (A structure $(A,\le,r)$ with these five properties is called a
  tree with root $r$. The directory-structure of a computer account is
  an example of a tree. Another example is the structure tree of a
  formula (see Section 2.3 above). These are example of finite trees,
  but if $A$ is infinite, then the tree $(A,\le,r)$ will have at
  least one infinite branch.)
\end{Exercise}

%unanswered
\begin{Exercise} [number=33]
  Consider the following relations on the natural numbers. Check their
  properties (some answers can be found in the text above). The
  \term{successor} relation is the relation given by
  $\Set{(n,m)}{n+1=m}$. The divisor relation is $\Set{(n,m)}{n
    \textrm{ divides } m}$. The \term{coprime} relation $C$ on
  $\naturals$ is given by $nCm :\equiv \operatorname{GCD}(n,m) = 1$,
  i.e., the only factor of $n$ that divides $m$ is 1, and vice versa.

  \dots (table) \dots
\end{Exercise}

%unanswered
\begin{Exercise} [number=35]
  Suppose that $R$ is a relation on $A$.
  \Question Show that $R \union \Delta_A$ is the reflexive closure of
  $R$.
  \Question Show that $R \union R^{-1}$ is the symmetric closure of $R$.
\end{Exercise}

%unanswered
\begin{Exercise} [number=36]
  Let $R$ be a transitive binary relation on $A$. Does it follow from
  the transitivity of $R$ that its symmetric reflexive closure $R
  \union R^{-1} \union \Delta_A$ is also transitive? Give a proof if
  your answer is yes, a counterexample otherwise.
\end{Exercise}

%unanswered
\begin{Exercise} [number=38]
  Determine the composition of the relation ``father of'' with
  itself. Determine the composition of the relations ``brother of''
  and ``parent of'' Give an example showing that $R \comp S = S \comp
  R$ can be false.
\end{Exercise}

%unanswered
\begin{Exercise} [number=39]
  Consider the relation \[R =
  \{(0,2),(0,3),(1,0),(1,3),(2,0),(2,3)\}\] on the set $A =
  \{0,1,2,3,4\}$.
  \Question Determine $R^2, R^3$ and $R^4$.
  \Question Give a relation $S$ on $A$ such that $R \union (S \comp R)
  = S$.
\end{Exercise}

%unanswered
\begin{Exercise} [number=40]
  Verify:
  \Question A relation $R$ on $A$ is transitive iff $R \comp R
  \subseteq R$.
  \Question Give an example of a transitive relation $R$ for which $R
  \comp R = R$ is false.
\end{Exercise}


%unanswered
\begin{Exercise} [number=41]
  Verify:
  \Question $Q \comp (R \comp S) = (Q \comp R) \comp S$. (Thus, the
  notation $Q \comp R \comp S$ is unambiguous.)
  \Question $(R \comp S)^{-1} = S^{-1} \comp R^{-1}$.
\end{Exercise}


%unanswered
\begin{Exercise} [number=45]
  Show that the relation $<$ on $\naturals$ is the transitive closure
  of the relation $R = \Set{(n,n+1)}{n \in \naturals}$.
\end{Exercise}

%unanswered
\begin{Exercise} [number=46]
  Let $R$ be a relation on $A$. Show that $R^+ \union \Delta_A$ is the
  reflexive transitive closure of $R$.
\end{Exercise}

%unanswered
\begin{Exercise} [number=47]
  Give the reflexive transitive closure of the following relation: \[R
  = \Set{(n,n+1)}{n \in \naturals}\].
\end{Exercise}

%unanswered
\begin{Exercise} [number=48,difficulty=1]
  \Question Show that an intersection of \emph{arbitrarily many} transitive
  relations is transitive.
  \Question Suppose that $R$ is a relation on $A$. Note that $A^2$ is
  one example of a transitive relation on $A$ that extends
  $R$. Conclude that the intersection of \emph{all} transitive relations
  extending $R$ is the \emph{least} transative relation extending $R$. In
  other words, $R^+$ equals the intersection of \emph{all} transitive
  relations extending $R$.
\end{Exercise}

%unanswered
\begin{Exercise} [difficulty=1,number=49]
  \Question Show that $(R^*)^{-1} = (R^{-1})^*$.
  \Question Show by means of a counter-example that $(R \union
  R^{-1})^* = R^* \union R^{-1*}$ may be false.
  \Question Prove: if $S \comp R \subseteq R \comp S$, then $(R \comp
  S)^* \subseteq R^* \comp S^*$.
\end{Exercise}

%unanswered
\begin{Exercise} [number=50]
  Suppose that $R$ and $S$ are reflexive relations on the set
  $A$. Then $\Delta_A \subseteq R$ and $\Delta_A \subseteq S$, so
  $\Delta_A \subseteq R \intersect S$, i.e., $R \intersect S$ is
  reflexive as well. We say: reflexivity is preserved under
  intersection. Similarly, if $R$ and $S$ are reflexive, then
  $\Delta_A \subseteq R \union S$, so $R \union S$ is
  reflexive. Reflexivity is preserved under union. If $R$ is
  reflexive, then $\Delta_A \subseteq R^{-1}$, so $R^{-1}$ is
  reflexive. Reflexivity is preserved under inverse. If $R$ and $S$
  are reflexive, $\Delta_A \subseteq R \comp S$, so $R \comp S$ is
  reflexive. Reflexivity is preserved under composition. Finally, if
  $R$ on $A$ is reflexive, then the complement of $R$, i.e., the
  relation $A^2 \setminus R$, is irreflexive. So reflexivity is not
  preserved under complement. These closure properties of reflexive
  relations are listed in the table below.

  We can ask the same questions for other relational
  properties. Suppose $R$ and $S$ are symmetric relations on $A$. Does
  it follow that $R \intersect S$ is symmetric? That $R \union S$ is
  symmetric? That $R^{-1}$ is symmetric? That $A^2 \setminus R$ is
  symmetric? That $R \comp S$ is symmetric? Similarly for the property
  of transitivity. These questions are summarized in the table
  below. Complete the table by putting `yes' or `no' in the
  appropriate places.

  \dots (table) \dots
\end{Exercise}

%%% Local Variables: 
%%% mode: latex
%%% TeX-master: "exercises_base"
%%% End: 
