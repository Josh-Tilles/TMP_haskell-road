\section{Let's Talk About Sets}

\begin{theorem}
  \label{THM:sets:RAT}
  For all sets $A$, $B$, and $C$, we have that:
  \begin{enumerate}
    \item ``reflexivity'': $A \subseteq A$ 
    \item ``antisymmetry'': $A \subseteq B \land B \subseteq A \implies A = B$ 
    \item ``transitivity'': $A \subseteq B \land B \subseteq C \implies A \subseteq C$ 
  \end{enumerate}
\end{theorem}

\begin{Exercise} [number=2]
  Show that the superset relation also has the properties of Theorem \ref{THM:sets:RAT}. I.e., show that~$\supseteq$ is reflexive, antisymmetric and transitive.
\end{Exercise}

\begin{Answer}
  $A \subseteq B \equiv B \supseteq A$
\end{Answer}

%unfinished
\begin{Exercise} [number=4]
  Show that $\bigl\{ \{ 1, 2 \}, \{ 0 \}, \{ 2, 1 \} \bigr\}  =  \bigl\{ \{ 0 \}, \{ 1, 2 \} \bigr\}$.
\end{Exercise}

\begin{Answer}
  \begin{structured_derivation}
    \task{Prove that $\bigl\{ \{ 1, 2 \}, \{ 0 \}, \{ 2, 1 \} \bigr\}  =  \bigl\{ \{ 0 \}, \{ 1, 2 \} \bigr\}$}
    \observation*
      {order and repetition in this notation are irrelevant}
      { $\{ 1, 2 \} \equiv \{ 2, 1 \} $ }
    \isProvedBy{demonstrating that each is a subset of the other}
    \begin{nested_derivation}
      \task{Prove that $\bigl\{ \{ 1, 2 \}, \{ 0 \}, \{ 2, 1 \} \bigr\} \subseteq \bigl\{ \{ 0 \}, \{ 1, 2 \} \bigr\}$.}
      
    \end{nested_derivation}
    \begin{nested_derivation}
      \task{Prove that $\bigl\{ \{ 1, 2 \}, \{ 0 \}, \{ 2, 1 \} \bigr\} \supseteq \bigl\{ \{ 0 \}, \{ 1, 2 \} \bigr\}$.}
    \end{nested_derivation}
  \end{structured_derivation}
\end{Answer}

%unanswered
\begin{Exercise} [number=7, difficulty=1]
  Assume that $A$ is a set of sets. Show that $\{ x \in A \suchthat x \notin x \} \notin A$.
\end{Exercise}

%answer as something to do with principle of extension. I think. Like,
%{x in A} = A, but not {x in A} in A. Well, UNLESS A can contain
%itself. If it can, then there exists x in A such that x = A. ...
