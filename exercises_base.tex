\documentclass{report}
\usepackage{haskell_exercises}

\usepackage{amssymb}
\newcommand*{\wrt}{w.r.t.}

%% ODER: format ==         = "\mathrel{==}"
%% ODER: format /=         = "\neq "
%
%
\makeatletter
\@ifundefined{lhs2tex.lhs2tex.sty.read}%
  {\@namedef{lhs2tex.lhs2tex.sty.read}{}%
   \newcommand\SkipToFmtEnd{}%
   \newcommand\EndFmtInput{}%
   \long\def\SkipToFmtEnd#1\EndFmtInput{}%
  }\SkipToFmtEnd

\newcommand\ReadOnlyOnce[1]{\@ifundefined{#1}{\@namedef{#1}{}}\SkipToFmtEnd}
\usepackage{amstext}
\usepackage{amssymb}
\usepackage{stmaryrd}
\DeclareFontFamily{OT1}{cmtex}{}
\DeclareFontShape{OT1}{cmtex}{m}{n}
  {<5><6><7><8>cmtex8
   <9>cmtex9
   <10><10.95><12><14.4><17.28><20.74><24.88>cmtex10}{}
\DeclareFontShape{OT1}{cmtex}{m}{it}
  {<-> ssub * cmtt/m/it}{}
\newcommand{\texfamily}{\fontfamily{cmtex}\selectfont}
\DeclareFontShape{OT1}{cmtt}{bx}{n}
  {<5><6><7><8>cmtt8
   <9>cmbtt9
   <10><10.95><12><14.4><17.28><20.74><24.88>cmbtt10}{}
\DeclareFontShape{OT1}{cmtex}{bx}{n}
  {<-> ssub * cmtt/bx/n}{}
\newcommand{\tex}[1]{\text{\texfamily#1}}	% NEU

\newcommand{\Sp}{\hskip.33334em\relax}


\newcommand{\Conid}[1]{\mathit{#1}}
\newcommand{\Varid}[1]{\mathit{#1}}
\newcommand{\anonymous}{\kern0.06em \vbox{\hrule\@width.5em}}
\newcommand{\plus}{\mathbin{+\!\!\!+}}
\newcommand{\bind}{\mathbin{>\!\!\!>\mkern-6.7mu=}}
\newcommand{\rbind}{\mathbin{=\mkern-6.7mu<\!\!\!<}}% suggested by Neil Mitchell
\newcommand{\sequ}{\mathbin{>\!\!\!>}}
\renewcommand{\leq}{\leqslant}
\renewcommand{\geq}{\geqslant}
\usepackage{polytable}

%mathindent has to be defined
\@ifundefined{mathindent}%
  {\newdimen\mathindent\mathindent\leftmargini}%
  {}%

\def\resethooks{%
  \global\let\SaveRestoreHook\empty
  \global\let\ColumnHook\empty}
\newcommand*{\savecolumns}[1][default]%
  {\g@addto@macro\SaveRestoreHook{\savecolumns[#1]}}
\newcommand*{\restorecolumns}[1][default]%
  {\g@addto@macro\SaveRestoreHook{\restorecolumns[#1]}}
\newcommand*{\aligncolumn}[2]%
  {\g@addto@macro\ColumnHook{\column{#1}{#2}}}

\resethooks

\newcommand{\onelinecommentchars}{\quad-{}- }
\newcommand{\commentbeginchars}{\enskip\{-}
\newcommand{\commentendchars}{-\}\enskip}

\newcommand{\visiblecomments}{%
  \let\onelinecomment=\onelinecommentchars
  \let\commentbegin=\commentbeginchars
  \let\commentend=\commentendchars}

\newcommand{\invisiblecomments}{%
  \let\onelinecomment=\empty
  \let\commentbegin=\empty
  \let\commentend=\empty}

\visiblecomments

\newlength{\blanklineskip}
\setlength{\blanklineskip}{0.66084ex}

\newcommand{\hsindent}[1]{\quad}% default is fixed indentation
\let\hspre\empty
\let\hspost\empty
\newcommand{\NB}{\textbf{NB}}
\newcommand{\Todo}[1]{$\langle$\textbf{To do:}~#1$\rangle$}

\EndFmtInput
\makeatother
%
%
%
%
%
%
% This package provides two environments suitable to take the place
% of hscode, called "plainhscode" and "arrayhscode". 
%
% The plain environment surrounds each code block by vertical space,
% and it uses \abovedisplayskip and \belowdisplayskip to get spacing
% similar to formulas. Note that if these dimensions are changed,
% the spacing around displayed math formulas changes as well.
% All code is indented using \leftskip.
%
% Changed 19.08.2004 to reflect changes in colorcode. Should work with
% CodeGroup.sty.
%
\ReadOnlyOnce{polycode.fmt}%
\makeatletter

\newcommand{\hsnewpar}[1]%
  {{\parskip=0pt\parindent=0pt\par\vskip #1\noindent}}

% can be used, for instance, to redefine the code size, by setting the
% command to \small or something alike
\newcommand{\hscodestyle}{}

% The command \sethscode can be used to switch the code formatting
% behaviour by mapping the hscode environment in the subst directive
% to a new LaTeX environment.

\newcommand{\sethscode}[1]%
  {\expandafter\let\expandafter\hscode\csname #1\endcsname
   \expandafter\let\expandafter\endhscode\csname end#1\endcsname}

% "compatibility" mode restores the non-polycode.fmt layout.

\newenvironment{compathscode}%
  {\par\noindent
   \advance\leftskip\mathindent
   \hscodestyle
   \let\\=\@normalcr
   \let\hspre\(\let\hspost\)%
   \pboxed}%
  {\endpboxed\)%
   \par\noindent
   \ignorespacesafterend}

\newcommand{\compaths}{\sethscode{compathscode}}

% "plain" mode is the proposed default.
% It should now work with \centering.
% This required some changes. The old version
% is still available for reference as oldplainhscode.

\newenvironment{plainhscode}%
  {\hsnewpar\abovedisplayskip
   \advance\leftskip\mathindent
   \hscodestyle
   \let\hspre\(\let\hspost\)%
   \pboxed}%
  {\endpboxed%
   \hsnewpar\belowdisplayskip
   \ignorespacesafterend}

\newenvironment{oldplainhscode}%
  {\hsnewpar\abovedisplayskip
   \advance\leftskip\mathindent
   \hscodestyle
   \let\\=\@normalcr
   \(\pboxed}%
  {\endpboxed\)%
   \hsnewpar\belowdisplayskip
   \ignorespacesafterend}

% Here, we make plainhscode the default environment.

\newcommand{\plainhs}{\sethscode{plainhscode}}
\newcommand{\oldplainhs}{\sethscode{oldplainhscode}}
\plainhs

% The arrayhscode is like plain, but makes use of polytable's
% parray environment which disallows page breaks in code blocks.

\newenvironment{arrayhscode}%
  {\hsnewpar\abovedisplayskip
   \advance\leftskip\mathindent
   \hscodestyle
   \let\\=\@normalcr
   \(\parray}%
  {\endparray\)%
   \hsnewpar\belowdisplayskip
   \ignorespacesafterend}

\newcommand{\arrayhs}{\sethscode{arrayhscode}}

% The mathhscode environment also makes use of polytable's parray 
% environment. It is supposed to be used only inside math mode 
% (I used it to typeset the type rules in my thesis).

\newenvironment{mathhscode}%
  {\parray}{\endparray}

\newcommand{\mathhs}{\sethscode{mathhscode}}

% texths is similar to mathhs, but works in text mode.

\newenvironment{texthscode}%
  {\(\parray}{\endparray\)}

\newcommand{\texths}{\sethscode{texthscode}}

% The framed environment places code in a framed box.

\def\codeframewidth{\arrayrulewidth}
\RequirePackage{calc}

\newenvironment{framedhscode}%
  {\parskip=\abovedisplayskip\par\noindent
   \hscodestyle
   \arrayrulewidth=\codeframewidth
   \tabular{@{}|p{\linewidth-2\arraycolsep-2\arrayrulewidth-2pt}|@{}}%
   \hline\framedhslinecorrect\\{-1.5ex}%
   \let\endoflinesave=\\
   \let\\=\@normalcr
   \(\pboxed}%
  {\endpboxed\)%
   \framedhslinecorrect\endoflinesave{.5ex}\hline
   \endtabular
   \parskip=\belowdisplayskip\par\noindent
   \ignorespacesafterend}

\newcommand{\framedhslinecorrect}[2]%
  {#1[#2]}

\newcommand{\framedhs}{\sethscode{framedhscode}}

% The inlinehscode environment is an experimental environment
% that can be used to typeset displayed code inline.

\newenvironment{inlinehscode}%
  {\(\def\column##1##2{}%
   \let\>\undefined\let\<\undefined\let\\\undefined
   \newcommand\>[1][]{}\newcommand\<[1][]{}\newcommand\\[1][]{}%
   \def\fromto##1##2##3{##3}%
   \def\nextline{}}{\) }%

\newcommand{\inlinehs}{\sethscode{inlinehscode}}

% The joincode environment is a separate environment that
% can be used to surround and thereby connect multiple code
% blocks.

\newenvironment{joincode}%
  {\let\orighscode=\hscode
   \let\origendhscode=\endhscode
   \def\endhscode{\def\hscode{\endgroup\def\@currenvir{hscode}\\}\begingroup}
   %\let\SaveRestoreHook=\empty
   %\let\ColumnHook=\empty
   %\let\resethooks=\empty
   \orighscode\def\hscode{\endgroup\def\@currenvir{hscode}}}%
  {\origendhscode
   \global\let\hscode=\orighscode
   \global\let\endhscode=\origendhscode}%

\makeatother
\EndFmtInput
%

% TODO put the above in an \input-able place
% TODO I wish grave marks (`) weren't rendered like left-single-quotes.

\begin{document}

\chapter{Getting Started}
We use LD(n) for the least natural number greater than 1 that divides n.

\begin{hscode}\SaveRestoreHook
\column{B}{@{}>{\hspre}l<{\hspost}@{}}%
\column{5}{@{}>{\hspre}l<{\hspost}@{}}%
\column{9}{@{}>{\hspre}l<{\hspost}@{}}%
\column{E}{@{}>{\hspre}l<{\hspost}@{}}%
\>[5]{}\Varid{divides}\;\Varid{d}\;\Varid{n}\mathrel{=}{}\<[E]%
\\
\>[5]{}\hsindent{4}{}\<[9]%
\>[9]{}\Varid{rem}\;\Varid{n}\;\Varid{d}\equiv \mathrm{0}{}\<[E]%
\ColumnHook
\end{hscode}\resethooks

It is useful to define LD in terms of a second function that
calculates the least divisor starting from a given threshold k, with
$k <= n$.

\begin{hscode}\SaveRestoreHook
\column{B}{@{}>{\hspre}l<{\hspost}@{}}%
\column{5}{@{}>{\hspre}l<{\hspost}@{}}%
\column{14}{@{}>{\hspre}l<{\hspost}@{}}%
\column{31}{@{}>{\hspre}l<{\hspost}@{}}%
\column{E}{@{}>{\hspre}l<{\hspost}@{}}%
\>[5]{}\Varid{ld}\;\Varid{n}\mathrel{=}\Varid{ldf}\;\mathrm{2}\;\Varid{n}{}\<[E]%
\\[\blanklineskip]%
\>[5]{}\Varid{ldf}\;\Varid{k}\;\Varid{n}{}\<[14]%
\>[14]{}\mid \Varid{k}\mathbin{`\Varid{divides}`}\Varid{n}{}\<[31]%
\>[31]{}\mathrel{=}\Varid{k}{}\<[E]%
\\
\>[14]{}\mid \Varid{k}^\mathrm{2}\mathbin{>}\Varid{n}{}\<[31]%
\>[31]{}\mathrel{=}\Varid{n}{}\<[E]%
\\
\>[14]{}\mid \Varid{otherwise}{}\<[31]%
\>[31]{}\mathrel{=}\Varid{ldf}\;(\Varid{k}\mathbin{+}\mathrm{1})\;\Varid{n}{}\<[E]%
\ColumnHook
\end{hscode}\resethooks

\begin{Exercise} [number=4]
  If \ensuremath{\Varid{ldf}} used $k^2 \geq n$ how would that change the function?
\end{Exercise}

\begin{Answer}
  It wouldn't, because otherwise \ensuremath{\Varid{divides}\;\Varid{k}\;\Varid{n}} would have been \ensuremath{\Conid{True}}.
\end{Answer}

\begin{hscode}\SaveRestoreHook
\column{B}{@{}>{\hspre}l<{\hspost}@{}}%
\column{5}{@{}>{\hspre}l<{\hspost}@{}}%
\column{9}{@{}>{\hspre}l<{\hspost}@{}}%
\column{22}{@{}>{\hspre}l<{\hspost}@{}}%
\column{E}{@{}>{\hspre}l<{\hspost}@{}}%
\>[5]{}\Varid{prime0}\;\Varid{n}{}\<[E]%
\\
\>[5]{}\hsindent{4}{}\<[9]%
\>[9]{}\mid \Varid{n}\mathbin{>}\mathrm{1}{}\<[22]%
\>[22]{}\mathrel{=}\Varid{error}\;\text{\tt \char34 not~a~positive~integer\char34}{}\<[E]%
\\
\>[5]{}\hsindent{4}{}\<[9]%
\>[9]{}\mid \Varid{n}\equiv \mathrm{1}{}\<[22]%
\>[22]{}\mathrel{=}\Conid{False}{}\<[E]%
\\
\>[5]{}\hsindent{4}{}\<[9]%
\>[9]{}\mid \Varid{otherwise}{}\<[22]%
\>[22]{}\mathrel{=}\Varid{ld}\;\Varid{n}\equiv \Varid{n}{}\<[E]%
\ColumnHook
\end{hscode}\resethooks

\begin{Exercise} [number=6]
  Can you gather from the definition of \ensuremath{\Varid{divides}} what the type
  declaration for \ensuremath{\Varid{rem}} would look like?
\end{Exercise}

\begin{Answer}
\begin{hscode}\SaveRestoreHook
\column{B}{@{}>{\hspre}l<{\hspost}@{}}%
\column{E}{@{}>{\hspre}l<{\hspost}@{}}%
\>[B]{}\Varid{rem}\mathbin{::}\Conid{Integer}\to \Conid{Integer}\to \Conid{Integer}{}\<[E]%
\ColumnHook
\end{hscode}\resethooks
\end{Answer}

\begin{Exercise} [number=9]
  Define a function that given the maximum of a list of integers. Use
  the predefined function \ensuremath{\Varid{max}}.
\end{Exercise}

\begin{Answer}
\begin{hscode}\SaveRestoreHook
\column{B}{@{}>{\hspre}l<{\hspost}@{}}%
\column{15}{@{}>{\hspre}l<{\hspost}@{}}%
\column{E}{@{}>{\hspre}l<{\hspost}@{}}%
\>[B]{}\Varid{maxIn}\mathbin{::}[\mskip1.5mu \Conid{Int}\mskip1.5mu]\to \Conid{Int}{}\<[E]%
\\
\>[B]{}\Varid{maxIn}\;[\mskip1.5mu \mskip1.5mu]{}\<[15]%
\>[15]{}\mathrel{=}\Varid{error}\;\text{\tt \char34 UNDEFINED:~empty~list\char34}{}\<[E]%
\\
\>[B]{}\Varid{maxIn}\;[\mskip1.5mu \Varid{x}\mskip1.5mu]{}\<[15]%
\>[15]{}\mathrel{=}\Varid{x}{}\<[E]%
\\
\>[B]{}\Varid{maxIn}\;(\Varid{x}\mathbin{:}\Varid{xs}){}\<[15]%
\>[15]{}\mathrel{=}\Varid{max}\;\Varid{x}\;(\Varid{maxIn}\;\Varid{xs}){}\<[E]%
\ColumnHook
\end{hscode}\resethooks
\end{Answer}

\begin{Exercise} [number=10]
  Define a function \text{\tt removeFst} that removes the first occurrence of an integer $m$ from a lst of integers. if \ensuremath{\Varid{m}} does not occur n the list, the list remains unchanged.
\end{Exercise}

\begin{Answer}
\begin{hscode}\SaveRestoreHook
\column{B}{@{}>{\hspre}l<{\hspost}@{}}%
\column{12}{@{}>{\hspre}l<{\hspost}@{}}%
\column{15}{@{}>{\hspre}l<{\hspost}@{}}%
\column{23}{@{}>{\hspre}l<{\hspost}@{}}%
\column{36}{@{}>{\hspre}l<{\hspost}@{}}%
\column{E}{@{}>{\hspre}l<{\hspost}@{}}%
\>[B]{}\Varid{removeFst}\mathbin{::}\Conid{Int}\to [\mskip1.5mu \Conid{Int}\mskip1.5mu]\to [\mskip1.5mu \Conid{Int}\mskip1.5mu]{}\<[E]%
\\
\>[B]{}\Varid{removeFst}\;{}\<[12]%
\>[12]{}\anonymous \;{}\<[15]%
\>[15]{}[\mskip1.5mu \mskip1.5mu]{}\<[36]%
\>[36]{}\mathrel{=}[\mskip1.5mu \mskip1.5mu]{}\<[E]%
\\
\>[B]{}\Varid{removeFst}\;{}\<[12]%
\>[12]{}\Varid{m}\;{}\<[15]%
\>[15]{}(\Varid{x}\mathbin{:}\Varid{xs}){}\<[23]%
\>[23]{}\mid \Varid{m}\equiv \Varid{x}{}\<[36]%
\>[36]{}\mathrel{=}\Varid{xs}{}\<[E]%
\\
\>[23]{}\mid \Varid{otherwise}{}\<[36]%
\>[36]{}\mathrel{=}\Varid{x}\mathbin{:}\Varid{removeFst}\;\Varid{m}\;\Varid{xs}{}\<[E]%
\ColumnHook
\end{hscode}\resethooks
\end{Answer}

\begin{Exercise} [number=13]
  Write a function \ensuremath{\Varid{count}} for counting the number of occurrences of a
  character in a string.
\end{Exercise}

\begin{Answer}
\begin{hscode}\SaveRestoreHook
\column{B}{@{}>{\hspre}l<{\hspost}@{}}%
\column{10}{@{}>{\hspre}l<{\hspost}@{}}%
\column{18}{@{}>{\hspre}l<{\hspost}@{}}%
\column{31}{@{}>{\hspre}l<{\hspost}@{}}%
\column{E}{@{}>{\hspre}l<{\hspost}@{}}%
\>[B]{}\Varid{count}\mathbin{::}\Conid{Char}\to \Conid{String}\to \Conid{Int}{}\<[E]%
\\
\>[B]{}\Varid{count}\;\anonymous \;{}\<[10]%
\>[10]{}[\mskip1.5mu \mskip1.5mu]\mathrel{=}\mathrm{0}{}\<[E]%
\\
\>[B]{}\Varid{count}\;\Varid{c}\;{}\<[10]%
\>[10]{}(\Varid{y}\mathbin{:}\Varid{ys}){}\<[18]%
\>[18]{}\mid \Varid{c}\equiv \Varid{y}{}\<[31]%
\>[31]{}\mathrel{=}\mathrm{1}\mathbin{+}\Varid{count}\;\Varid{c}\;\Varid{ys}{}\<[E]%
\\
\>[18]{}\mid \Varid{otherwise}{}\<[31]%
\>[31]{}\mathrel{=}\Varid{count}\;\Varid{c}\;\Varid{ys}{}\<[E]%
\ColumnHook
\end{hscode}\resethooks
\end{Answer}

\begin{Exercise} [number=14]
  Write a function \text{\tt blowup} such that \ensuremath{\Varid{blowup}``\Varid{bang}\mathbin{!}\text{\tt ''}} should yield
  \ensuremath{\text{\tt \char34 baannngggg!!!!!\char34}}
\end{Exercise}

\begin{Answer}

\begin{hscode}\SaveRestoreHook
\column{B}{@{}>{\hspre}l<{\hspost}@{}}%
\column{E}{@{}>{\hspre}l<{\hspost}@{}}%
\>[B]{}\Varid{blowup}\mathbin{::}\Conid{String}\to \Conid{String}{}\<[E]%
\\
\>[B]{}\Varid{blowup}\mathrel{=}\Varid{concat}\mathbin{\circ}\Varid{zipWith}\;\Varid{replicate}\;[\mskip1.5mu \mathrm{1}\mathinner{\ldotp\ldotp}\mskip1.5mu]{}\<[E]%
\ColumnHook
\end{hscode}\resethooks

The above solution is in point-free form because I'm under the
impression that point-free form is what Haskellers aim for by
default. Like, to write in point-free form often is an achievement.
Check out the slight difference (the readability in particular) in the following definition:
\begin{hscode}\SaveRestoreHook
\column{B}{@{}>{\hspre}l<{\hspost}@{}}%
\column{E}{@{}>{\hspre}l<{\hspost}@{}}%
\>[B]{}\Varid{blowup}\;\Varid{chrs}\mathrel{=}\Varid{concat}\;(\Varid{zipWith}\;\Varid{replicate}\;[\mskip1.5mu \mathrm{1}\mathinner{\ldotp\ldotp}\mskip1.5mu]\;\Varid{chrs}){}\<[E]%
\ColumnHook
\end{hscode}\resethooks

\end{Answer}


\begin{Exercise} [number=15]
Write a function
\begin{hscode}\SaveRestoreHook
\column{B}{@{}>{\hspre}l<{\hspost}@{}}%
\column{E}{@{}>{\hspre}l<{\hspost}@{}}%
\>[B]{}\Varid{srtString}\mathbin{::}[\mskip1.5mu \Conid{String}\mskip1.5mu]\to [\mskip1.5mu \Conid{String}\mskip1.5mu]{}\<[E]%
\ColumnHook
\end{hscode}\resethooks
that sorts a list of strings in alphabetical order.
\end{Exercise}

\begin{Answer}
\begin{hscode}\SaveRestoreHook
\column{B}{@{}>{\hspre}l<{\hspost}@{}}%
\column{E}{@{}>{\hspre}l<{\hspost}@{}}%
\>[B]{}\Varid{srtString}\;[\mskip1.5mu \mskip1.5mu]\mathrel{=}[\mskip1.5mu \mskip1.5mu]{}\<[E]%
\\
\>[B]{}\Varid{srtString}\;\Varid{xs}\mathrel{=}\Varid{srtString}\;(\Varid{remove}){}\<[E]%
\ColumnHook
\end{hscode}\resethooks

I know I probably could have reversed the definition order to be a little more terse, like this:
\begin{hscode}\SaveRestoreHook
\ColumnHook
\end{hscode}\resethooks
\end{Answer}




\problem{1.17} Write a function \ensuremath{\Varid{substring}\mathbin{::}\Conid{String}\to \Conid{String}\to \Conid{Bool}} that checks whether \ensuremath{\Varid{str1}} is a substring of \ensuremath{\Varid{str2}}.

The \text{\tt prefix} function was given in Example 1.16:
\begin{hscode}\SaveRestoreHook
\column{B}{@{}>{\hspre}l<{\hspost}@{}}%
\column{5}{@{}>{\hspre}l<{\hspost}@{}}%
\column{20}{@{}>{\hspre}l<{\hspost}@{}}%
\column{28}{@{}>{\hspre}l<{\hspost}@{}}%
\column{E}{@{}>{\hspre}l<{\hspost}@{}}%
\>[5]{}\Varid{prefix}\mathbin{::}\Conid{String}\to \Conid{String}\to \Conid{Bool}{}\<[E]%
\\
\>[5]{}\Varid{prefix}\;[\mskip1.5mu \mskip1.5mu]\;{}\<[20]%
\>[20]{}\anonymous {}\<[28]%
\>[28]{}\mathrel{=}\Conid{True}{}\<[E]%
\\
\>[5]{}\Varid{prefix}\;\anonymous \;{}\<[20]%
\>[20]{}[\mskip1.5mu \mskip1.5mu]{}\<[28]%
\>[28]{}\mathrel{=}\Conid{False}{}\<[E]%
\\
\>[5]{}\Varid{prefix}\;(\Varid{x}\mathbin{:}\Varid{xs})\;{}\<[20]%
\>[20]{}(\Varid{y}\mathbin{:}\Varid{ys}){}\<[28]%
\>[28]{}\mathrel{=}(\Varid{x}\equiv \Varid{y})\mathrel{\wedge}\Varid{prefix}\;\Varid{xs}\;\Varid{ys}{}\<[E]%
\ColumnHook
\end{hscode}\resethooks

The actual definition for \text{\tt substring} is here:
\begin{hscode}\SaveRestoreHook
\column{B}{@{}>{\hspre}l<{\hspost}@{}}%
\column{5}{@{}>{\hspre}l<{\hspost}@{}}%
\column{9}{@{}>{\hspre}l<{\hspost}@{}}%
\column{29}{@{}>{\hspre}l<{\hspost}@{}}%
\column{E}{@{}>{\hspre}l<{\hspost}@{}}%
\>[5]{}\Varid{substring}\mathbin{::}\Conid{String}\to \Conid{String}\to \Conid{Bool}{}\<[E]%
\\
\>[5]{}\Varid{substring}\;\Varid{str1}\;\Varid{str2}\mathord{@}(\anonymous \mathbin{:}\Varid{restOfStr2}){}\<[E]%
\\
\>[5]{}\hsindent{4}{}\<[9]%
\>[9]{}\mid \Varid{prefix}\;\Varid{str1}\;\Varid{str2}{}\<[29]%
\>[29]{}\mathrel{=}\Conid{True}{}\<[E]%
\\
\>[5]{}\hsindent{4}{}\<[9]%
\>[9]{}\mid \Varid{otherwise}{}\<[29]%
\>[29]{}\mathrel{=}\Varid{prefix}\;\Varid{str1}\;\Varid{restOfStr2}{}\<[E]%
\ColumnHook
\end{hscode}\resethooks


\problem{1.20} Use \ensuremath{\Varid{map}} to write a function \ensuremath{\Varid{lengths}} that takes a list of lists and returns a list of the corresponding lengths.
\begin{hscode}\SaveRestoreHook
\column{B}{@{}>{\hspre}l<{\hspost}@{}}%
\column{5}{@{}>{\hspre}l<{\hspost}@{}}%
\column{E}{@{}>{\hspre}l<{\hspost}@{}}%
\>[5]{}\Varid{lengths}\mathbin{::}[\mskip1.5mu [\mskip1.5mu \Varid{a}\mskip1.5mu]\mskip1.5mu]\to [\mskip1.5mu \Conid{Int}\mskip1.5mu]{}\<[E]%
\\
\>[5]{}\Varid{lengths}\mathrel{=}\Varid{map}\;\Varid{length}{}\<[E]%
\ColumnHook
\end{hscode}\resethooks

\problem{1.21} Use \ensuremath{\Varid{map}} to write a function \ensuremath{\Varid{sumLengths}} that takes a list of lists and returns the sum of their lengths.
\begin{hscode}\SaveRestoreHook
\column{B}{@{}>{\hspre}l<{\hspost}@{}}%
\column{5}{@{}>{\hspre}l<{\hspost}@{}}%
\column{18}{@{}>{\hspre}l<{\hspost}@{}}%
\column{E}{@{}>{\hspre}l<{\hspost}@{}}%
\>[5]{}\Varid{sumLengths}\mathbin{::}[\mskip1.5mu [\mskip1.5mu \Varid{a}\mskip1.5mu]\mskip1.5mu]\to \Conid{Int}{}\<[E]%
\\
\>[5]{}\Varid{sumLengths}{}\<[18]%
\>[18]{}\mathrel{=}\Varid{sum}\mathbin{\circ}\Varid{map}\;\Varid{length}{}\<[E]%
\\
\>[5]{}\Varid{sumLengths'}{}\<[18]%
\>[18]{}\mathrel{=}\Varid{sum}\mathbin{\circ}\Varid{lengths}{}\<[E]%
\ColumnHook
\end{hscode}\resethooks

\begin{hscode}\SaveRestoreHook
\column{B}{@{}>{\hspre}l<{\hspost}@{}}%
\column{5}{@{}>{\hspre}l<{\hspost}@{}}%
\column{14}{@{}>{\hspre}l<{\hspost}@{}}%
\column{16}{@{}>{\hspre}l<{\hspost}@{}}%
\column{20}{@{}>{\hspre}l<{\hspost}@{}}%
\column{27}{@{}>{\hspre}l<{\hspost}@{}}%
\column{29}{@{}>{\hspre}l<{\hspost}@{}}%
\column{36}{@{}>{\hspre}l<{\hspost}@{}}%
\column{E}{@{}>{\hspre}l<{\hspost}@{}}%
\>[5]{}\Varid{factors}\mathbin{::}\Conid{Integer}\to [\mskip1.5mu \Conid{Integer}\mskip1.5mu]{}\<[E]%
\\
\>[5]{}\Varid{factors}\;\Varid{n}{}\<[16]%
\>[16]{}\mid \Varid{n}\mathbin{<}\mathrm{1}{}\<[29]%
\>[29]{}\mathrel{=}\Varid{error}\;\text{\tt \char34 argument~not~positive\char34}{}\<[E]%
\\
\>[16]{}\mid \Varid{n}\equiv \mathrm{1}{}\<[29]%
\>[29]{}\mathrel{=}[\mskip1.5mu \mskip1.5mu]{}\<[E]%
\\
\>[16]{}\mid \Varid{otherwise}{}\<[29]%
\>[29]{}\mathrel{=}\Varid{p}\mathbin{:}\Varid{factors}\;(\Varid{n}\mathbin{\Varid{`div`}}\Varid{p}){}\<[E]%
\\
\>[16]{}\hsindent{4}{}\<[20]%
\>[20]{}\mathbf{where}\;\Varid{p}\mathrel{=}\Varid{ld}\;\Varid{n}{}\<[E]%
\\[\blanklineskip]%
\>[5]{}\Varid{primes0}\mathbin{::}[\mskip1.5mu \Conid{Integer}\mskip1.5mu]{}\<[E]%
\\
\>[5]{}\Varid{primes0}\mathrel{=}\Varid{filter}\;\Varid{prime0}\;[\mskip1.5mu \mathrm{2}\mathinner{\ldotp\ldotp}\mskip1.5mu]{}\<[E]%
\\[\blanklineskip]%
\>[5]{}\Varid{ldp}\mathbin{::}\Conid{Integer}\to \Conid{Integer}{}\<[E]%
\\
\>[5]{}\Varid{ldp}\;\Varid{n}\mathrel{=}\Varid{ldpf}\;\Varid{primes1}\;\Varid{n}{}\<[E]%
\\[\blanklineskip]%
\>[5]{}\Varid{ldpf}\mathbin{::}[\mskip1.5mu \Conid{Integer}\mskip1.5mu]\to \Conid{Integer}\to \Conid{Integer}{}\<[E]%
\\
\>[5]{}\Varid{ldpf}\;(\Varid{p}\mathbin{:}\Varid{ps})\;\Varid{n}{}\<[20]%
\>[20]{}\mid \Varid{rem}\;\Varid{n}\;\Varid{p}\equiv \mathrm{0}{}\<[36]%
\>[36]{}\mathrel{=}\Varid{p}{}\<[E]%
\\
\>[20]{}\mid \Varid{p}^\mathrm{2}\mathbin{>}\Varid{n}{}\<[36]%
\>[36]{}\mathrel{=}\Varid{n}{}\<[E]%
\\
\>[20]{}\mid \Varid{otherwise}{}\<[36]%
\>[36]{}\mathrel{=}\Varid{ldpf}\;\Varid{ps}\;\Varid{n}{}\<[E]%
\\[\blanklineskip]%
\>[5]{}\Varid{primes1}\mathbin{::}[\mskip1.5mu \Conid{Integer}\mskip1.5mu]{}\<[E]%
\\
\>[5]{}\Varid{primes1}\mathrel{=}\mathrm{2}\mathbin{:}\Varid{filter}\;\Varid{prime}\;[\mskip1.5mu \mathrm{3}\mathinner{\ldotp\ldotp}\mskip1.5mu]{}\<[E]%
\\[\blanklineskip]%
\>[5]{}\Varid{prime}\mathbin{::}\Conid{Integer}\to \Conid{Bool}{}\<[E]%
\\
\>[5]{}\Varid{prime}\;\Varid{n}{}\<[14]%
\>[14]{}\mid \Varid{n}\mathbin{<}\mathrm{1}{}\<[27]%
\>[27]{}\mathrel{=}\Varid{error}\;\text{\tt \char34 not~a~positive~integer\char34}{}\<[E]%
\\
\>[14]{}\mid \Varid{n}\equiv \mathrm{1}{}\<[27]%
\>[27]{}\mathrel{=}\Conid{False}{}\<[E]%
\\
\>[14]{}\mid \Varid{otherwise}{}\<[27]%
\>[27]{}\mathrel{=}\Varid{ldp}\;\Varid{n}\equiv \Varid{n}{}\<[E]%
\ColumnHook
\end{hscode}\resethooks

\problem{1.24} What happens when you modify the defining equation of \text{\tt ldp} to \text{\tt ldp~\char61{}~ldpf~primes1}?
Nothing. It's just in point-free form.

\chapter{Talking about Mathematical Objects}
\section*{Making Symbolic Form Explicit} % (fold)
\label{sec:making_symbolic_form_explicit}

%unfinished
\begin{Exercise} [number=22]
  \begin{stmt} \label{stmt:DensityOfRationals}
    Between every two rational numbers there is a third one.
  \end{stmt}

  Can you think of an argument showing that \ref{stmt:DensityOfRationals} is true?
\end{Exercise}
\begin{Answer}
  a = i/j
  b = m/n
  a and b are rationals
  i,j,m and n are integers
  suppose c = (2i - 1) / j
  suppose d = b + c

  avg of a + b = ((i*n)(j*m))/(j*n)*2
  a + b / 2 is a rational?

  prove: $\forall{x,y} \in \rationals \colon \quad \exists{z} \in \rationals \suchthat x < z < y$
\end{Answer}

\begin{Exercise}
  Give structure trees of the following formulas:
  \Question $\forall{x} \Bigl( A(x) \implies \bigl(B(x) \implies C(x) \bigr)\Bigr)$
  
  \Question $\exists{x} \bigl( A(x) \land B(x) \bigr)$
   
  \Question $\exists{x} A(x) \land \exists{x} B(x)$
\end{Exercise}
\begin{ExerciseList}[start=1]
  \Exercise $\forall{x} \Bigl( A(x) \implies \bigl(B(x) \implies C(x) \bigr)\Bigr)$
  \Answer \Tree [.{$\forall{x} \Bigl( A(x) \implies \bigl(B(x) \implies C(x) \bigr)\Bigr)$}
                 [.{$A(x) \implies \bigl( B(x) \implies C(x) \bigr)$}
                  [.{$A(x)$} ]
                  [.{$B(x) \implies C(x)$} 
                   {$B(x)$} 
                   {$C(x)$} 
                  ]
                 ]
                ]

  \Exercise $\exists{x} \bigl( A(x) \land B(x) \bigr)$
  \Answer \Tree [.{$\exists{x} \suchThat \bigl( A(x) \land B(x) \bigr)$}
                 [.{$A(x) \land B(x)$}
                  {$A(x)$}
                  {$B(x)$} ] ]

  \Exercise $\exists{x} A(x) \land \exists{x} B(x)$
  \Answer \Tree [.{$\exists{x} A(x) \land \exists{x} B(x)$}
                 [.{$\exists{x} A(x)$} 
                  {$A(x)$} ]
                 [.{$\exists{x} B(x)$} 
                  {$B(x)$} ] ]
\end{ExerciseList}

\begin{ExerciseList}
  Write as formulas with restricted quantifiers:
  \Exercise $\exists{x} \suchThat \bigl( \exists{y} \suchThat x \in \rationals \land y \in \rationals \land x < y \bigr)$
  \Answer $\exists{x,y} \in \rationals \suchthat x < y$
  \Exercise $\forall{x} \bigl( x \in \rationals \implies \exists{y} \suchThat y \in \rationals \land x < y \bigr) $
  \Answer $\forall{x} \in \reals \left( \exists{y} \in \reals \suchthat x < y \right)$
  \Exercise $\forall{x} \bigl( x \in \integers \implies \exists{m,n} \left( m \in \naturals \land n \in \naturals \land x = m - n \right) \bigr)$
  \Answer $\forall{x} \in \integers %
              \left( %
                  \exists{m,n} \in \naturals \suchthat x = m - n %
              \right)$
\end{ExerciseList}

%unanswered
\begin{Exercise} [number=31]
  Translate into formulas, taking care to express the intended meaning:

  \Question The equation $x^2 + 1 = 0$ has a solution.

  \Question A largest natural number does not exist.

  \Question The number 13 is prime (use $d \mid n$ for `$d$ divides $n$').

  \Question The number $n$ is prime.

  \Question There are infinitely many primes.

\end{Exercise}

\begin{Answer} [number=31.1]
\begin{displaymath}
\exists{x} \suchThat x = \pm \sqrt{-1}
\end{displaymath}
\end{Answer}


\begin{Answer} [number=31.2]
\begin{displaymath}
\neg \exists{m \in \naturals} \suchthat \forall{n \in \naturals} \colon m \geq n
\end{displaymath}
\end{Answer}

\begin{Answer}
  \begin{displaymath}
    \nexists{n \in \naturals} \suchthat n > 1 \land n \divides 13
  \end{displaymath}
\end{Answer}


%unanswered
\begin{Exercise} [number=32]
  Translate into formulas [taking care to express the intended meaning]
  \Question Everyone loved Diana. (Use the expression $L(x,y)$ for: $x$ loved $y$, and the name $d$ for Diana)
  \Question Diana loved everyone.
  \Question Man is mortal. (Use $M(x)$ for `$x$ is a man', and $M\prime(x)$ for `$x$ is mortal'.)
  \Question Some birds do not fly. (Use $B(x)$ for `$x$ is a bird' and $F(x)$ for `$x$ can fly'.)
\end{Exercise}

%unanswered
\begin{Exercise} [number=33]
  Translate into formulas, using appropriate expressions for the predicates:
  \Question Dogs that bark do not bite.
  \Question All that glitters is not gold.
  \Question Friends of Diana's friends are her friends.
  \Question The limit of $\frac{1}{n}$ as $n$ approaches infinity is zero.
\end{Exercise}

%unanswered
\begin{Exercise} [number=34]
  \Question Everyone loved Diana except Charles.
  \Question Every man adores at least two women.
  \Question No man is married to more than one woman.
\end{Exercise}

%unanswered
\begin{Exercise} [number=35]
  \Question The King is not raging.
  \Question The King is loved by all his subjects. (use $K(x)$ for `$x$ is a King', and $S(x,y)$ for `$x$ is a subject of $y$').
\end{Exercise}

%unanswered
\begin{Exercise} [number=36]
  \Question $\exists{x} \in \rationals \suchThat x^2 = 5$
  \Question $\forall{n} \in \naturals \colon \exists{m} \in \naturals \suchThat n < m$
  \Question \[\forall{n} \in \naturals 
              \nexists{d} \in \naturals \suchThat 
                 1 < d < (2^n + 1) \land d \divides (2^n + 1)
            \]
  \Question $\forall{n} \in \naturals \exists{m} \in \naturals \suchThat \left( n < m \land \forall{p} \in \naturals \left(p \leq n \lor m \leq p \right)\right)$
  \Question $\forall{\varepsilon} \in \reals^{+} 
             \Bigl(
                \exists{n} \in \naturals \suchThat
                    \forall{m} \in \naturals
                    \bigl( 
                        m \geq n \implies 
                        \left(
                            \abs{a - a_m} \leq \varepsilon 
                        \right) 
                    \bigr) 
             \Bigr)$
            ($a$, $a_0$, $a_1$ refer to real numbers)
\end{Exercise}
  
% section making_symbolic_form_explicit (end)
\section{Abstract Formulas and Concrete Structures}

%unanswered
\begin{Exercise} [number=37]
Consider the following formulas:

Are these formulas true or false in the following contexts?
\end{Exercise}

%unanswered
\begin{Exercise} [number=38]
In Exercise 37, delete the first quantifier on $x$ in the five formulas. Determine for which values of $x$ the resulting open formulas are satisfied in each of the structures.
\end{Exercise}
\section{Logical Handling of the Quantifiers}

%unanswered
\begin{Exercise} [number=39]
    (the propositional equivalent of this was in Exercise 2.19)
    Argue that $\Phi$ and $\Psi$ are equivalent iff $\Phi \iff \Psi$ is valid.
\end{Exercise}

%unanswered
\begin{Exercise} [number=39]
    For every sentence $\Phi$ in exercise 2.36, consider its negation $\neg\Phi$, and produce a more positive equivalent by working the negation symbol through the quantifiers.
\end{Exercise}

%unanswered
\begin{Exercise} [number=46]
    Does it hold that $\neg\exists{x} \in A \suchthat \Phi(x)$ is equivalent to $\exists{x} \notin A \suchthat \Phi(x)$? If your answer is `yes', give a proof; if `no', then you should show this by giving a simple refutation (an example of formulas and structures where the two formulas have different truth values).
\end{Exercise}

%unanswered
\begin{Exercise} [number=47]
    Is $\exists{x} \notin A \suchthat \Phi(x)$ equivalent to $\exists{x} \in A \suchthat \neg\Phi(x)$? Give a proof if your answer is `yes', and a refutation otherwise.
\end{Exercise}

%unanswered
\begin{Exercise} [number=48]
    Produce the version of Theorem 2.40 that employs restricted quantification. Argue that your version is correct.
\end{Exercise}

%unanswered
\begin{Exercise} [number=50]
    That the sequence $a_0, a_1, a_2, \dots \in \reals$ converges to $a$, i.e., that $lim_{n \to \infty} a_n = a$, means that $forall{\delta} > 0 \colon \exists{n} \suchthat \forall{m \geq n} \colon \abs{a - a_M} < \delta$. Give a possible equivalent for the statement that the sequence $a_0, a_1, a_2, \dots \in \reals$ does not converge.
\end{Exercise}
\section{Quantifiers as Procedures}
\label{sec:Quantifiers_as_procedures}

\begin{hscode}\SaveRestoreHook
\column{B}{@{}>{\hspre}l<{\hspost}@{}}%
\column{8}{@{}>{\hspre}l<{\hspost}@{}}%
\column{E}{@{}>{\hspre}l<{\hspost}@{}}%
\>[B]{}\Varid{every},\Varid{some}\mathbin{::}[\mskip1.5mu \Varid{a}\mskip1.5mu]\to (\Varid{a}\to \Conid{Bool})\to \Conid{Bool}{}\<[E]%
\\
\>[B]{}\Varid{every}\;{}\<[8]%
\>[8]{}\Varid{xs}\;\Varid{p}\mathrel{=}\Varid{all}\;\Varid{p}\;\Varid{xs}{}\<[E]%
\\
\>[B]{}\Varid{some}\;{}\<[8]%
\>[8]{}\Varid{xs}\;\Varid{p}\mathrel{=}\Varid{any}\;\Varid{p}\;\Varid{xs}{}\<[E]%
\ColumnHook
\end{hscode}\resethooks

\begin{Exercise} [number=51]
  Define a function
\begin{hscode}\SaveRestoreHook
\column{B}{@{}>{\hspre}l<{\hspost}@{}}%
\column{E}{@{}>{\hspre}l<{\hspost}@{}}%
\>[B]{}\Varid{unique}\mathbin{::}(\Varid{a}\to \Conid{Bool})\to [\mskip1.5mu \Varid{a}\mskip1.5mu]\to \Conid{Bool}{}\<[E]%
\ColumnHook
\end{hscode}\resethooks
  that gives \text{\tt True} for \text{\tt unique~p~xs} just in case there is exactly
  one object among \text{\tt xs} that satisfies \text{\tt p}.
\end{Exercise}

\begin{Answer}
\begin{hscode}\SaveRestoreHook
\column{B}{@{}>{\hspre}l<{\hspost}@{}}%
\column{5}{@{}>{\hspre}l<{\hspost}@{}}%
\column{18}{@{}>{\hspre}l<{\hspost}@{}}%
\column{E}{@{}>{\hspre}l<{\hspost}@{}}%
\>[B]{}\Varid{unique}\;\Varid{p}\;\Varid{xs}\mathrel{=}\Varid{length}\;(\Varid{filter}\;\Varid{p}\;\Varid{xs})\equiv \mathrm{1}{}\<[E]%
\\[\blanklineskip]%
\>[B]{}\Varid{none}\mathbin{::}[\mskip1.5mu \Varid{a}\mskip1.5mu]\to (\Varid{a}\to \Conid{Bool})\to \Conid{Bool}{}\<[E]%
\\
\>[B]{}\Varid{none}\;\Varid{xs}\;\Varid{p}\mathrel{=}\neg \;(\Varid{some}\;\Varid{xs}\;\Varid{p}){}\<[E]%
\\[\blanklineskip]%
\>[B]{}\Varid{unique'}\mathbin{::}(\Varid{a}\to \Conid{Bool})\to [\mskip1.5mu \Varid{a}\mskip1.5mu]\to \Conid{Bool}{}\<[E]%
\\
\>[B]{}\Varid{unique'}\;\Varid{p}\;[\mskip1.5mu \mskip1.5mu]{}\<[18]%
\>[18]{}\mathrel{=}\Varid{false}{}\<[E]%
\\
\>[B]{}\Varid{unique'}\;\Varid{p}\;(\Varid{x}\mathbin{:}\Varid{xs}){}\<[E]%
\\
\>[B]{}\hsindent{5}{}\<[5]%
\>[5]{}\mid \Varid{p}\;\Varid{x}{}\<[18]%
\>[18]{}\mathrel{=}\Varid{none}\;\Varid{xs}\;\Varid{p}{}\<[E]%
\\
\>[B]{}\hsindent{5}{}\<[5]%
\>[5]{}\mid \Varid{otherwise}{}\<[18]%
\>[18]{}\mathrel{=}\Varid{unique'}\;\Varid{p}\;\Varid{xs}{}\<[E]%
\ColumnHook
\end{hscode}\resethooks
\end{Answer}

\begin{Exercise}
Define a function
\begin{hscode}\SaveRestoreHook
\column{B}{@{}>{\hspre}l<{\hspost}@{}}%
\column{E}{@{}>{\hspre}l<{\hspost}@{}}%
\>[B]{}\Varid{parity}\mathbin{::}[\mskip1.5mu \Conid{Bool}\mskip1.5mu]\to \Conid{Bool}{}\<[E]%
\ColumnHook
\end{hscode}\resethooks
that gives \text{\tt True} for \text{\tt parity~xs} just in case an even number of the \text{\tt xs}s equals \text{\tt True}.
\end{Exercise}

\begin{Answer}
\begin{hscode}\SaveRestoreHook
\column{B}{@{}>{\hspre}l<{\hspost}@{}}%
\column{3}{@{}>{\hspre}l<{\hspost}@{}}%
\column{E}{@{}>{\hspre}l<{\hspost}@{}}%
\>[B]{}\Varid{parity}\mathrel{=}\Varid{reduce}\;(\not\equiv )\;\Conid{True}{}\<[E]%
\\
\>[B]{}\hsindent{3}{}\<[3]%
\>[3]{}\mathbf{where}\;\Varid{reduce}\mathrel{=}\Varid{foldl}{}\<[E]%
\ColumnHook
\end{hscode}\resethooks
An empty list has 0 \text{\tt True}s. Since 0 is an even number, the ``initial'' answer
(or ``base case''?) should be \text{\tt True}.

I use \text{\tt reduce} because I'm not sure whether \text{\tt foldl} or \text{\tt foldr} is more
appropriate here.
\end{Answer}

\begin{Exercise}
Define a function
\begin{hscode}\SaveRestoreHook
\column{B}{@{}>{\hspre}l<{\hspost}@{}}%
\column{E}{@{}>{\hspre}l<{\hspost}@{}}%
\>[B]{}\Varid{evenNR}\mathbin{::}(\Varid{a}\to \Conid{Bool})\to [\mskip1.5mu \Varid{a}\mskip1.5mu]\to \Conid{Bool}{}\<[E]%
\ColumnHook
\end{hscode}\resethooks
that gives \text{\tt True} for \text{\tt evenNR~p~xs} just in case an even number of the \text{\tt xs}s have property \text{\tt p}. (Use the \text{\tt parity} function from the previous exercise.)
\end{Exercise}

\begin{Answer}
\begin{hscode}\SaveRestoreHook
\column{B}{@{}>{\hspre}l<{\hspost}@{}}%
\column{E}{@{}>{\hspre}l<{\hspost}@{}}%
\>[B]{}\Varid{evenNR}\;\Varid{p}\mathrel{=}\Varid{parity}\mathbin{\circ}\Varid{map}\;\Varid{p}{}\<[E]%
\ColumnHook
\end{hscode}\resethooks
\end{Answer}

\chapter{The Use of Logic: Proof}
\section{Proof Style}
\section{Proof Recipes}
\section{Rules for the Connectives}

\begin{Exercise} [number=2]
  Apply both implication rules to prove $P \implies R$ from the givens $P \implies Q$, $P \implies (Q \implies R)$.
\end{Exercise}
\begin{Answer}
  \begin{structured_derivation}
    \task{Prove that $P \implies R$ when:}
    \assumption[FACT:P->Q]{$P \implies Q$}
    \assumption[FACT:P->Q->R]{$P \implies (Q \implies R)$}
    \isProvedBy{let us assume that $P$ is true. If we can then also show that $R$ is true, then the implication holds.}
    \begin{nested_derivation}
      \task{Prove that $Q \implies R$ is true when:}
      \assumption[FACT:P]{$P$ is true}
      \observation [FACT:Q->R]
        {\term{modus ponens} with \ref{FACT:P->Q->R} and \ref{FACT:P} }
        {$Q \implies R$}
      \observation[FACT:Q]
        {\term{modus ponens} with \ref{FACT:P->Q} and \ref{FACT:P} }
        {$Q$}
      \observation[FACT:R]
        {\term{modus ponens} with \ref{FACT:Q} and \ref{FACT:Q->R} }
        {$R$}
    \end{nested_derivation}
  \end{structured_derivation}
\end{Answer}

\begin{Exercise} [number=4]
  \noindent
  Assume that $n,m \in \naturals$. \\
  Show: ($m$ is odd $\land$ $n$ is odd) $\implies$ $m + n$ is even.
\end{Exercise}
\begin{Answer}  
  \begin{structured_derivation}
    \task{Prove that ($m$ is odd $\land$ $n$ is odd) $\implies m + n$ is even when:}
    \task{Prove that $m + n$ is necessarily even if $m$ and $n$ are both odd.}
    \assumption*{$n \in \naturals$}
    \assumption*{$m \in \naturals$}
    \isProvedBy{If the antecedent isn't true, then the implication is \term{vacuously true}. So in order to prove the implication let's assume the antecedent is true and then demonstrate that the consequent is true.}
    \begin{nested_derivation}
      \task{Prove that $m + n$ is even when:}
      \assumption{$m$ is odd}
      \assumption{$n$ is odd}
      \observation
        {definition of odd-ness}
        {$\exists{x} \in \integers \suchthat m = 2x + 1$}
      \observation
        {definition of odd-ness}
        {$\exists{y} \in \integers \suchthat n = 2y + 1$}
      \isProvedBy{formalizing the definition of ``even'' to make it amenable to proof}
      \begin{nested_derivation}
        \task{Prove that $\exists{i} \in \integers \suchthat m + n = 2i $}
        $m + n$\\
        $(2x + 1) + (2y + 1)$\\
        $2x + 2y + 2$\\
        $2(x + y + 1)$\\
        % (x + y + 1) is a witness for i?
        \begin{nested_derivation}
          \task{Prove that $x + y + 1$ is an integer, and is therefore a witness of i (??)}
        \end{nested_derivation}
      \end{nested_derivation}
    \end{nested_derivation}
  \end{structured_derivation}
\end{Answer}

\begin{Exercise} [number=5]
  Show:
  \Question{From $P \iff Q$ it follows that $(P \implies R) \iff (Q \implies R)$,}
  \Question{From $P \iff Q$ it follows that $(R \implies P) \iff (R \implies Q)$.}
\end{Exercise}

\begin{Answer} [number=5.1]
  \begin{structured_derivation}
    \task{Prove that $(P \implies R) \iff (Q \implies R)$ when:}
    \assumption*{$P \iff Q$}
    \isProvedBy{show that each implies the other}
    \begin{nested_derivation}
      \task{Prove $Q \implies R$ when:}
      \assumption*{$P \implies R$}
      \isProvedBy{deduction rule}
      \begin{nested_derivation}
        \task{Prove $R$ when:}
        \assumption*{$Q$}
        \observation*
          {\term{modus ponens}}
          {$P$}
        \observation*
          {\term{modus ponens}}
          {$R$}
        \qed
      \end{nested_derivation}
    \end{nested_derivation}
    
    \begin{nested_derivation}
      \task{Prove $P \implies R$ when:}
      \assumption*{$Q \implies R$}
      \isProvedBy{deduction rule}
      \begin{nested_derivation}
        \task{Prove $R$ when:}
        \assumption*{$P$}
        \observation*
          {\term{modus ponens}}
          {$Q$}
        \observation*
          {\term{modus ponens}}
          {$R$}
        \qed
      \end{nested_derivation}
    \end{nested_derivation}
  \end{structured_derivation}
\end{Answer}

\begin{Answer} [number=5.2]
  \begin{structured_derivation}
    \task{Prove $(R \implies P) \iff (R \implies Q)$ when:}
    \assumption[PiffQ]{$P \iff Q$}
    \isProvedBy{show that each implies the other to demonstrate equivalence}
    \begin{nested_derivation}
      \task{Prove that $R \implies P$ when:}
      \assumption*{$R \implies Q$}
      \isProvedBy{deduction rule}
      \begin{nested_derivation}
        \task{Prove that $P$ when:}
        \assumption*{$R$}
        \observation*
          {\term{modus ponens}}
          {$Q$}
        \observation*
          {\term{modus ponens} / Elimination Rule}
          {$P$}
      \end{nested_derivation}
    \end{nested_derivation}
    \begin{nested_derivation}
      \task{Prove that $R \implies Q$ when:}
      \assumption*{$R \implies P$}
      \isProvedBy{deduction rule}
      \begin{nested_derivation}
        \task{Prove $Q$ when:}
        \assumption*{$R$}
        \observation*
          {\term{modus ponens}}
          {$P$}
        \observation*
          {\term{modus ponens}}
          {$Q$}
      \end{nested_derivation}
    \end{nested_derivation}
  \end{structured_derivation}
\end{Answer}

\begin{Exercise} [number=7]
  Produce proofs for:
  \Question Given: $P \implies Q$. To show: $\neg{Q} \implies \neg{P}$,
  \Question Given: $P \iff Q$. To show: $\neg{P} \iff \neg{Q}$.
\end{Exercise}

\begin{Answer} [number=7.1]
  \begin{structured_derivation}
    \task{Prove that $\neg{Q} \implies \neg{P}$ when:}
    \assumption{$P \implies Q$}
    \isProvedBy{deduction rule}
    \begin{nested_derivation}
      \task{Prove $\neg{P}$ when:}
      \assumption{$\neg{Q}$}
      \isProvedBy{assume the opposite and then derive $\bot$}
      \begin{nested_derivation}
        \task{Prove ?? \#\# Q when:}
        \assumption{$P$}
        \observation
          {\term{modus ponens}}
          {$Q$}
        \observation
          {$Q$ and $\neg{Q}$}
          {$\bot$}
      \end{nested_derivation}
      \andbutso{$\neg{P}$}      
    \end{nested_derivation}
  \end{structured_derivation}
\end{Answer}

\begin{Answer} [number=7.2]
  \begin{structured_derivation}
    \task{Prove that $\neg{P} \iff \neg{Q}$ when:}
    \assumption{$P \iff Q$}
    \isProvedBy{division into cases. Show that the antecedent/consequent relationship runs both ways to prove equivalence}
    \begin{nested_derivation}
      \task{Prove that $\neg{P} \implies \neg{Q}$}
      \isProvedBy{deduction rule}
      \begin{nested_derivation}
        \task{Prove $\neg{Q}$ provided that:}
        \assumption*{$\neg{P}$}
        \isProvedBy{assume the opposite and then derive $\bot$}
        \begin{nested_derivation}
          \task{Prove $P$ when:}
          \assumption*{$Q$}
          \observation*
            {\term{modus ponens} with root assumption}
            {$P$}
          contradiction!
        \end{nested_derivation}
        \andbutso{$\neg{Q}$}
      \end{nested_derivation}
    \end{nested_derivation}
    
    \begin{nested_derivation}
      \task{Prove that $\neg{Q} \implies \neg{P}$}
      \isProvedBy{deduction rule}
      \begin{nested_derivation}
        \task{Prove $\neg{P}$ when:}
        \assumption*{$\neg{Q}$}
        \isProvedBy{assume the opposite and then derive something false}
        \begin{nested_derivation}
          \task{Prove $Q$ supposing:}
          \assumption*{$P$}
          \observation*
            {\term{modus ponens} with root assumption}
            {$Q$}
        \end{nested_derivation}
        \andbutso{$\neg{P}$}
      \end{nested_derivation}
    \end{nested_derivation}
  \end{structured_derivation}
\end{Answer}



\begin{Exercise} [number=9, difficulty=1]
  Show that from $(P \implies Q) \implies P$ it follows that $P$.
  
  \ExeText Hint: Apply Proof by Contradiction. (The implication rules do not suffice for this admittedly exotic example.)
\end{Exercise}

\begin{Answer}
  \begin{structured_derivation}
    \task{Prove $P$ when:}
    \assumption*{$(P \implies Q) \implies P$}
    \isProvedBy{contradiction}
    \begin{nested_derivation}
      \task{Prove $\bot$ when:}
      \assumption*{$\neg{P}$}
      \observation
        {the antecedent is false so the implication is vacuously true}
        {$P \implies Q$}
      \observation
        {\term{modus ponens}}
        {$P$}
      \observation
        {law of the excluded middle; $P \land \neg{P}$}
        {$\bot$}      
    \end{nested_derivation}
  \end{structured_derivation}
\end{Answer}

\begin{theorem}
  From $P \lor Q$, $\neg{P}$ it follows that $Q$.
\end{theorem}

\begin{Exercise} [number=11]
  Assume that $A$, $B$, $C$, and $D$ are statements.
  \Question From the given $A \implies (B \lor C)$ and $B \implies \neg{A}$, derive that $A \implies C$.
  \Question From the given $(A \lor B) \implies (C \lor D)$, $C \implies A$, and $B \implies \neg{A}$, derive that $B \implies D$.
\end{Exercise}

\begin{Answer} [number=11.1]
  \begin{structured_derivation}
    \task{Derive $A \implies C$ when:}
    \assumption*{$B \implies \neg{A}$}
    \assumption*{$A \implies (B \lor C)$}
    \isProvedBy{deduction rule}
    \begin{nested_derivation}
      \task{Prove $C$ when:}
      \assumption*{$A$}
      \observation*
        {\term{modus ponens}}
        {$B \lor C$}
      \introbserv*{show that $B$ cannot be true}
      \begin{nested_derivation}
        \task{Derive $\bot$ when:}
        \assumption*{$B$}
        \observation*{\term{modus ponens}}
          {$\neg{A}$}
        \observation*{law of the excluded middle; $A \land \neg{A}$}
          {$\bot$}
      \end{nested_derivation}
      \andbutso{$\neg{B}$}
      \observation*
        {? by example 3.10}
        {$C$}
    \end{nested_derivation}
  \end{structured_derivation}
\end{Answer}

\begin{Answer} [number=11.2]
  \begin{structured_derivation}
    \task{Derive that $B \implies D$ when:}
    \assumption{$A \lor B \implies C \lor D$}
    \assumption{$C \implies A$}
    \assumption{$B \implies \neg{A}$}
    \isProvedBy{deduction rule}
    \begin{nested_derivation}
      \task{Prove $D$, assuming:}
      \assumption*{$B$}
      \observation
        {A disjunction follows from each of its disjuncts.}
        {$A \lor B$}
      \observation
        {\term{modus ponens}}
        {$C \lor D$}
      \observation
        {\term{modus ponens}}
        {$\neg{A}$}
      \observation
        {\term{modus tollens}}
        {$\neg{C}$}
      \observation
        {? by 3.10}
        {$D$}
    \end{nested_derivation}
  \end{structured_derivation}
\end{Answer}

\begin{Exercise} [number=15]
  Show that for any $n \in \naturals$, division of $n^2$ by 4 gives a remainder 0 or 1.
\end{Exercise}

\begin{Answer}
  
  % I don't want the supertext of a number to also be a number.
  \renewcommand{\thefootnote}{\fnsymbol{footnote}}
  
  % TODO also kinda like problem # 4, where I'm not sure how to formally discuss divisibility concisely.
  
  \begin{structured_derivation}
    \task{Prove that $\frac{n^2}{4}$ gives 0 or 1 as a remainder when:}
    \assumption*{$n \in \naturals$}
    \observation*
      {simple experimentation}
      {$9 \equiv 1 \imod{4}$}
    \observation*
      {simple experimentation}
      {$16 \equiv 0 \imod{4}$}
    \observation*
      {simple experimentation}
      {$25 \equiv 1 \imod{4}$}
    \isProvedBy{division into cases; every natural number is either even or odd}
    \begin{nested_derivation}
      \task{Prove that $\frac{n^2}{4}$ gives 0\footnotemark\ as a remainder when:}
      \assumption*{$n$ is even}
      \footnotetext{yes, yes, it's more proper to define the task as aiming for 0 \emph{or} 1, but this makes the \emph{intent} of the goal more clear. Besides, proving one of the conjuncts makes the entire expression true. \emph{boo-yah!}}
      \observation*
        {definition of what it means for a number to be even}
        {$\exists{k} \in \integers \suchthat n = 2k$}
      \introbserv*{figure out $n^2$ in terms of $k$}
      \begin{nested_derivation}
        \task{$n^2$}
        \justification[=]{substitute based on observation}
        \dijk{$(2k)^2$}
        \justification[=]{?}
        \dijk{$4 k^2$}
      \end{nested_derivation}
      \andbutso{$n^2 = 4 k^2$}
      now prove that 4 divides 4k\^{}2.\\
      Then prove that 4 nearly divides 4k\^{}2 + 4k + 1
    \end{nested_derivation}
  \end{structured_derivation}  
\end{Answer}

%unanswered
\begin{Exercise} [number=17]
  Prove the remaining items of THEOREM 2.10 (p. 45). To prove $\Phi \equiv \Psi$ means [something similar to proving set equivalence].
\end{Exercise}
  
\section{Rules for the Quantifiers}

\begin{Exercise}[number=18]
  Show, using $\forall$-introduction and Deduction Rule: if from $\Gamma$, $P(c)$ it follows that $Q(c)$ (where $c$ satisfies $P$, but is otherwise ``arbitrary''), then from $\Gamma$ it follows that $\forall{x} \colon P(x) \implies Q(x)$.
\end{Exercise}

\begin{Answer} [number=18]
  \begin{structured_derivation}
    \task{Prove $\Gamma \implies \bigl( \forall{x} \colon P(x) \implies Q(x) \bigr)$ provided that:}
    \assumption{$\bigl( P(c) \land \Gamma \bigr) \implies Q(c)$}
    \isProvedBy{deduction rule; suppose $\Gamma$ (because it's the antecedent)}
    \begin{nested_derivation}
      \task{Prove $\forall{x} \colon P(x) \implies Q(x)$ when:}
      \assumption{$\Gamma$}
      \isProvedBy{deduction rule}
      \begin{nested_derivation}
        \task{Prove $Q(x)$ when:}
        \assumption{$\exists{x} \suchthat P(x)$}
        \observation
          {A conjunction follows from its two conjuncts taken together.}
          { $\Gamma \land P(x)$ }
        \observation
          {\term{modus ponens}}
          { $Q(x)$ }
      \end{nested_derivation}
      \andbutso{$\forall{x} \colon P(x) \implies Q(x)$}
    \end{nested_derivation}
  \end{structured_derivation}
\end{Answer}
\section{Summary of the Proof Recipes}
\section{Some Strategic Guidelines}

%unanswered
\begin{Exercise} [number=25]
    Show:
    \Question from $\forall{x}\bigl(P(x) \implies Q(x)\bigr)$, $\forall{x} \colon P(x)$ it follows that $\forall{x} \colon Q(x)$,
    \Question from $\exists{x}\bigl(P(x) \implies Q(x)\bigr)$, $\forall{x} \colon P(x)$ it follows that $\exists{x} \suchthat Q(x)$.
\end{Exercise}

\begin{Answer} [number=25.1]
    \begin{structured_derivation}
        \task{Prove that $\forall{x} \colon Q(x)$ when:}
        \assumption*{$\forall{x} \colon P(x) \implies Q(x)$}
        \assumption*{$\forall{x} \colon P(x)$}
        \isProvedBy{??}
        \begin{nested_derivation}
            basically modus ponens but with quantifiers. 
        \end{nested_derivation}
    \end{structured_derivation}
\end{Answer}
\section{Reasoning and Computation with Primes}

%unanswered
\begin{Exercise} [number=34]
    \ExePart Let $A = \Set{4n + 3}{n \in \naturals}$. Show that $A$ contains infinitely many prime numbers.
    
    \ExeText Hint: any prime > 2 is odd, hence of the form $4n + 1$ or $4n + 3$. Assume that there are only finietly many primes of the form $4n + 3$, say $p_1, \dots p_m$. Consider the number $N = 4p_1 \dots p_m - 1 = 4(p_1 \dots p_m - 1) + 3$. Argue that $N$ must contain a factor $4q + 3$, using the fact that $(4a + 1)(4b + 1)$ is of the form $4c + 1$.
    
    \ExePart Use \text{\tt filter~prime~\char91{}~4\char42{}n~\char43{}~3~\char124{}~n~\char60{}\char45{}~\char91{}0\char46{}\char46{}\char93{}~\char93{}} to generate the primes of this form.
\end{Exercise}

%unanswered
\begin{Exercise} [number=36]
    It is not very difficult to show that if $n$ is composite, $M_n = 2^n - 1$ is composite too. Show this.
    \ExeText Hint: Assume that $n = ab$ and probe that $xy = 2^n - 1$ for the numbers $x = 2^b - 1$ and $y = 1 + 2^b + 2^{2b} + \cdots + 2^{(a-1)b}$.
\end{Exercise}

%unanswered
\begin{Exercise} [number=38]
A slightly faster way to generate the primes is by starting out from the odd numbers. The stepping and marking will work as before, for you count $k$ positions in the odd numbers starting from any odd number $a = 2n + 1$, you will move on o number $(2n + 1) + 2k$, and if $a$ is a multiile of $k$, then so is $a + 2k$. Implement a function \text{\tt fasterprimes~\char58{}\char58{}~\char91{}Integer\char93{}} using this idea. The odd natural numbers, staring from 3, can be generated as follows:
\begin{hscode}\SaveRestoreHook
\column{B}{@{}>{\hspre}l<{\hspost}@{}}%
\column{E}{@{}>{\hspre}l<{\hspost}@{}}%
\>[B]{}\Varid{oddsFrom3}\mathbin{::}[\mskip1.5mu \Conid{Integer}\mskip1.5mu]{}\<[E]%
\\
\>[B]{}\Varid{oddsFrom3}\mathrel{=}\mathrm{3}\mathbin{:}\Varid{map}\;(\mathbin{+}\mathrm{2})\;\Varid{oddsFrom3}{}\<[E]%
\ColumnHook
\end{hscode}\resethooks
\end{Exercise}

%unanswered
\begin{Exercise} [number=39]
Write a Haskell program to refute the following statement about prime numbers: ``if $p_1, \dots, p_k$ are all the primes $< n$, then $(p_1 \times \cdots \times p_k) + 1$ is a prime.''
\end{Exercise}

%unanswered
\begin{Exercise} [number=41]
How would you go about yourself to prove the fact Euclid proved? (if $2^n - 1$ is prime, then $2^{n-1}(2^n - 1)$ is perfect). Here is a hint: if $2^n - 1$ is prime, then the proper divisors of $2^{n-1}(2^n - 1)$ are
\begin{displaymath}
1, 2, 2^2, \dots, 2^{n-1}, 2^n - 1, 2(2^n - 1), 2^2(2^n - 1), \dots, 2^{n-2}(2^n - 1).
\end{displaymath}
\end{Exercise}

%unanswered
\begin{Exercise} [number=42]

\end{Exercise}

%unanswered
\begin{Exercise} [number=43]

\end{Exercise}



\chapter{Sets, Types and Lists}
\section{Let's Talk About Sets}

\begin{theorem}
  \label{THM:sets:RAT}
  For all sets $A$, $B$, and $C$, we have that:
  \begin{enumerate}
    \item ``reflexivity'': $A \subseteq A$ 
    \item ``antisymmetry'': $A \subseteq B \land B \subseteq A \implies A = B$ 
    \item ``transitivity'': $A \subseteq B \land B \subseteq C \implies A \subseteq C$ 
  \end{enumerate}
\end{theorem}

\begin{Exercise} [number=2]
  Show that the superset relation also has the properties of Theorem \ref{THM:sets:RAT}. I.e., show that~$\supseteq$ is reflexive, antisymmetric and transitive.
\end{Exercise}

\begin{Answer}
  $A \subseteq B \equiv B \supseteq A$
\end{Answer}

%unfinished
\begin{Exercise} [number=4]
  Show that $\bigl\{ \{ 1, 2 \}, \{ 0 \}, \{ 2, 1 \} \bigr\}  =  \bigl\{ \{ 0 \}, \{ 1, 2 \} \bigr\}$.
\end{Exercise}

\begin{Answer}
  \begin{structured_derivation}
    \task{Prove that $\bigl\{ \{ 1, 2 \}, \{ 0 \}, \{ 2, 1 \} \bigr\}  =  \bigl\{ \{ 0 \}, \{ 1, 2 \} \bigr\}$}
    \observation*
      {order and repetition in this notation are irrelevant}
      { $\{ 1, 2 \} \equiv \{ 2, 1 \} $ }
    \isProvedBy{demonstrating that each is a subset of the other}
    \begin{nested_derivation}
      \task{Prove that $\bigl\{ \{ 1, 2 \}, \{ 0 \}, \{ 2, 1 \} \bigr\} \subseteq \bigl\{ \{ 0 \}, \{ 1, 2 \} \bigr\}$.}
      
    \end{nested_derivation}
    \begin{nested_derivation}
      \task{Prove that $\bigl\{ \{ 1, 2 \}, \{ 0 \}, \{ 2, 1 \} \bigr\} \supseteq \bigl\{ \{ 0 \}, \{ 1, 2 \} \bigr\}$.}
    \end{nested_derivation}
  \end{structured_derivation}
\end{Answer}

%unanswered
\begin{Exercise} [number=7, difficulty=1]
  Assume that $A$ is a set of sets. Show that $\{ x \in A \suchthat x \notin x \} \notin A$.
\end{Exercise}

%answer as something to do with principle of extension. I think. Like,
%{x in A} = A, but not {x in A} in A. Well, UNLESS A can contain
%itself. If it can, then there exists x in A such that x = A. ...
\section{Paradoxes, Types and Type Classes}

\begin{hscode}\SaveRestoreHook
\column{B}{@{}>{\hspre}l<{\hspost}@{}}%
\column{10}{@{}>{\hspre}l<{\hspost}@{}}%
\column{23}{@{}>{\hspre}l<{\hspost}@{}}%
\column{E}{@{}>{\hspre}l<{\hspost}@{}}%
\>[B]{}\Varid{funny}\;\Varid{x}{}\<[10]%
\>[10]{}\mid \Varid{halts}\;\Varid{x}\;\Varid{x}{}\<[23]%
\>[23]{}\mathrel{=}\bot {}\<[E]%
\\
\>[10]{}\mid \Varid{otherwise}{}\<[23]%
\>[23]{}\mathrel{=}\Conid{True}{}\<[E]%
\ColumnHook
\end{hscode}\resethooks
So, does \text{\tt funny~funny} diverge or halt? IT'S A PARADOX!!

\begin{hscode}\SaveRestoreHook
\column{B}{@{}>{\hspre}l<{\hspost}@{}}%
\column{3}{@{}>{\hspre}l<{\hspost}@{}}%
\column{14}{@{}>{\hspre}l<{\hspost}@{}}%
\column{27}{@{}>{\hspre}l<{\hspost}@{}}%
\column{E}{@{}>{\hspre}l<{\hspost}@{}}%
\>[B]{}\Varid{halts}\;\Varid{f}\;\Varid{x}\mathrel{=}\Varid{f}\not\equiv \Varid{g}{}\<[E]%
\\
\>[B]{}\hsindent{3}{}\<[3]%
\>[3]{}\mathbf{where}\;\Varid{g}\;\Varid{y}{}\<[14]%
\>[14]{}\mid \Varid{y}\equiv \Varid{x}{}\<[27]%
\>[27]{}\mathrel{=}\bot {}\<[E]%
\\
\>[14]{}\mid \Varid{otherwise}{}\<[27]%
\>[27]{}\mathrel{=}\Varid{f}\;\Varid{y}{}\<[E]%
\ColumnHook
\end{hscode}\resethooks

\begin{hscode}\SaveRestoreHook
\column{B}{@{}>{\hspre}l<{\hspost}@{}}%
\column{12}{@{}>{\hspre}l<{\hspost}@{}}%
\column{22}{@{}>{\hspre}l<{\hspost}@{}}%
\column{E}{@{}>{\hspre}l<{\hspost}@{}}%
\>[B]{}\Varid{collatz}\mathbin{::}\Conid{Integer}\to [\mskip1.5mu \Conid{Integer}\mskip1.5mu]{}\<[E]%
\\
\>[B]{}\Varid{collatz}\;\Varid{n}{}\<[12]%
\>[12]{}\mid \Varid{n}\mathbin{<}\mathrm{1}{}\<[22]%
\>[22]{}\mathrel{=}\Varid{error}\;\text{\tt \char34 argument~not~positive\char34}{}\<[E]%
\\
\>[12]{}\mid \Varid{n}\equiv \mathrm{1}{}\<[22]%
\>[22]{}\mathrel{=}[\mskip1.5mu \mathrm{1}\mskip1.5mu]{}\<[E]%
\\
\>[12]{}\mid \Varid{even}\;\Varid{n}{}\<[22]%
\>[22]{}\mathrel{=}\Varid{n}\mathbin{:}\Varid{collatz}\;(\Varid{n}\mathbin{\Varid{`div`}}\mathrm{2}){}\<[E]%
\\
\>[12]{}\mid \Varid{odd}\;\Varid{n}{}\<[22]%
\>[22]{}\mathrel{=}\Varid{n}\mathbin{:}\Varid{collatz}\;(\mathrm{3}\mathbin{*}\Varid{n}\mathbin{+}\mathrm{1}){}\<[E]%
\ColumnHook
\end{hscode}\resethooks

\begin{Exercise} [number=8]
  Explain the following error message:
  
  \begin{tabbing}\tt
~~~~~Prelude\char62{}~elem~1~1\\
\tt ~~~~~ERROR\char58{}~\char91{}a\char93{}~is~not~an~instance~of~class~\char34{}Num\char34{}\\
\tt ~~~~~Prelude\char62{}
\end{tabbing}
\end{Exercise}

\begin{Answer}
  An intuitive explanation is that you cannot test anything for 
  being a ``member'' of a number.

  The function \ensuremath{\Varid{elem}} has type \ensuremath{\Conid{Eq}\;\Varid{a}\Rightarrow \Varid{a}\to [\mskip1.5mu \Varid{a}\mskip1.5mu]\to \Conid{Bool}}.
  However, \text{\tt \char40{}elem~1\char41{}~\char58{}\char58{}~Num~a~\char61{}\char62{}~\char91{}a\char93{}~\char45{}\char62{}~Bool}. This is because
  when \text{\tt elem} is partially applied to 1, Haskell determines that
  \text{\tt elem} will be operating on some \term{instance} of the \text{\tt Num}
  typeclass\footnote{however, since 1 can be an \text{\tt Int} \emph{or} an \text{\tt Integer}, it must stick with the more general \text{\tt Num}}. 
  As such, the type changes. The function produced by applying
  \text{\tt elem} to \text{\tt 1} can only be sensibly applied to a list of \text{\tt Num}s.
  What the function received was \emph{not} a list of some type \ensuremath{\Varid{a}} such
  that \ensuremath{\Varid{a}} was an instance of \text{\tt Num}, but rather another \text{\tt 1}. I think that
  the error is, in effect, saying that ``my expected parameter is not compatible
  with what you've supplied me''.
  
\end{Answer}
\section{Special Sets}

%unfinished
\begin{Exercise} [number=10]
  Show:
  \Question $\{ a \} = \{ b \} \textrm{ iff } a = b$,
  \Question $\{ a_1, a_2 \} = \{ b_1, b_2 \} \textrm{ iff: } a_1 = b_1 \land a_2 = b_2 \textrm{, or } a_1 = b_2 \land a_2 = b_1$.

  NOTE: these \emph{can} be formally answered!
\end{Exercise}

\begin{Answer} [number=10.1]
  Because $\{ a \}$ and $\{ b \}$ are both \term{singletons}, in order
  to test if the two sets are the same (by Extensionality and such) we
  only need to compare the element from one to the element from the
  other.
\end{Answer}

\begin{Exercise} [number=11]
  Explain that $\emptyset \neq \{\emptyset\}$. And that $\lbrace
  \emptyset \rbrace \neq \bigl\{\{ \emptyset \}\bigr\}$.
\end{Exercise}

\begin{Answer}
  First, remember the \term{Principle of Extensionality}: that a set
  is completely defined by its elements, and as such is equal to
  another set with the same elements.
	
  I'll start with showing why $\emptyset \neq \{\emptyset\}$. What are
  the contents of $\emptyset$? Nothing---which can also be seen in the
  fact that $\cardinality{\emptyset} = 0$.  In contrast,
  $\cardinality{ \{ \emptyset \} } = 1$ because it contains one
  element: the empty set!

  The same reasoning can be applied to the other two, because the
  first pair happen to be the respective contents of the second
  pair. Since sets are totally defined by their elements, and we've
  shown that the elements are not equal, then the respective enclosing
  sets are also not equal.
\end{Answer}
\section{Algebra of Sets}

\begin{Exercise} [number=13]
  What are the types of the set difference operator $\setminus$ and of the inclusion operator $\subseteq$?
\end{Exercise}

\begin{Answer}

  \\$\subseteq$ :: $s \rightarrow s \rightarrow t$\\
  $\setminus$ :: $s \rightarrow s \rightarrow s$
\end{Answer}


\begin{Exercise} [number=14]
  Give the types of the following expressions:
  \Question $x \in \Set{x}{E(x)}$.
  \Question $\Set{x}{E(x)}$.
  \Question $(A \intersect B) \subseteq C$.
  \Question $(A \union B) \intersect C$.
  \Question $\forall{x} \colon x \in A \implies x \in B$.
  \Question $A=B$.
  \Question $a \in A \iff a \in B$.
\end{Exercise}

\begin{Answer}
$t$, $s$, $t$, $s$, $t$, $t$, $t$
\end{Answer}

%unfinished
%TODO leave the set notation and dive into logical quantifiers and
%then do Dijkstra calculation
\begin{Exercise} [number=17]
  $A$, $B$ and $C$ are sets. Show:
  \Question $A \nsubseteq B \iff A - B \neq \emptyset$.
  \Question $A \intersect B = A - (A - B)$.
\end{Exercise}

\begin{Answer} [number=17.1]
  \begin{structured_derivation}
    \task{Show that $A \nsubseteq B$ when:}
    \assumption{$A - B \neq \emptyset$}
    \task{$A - B \neq \emptyset$}
    \justification[\iff]{Rewrite the set difference notation (?)}
    \dijk{$\Set{x}{x \in A \land x \notin B} \neq \emptyset$}
    \justification[\iff]{Rewrite the \dots}
    \dijk{$\exists{x \in A} \suchThat x \notin B$}
    \observation{$A-B$ can be rewritten as:}{}    
  \end{structured_derivation}
\end{Answer}


\begin{Answer} [number=17.2]
  \begin{structured_derivation}
    \task{Prove that $A \intersect B = A - (A-B)$}
    \isProvedBy{Demonstrate that the RHS is a subset of the LHS and vice-versa}
    \begin{nested_derivation}
      \task{Prove that $A \intersect B \subseteq A - (A-B)$}
      \isProvedBy{??}
      \begin{nested_derivation}
        \task{Prove that $x = y$ when:}
        \assumption{$\exists{x} \suchThat x \in (A \intersect B)$}
        \observation{definition of subset \& intersection of
          sets}{$\exists{x} \suchThat x \in A \land x \in B$}
        \assumption{$\exists{y} \suchThat y \in (A - (A-B))$}
        \observation{??}{$\exists{y \in A} \suchThat y \notin A-B$}
        
        
      \end{nested_derivation}
    \end{nested_derivation}
    \begin{nested_derivation}
      \task{Prove that $A - (A - B) \subseteq A \intersect B$}    
    \end{nested_derivation}
  \end{structured_derivation}
\end{Answer}

%TODO missed steps on commutativity
\begin{Exercise} [number=19]
  Express $(A \union B) \intersect (C \union D)$ as the union of four intersections.
\end{Exercise}

\begin{Answer}
  \begin{structured_derivation}
    \task{$(A \union B) \intersect (C \union D)$}
    \justification[\equiv]{distributivity}
    \dijk{$\bigl(A \intersect (C \union D)\bigr) \union \bigl(B
      \intersect (C \union D)\bigr)$}
    \justification[\equiv]{distributivity}
    \dijk{$\bigl( (A \intersect C) \union (A \intersect D) \bigr)
      \union \bigl(B \intersect (C \union D)\bigr)$}
    \justification[\equiv]{distributivity}
    \dijk{$\bigl( (A \intersect C) \union (A \intersect D) \bigr)
      \union \bigl( (B \intersect C) \union (B \intersect D) \bigr)$}
    \justification[\equiv]{trivial rewrite}
    \dijk{$\cumulUnion \bigl( (A \intersect C), (A \intersect D), (B
        \intersect C), (B \intersect D) \bigr)$}
  \end{structured_derivation}
\end{Answer}

%unanswered
\begin{Exercise} [number=21]
  Show that $A \xor B = (A - B) \union (B - A) = (A \union B) - (A \intersect B)$.
  
  ($A \xor B$ is the set of all objects that are in $A$ or $B$ but \emph{not both}.)
\end{Exercise}

\begin{Answer}
  \begin{structured_derivation}
    \task{see above (I'm getting lazy)}
    \observation{??}{$A \xor B \equiv \Set{x}{x \in A \xor x \in B}$}
  \end{structured_derivation}
\end{Answer}

%unanswered
%@next
\begin{Exercise} [number=23] 
  Let $X$ be a set with at least two elements. Then by Theorem \ref{THM:sets:RAT},
  the relation $\subseteq$ on $\powerset(X)$ has the properties of reflexivity,
  antisymmetry, and transitivity. The relation $\leq$ on $\reals$ also has these
  properties. The relation $\leq$ on $\reals$ has the further property of 
  \term{linearity}: for all $x, y \in \reals$, either $x \leq y$ or $y \leq x$.
  Show that $\subseteq$ on $\powerset(X)$ lacks this property.
\end{Exercise}

%TODO review the example just before this problem
%unanswered
\begin{Exercise} [number=26]
  Give a logical translation of $\cumulIntersect \mathcal{F} \subseteq
  \cumulUnion \mathcal{G}$ using only the relation $\in$.
\end{Exercise}

%unanswered
\begin{Exercise} [number=27]
  Let $\mathcal{F}$ be a family of sets. Show that there is a set $A$ with the 
  following properties:
  \Question $\mathcal{F} \subseteq \powerset(A)$.
  \Question For all sets $B$: if $\mathcal{F} \subseteq \powerset(B)$ then $A \subseteq B$.
\end{Exercise}

%unanswered
\begin{Exercise} [number=29]
  fill in and prove Theorem 4.20
\end{Exercise}

%unfinished: not transferred from notebook.
%TODO: make the powersets cascadingly smaller
\begin{Exercise} [number=30]
Answer as many of the following questions as you can.
\Question Determine: $\powerset(\emptyset)$, $\powerset(\powerset(\emptyset))$, $\powerset(\powerset(\powerset(\emptyset)))$
\Question How many elements has $\powerset^{5}(\emptyset)$?
\Question How many elements has $\powerset(A)$, given that $A$ as $n$ elements?
\end{Exercise}

%unfinished: need to LaTeX-ize and formalize/arrange argument from notebook.
\begin{Exercise} [number=31]
Check whether the following is true: if two sets have the same subsets, then they are equal. I.e.: if $\powerset(A) = \powerset(B)$, then $A = B$. Give a proof or a refutation by means of a counterexample.
\end{Exercise}

%unanswered
\begin{Exercise} [number=32]
Is it true that for all sets $A$ and $B$:
\Question $\powerset(A \intersect B) = \powerset(A) \intersect \powerset(B)$?
\Question $\powerset(A \union B) = \powerset(A) \union \powerset(B)$?
\ExeText Provide either a proof or a refutation by counter-example.
\end{Exercise}

%unanswered
\begin{Exercise} [number=33, difficulty=1]
  Show:
  \Question $B \intersect \bigl( \cumulUnion_{i \in I} A_i \bigr) =
  \cumulUnion_{i \in I}(B \intersect A_i)$
  
  \Question $B \union \bigl( \cumulIntersect_{i \in I} A_i \bigr) = \cumulIntersect_{i
    \in I}(B \union A_i)$
  
  \Question $\bigl(\cumulUnion_{i \in I} A_i \bigr)^c =
  \cumulIntersect_{i \in I} A_{i}^{c}$, assuming that $\forall{i} \in
  I \colon A_i \subseteq X$

  \Question $\bigl(\cumulIntersect_{i \in I} A_i \bigr)^c =
  \cumulUnion_{i \in I} A_{i}^{c}$, assuming that $\forall \in I
  \colon A_i \subseteq X$
\end{Exercise}

%unanswered
\begin{Exercise} [number=34, difficulty=1]
  Assume that you are given a certain set $A_0$. Suppose you are
  assigned the task of finding sets $A_1$, $A_2$, $A_3$, \dots, such
  that $\powerset(A_1) \subseteq A_0$, $\powerset(A_2) \subseteq A_1$,
  $\powerset(A_3) \subseteq A_2$, \dots Show that no matter how hard
  you try, you will eventually fail, that is: hit a set $A_n$ for
  which no $A_{n+1}$ exists such that $\powerset(A_{n+1}) \subseteq
  A_n$. (I.e., $\emptyset \notin A_n$.)

  \ExeText Suppose you can go on forever. Show this would entail
  $\powerset\bigl(\cumulIntersect_{i \in \naturals} A_i \bigr)
  \subseteq \cumulIntersect_{i \in \naturals} A_i$. Apply exercise 4.7.
\end{Exercise}

%unanswered
\begin{Exercise} [number=35, difficulty=1]
  Suppose that the collection $\mathcal{K}$ of sets satisfies the
  following condition:
  \begin{displaymath}
    \forall{A} \in \mathcal{K} \colon A = \emptyset \lor \exists{B \in
    \mathcal{K}} \suchthat A = \powerset(B)
  \end{displaymath}
  \ExeText Show that every element of $\mathcal{K}$ has the form
  $\powerset^n(\emptyset)$ for some $n \in \naturals$. (N.B.:
  $\powerset^0(\emptyset) = \emptyset$)
\end{Exercise}

%%% Local Variables:
%%% TeX-master: "exercises_base"
%%% End:
\section{Ordered Pairs and Products}

%unanswered
\begin{Exercise} [number=39]
fill out and prove the other items of Theorem 4.38
\end{Exercise}

%unanswered
\begin{Exercise} [number=40]
  \Question Assume that $A$ and $B$ are non-empty and that $A \times B = B \times A$. Show that $A = B$.
  \Question Show by means of an example that the condition of non-emptiness in 1 (the previous part/Question) is necessary. (Did you use this in your proof of 1?)
\end{Exercise}

%unanswered
\begin{Exercise} [number=41, difficulty=1]
  To show that defining $(a,b)$ as $\bigl\{ \{a\}, \{a,b\} \bigr\}$ works, prove that:
  \Question $\{a, b\} = \{a, c\} \quad \implies \quad b = c$.
  \Question $\bigl\{ \{a\}, \{a, b\}\bigr\} = \bigl\{ \{x\}, \{x, y\} \bigr\} \qquad \implies \qquad a = x \land b = y$.
\end{Exercise}
\section{Lists and List Operations}

%unanswered
\begin{Exercise} [number=43]
  How does it follow from this definition that lists of different length are unequal?
\end{Exercise}

%unanswered
\begin{Exercise} [number=44]
  Another ordering of lists is as follows: shorter lists come before
  longer ones, and for lists of the same length we compare their first
  elements, and if these are the same, the remainder lists. Give a
  formal definition of this ordering. How would you implement it in Haskell?
\end{Exercise}

\begin{Exercise} [number=45]
  Which operation on lists is specified by the Haskell definition in
  the frame below?
  \begin{hscode}\SaveRestoreHook
\column{B}{@{}>{\hspre}l<{\hspost}@{}}%
\column{5}{@{}>{\hspre}l<{\hspost}@{}}%
\column{18}{@{}>{\hspre}l<{\hspost}@{}}%
\column{E}{@{}>{\hspre}l<{\hspost}@{}}%
\>[5]{}\Varid{init}\mathbin{::}[\mskip1.5mu \Varid{a}\mskip1.5mu]\to [\mskip1.5mu \Varid{a}\mskip1.5mu]{}\<[E]%
\\
\>[5]{}\Varid{init}\;[\mskip1.5mu \Varid{x}\mskip1.5mu]{}\<[18]%
\>[18]{}\mathrel{=}[\mskip1.5mu \mskip1.5mu]{}\<[E]%
\\
\>[5]{}\Varid{init}\;(\Varid{x}\mathbin{:}\Varid{xs}){}\<[18]%
\>[18]{}\mathrel{=}\Varid{x}\mathbin{:}\Varid{init}\;\Varid{xs}{}\<[E]%
\ColumnHook
\end{hscode}\resethooks
\end{Exercise}

\begin{Answer}
  It's just like Clojure's/STk's \texttt{butlast} function.
\end{Answer}

\begin{Exercise} [number=46]
  Write your own definition of a Haskell operation \text{\tt reverse} that
  reverses a list.
\end{Exercise}

\begin{Answer}
\begin{hscode}\SaveRestoreHook
\column{B}{@{}>{\hspre}l<{\hspost}@{}}%
\column{3}{@{}>{\hspre}l<{\hspost}@{}}%
\column{10}{@{}>{\hspre}l<{\hspost}@{}}%
\column{13}{@{}>{\hspre}l<{\hspost}@{}}%
\column{26}{@{}>{\hspre}l<{\hspost}@{}}%
\column{E}{@{}>{\hspre}l<{\hspost}@{}}%
\>[B]{}\Varid{reverse}\mathbin{::}[\mskip1.5mu \Varid{a}\mskip1.5mu]\to [\mskip1.5mu \Varid{a}\mskip1.5mu]{}\<[E]%
\\
\>[B]{}\Varid{reverse}\;[\mskip1.5mu \mskip1.5mu]{}\<[13]%
\>[13]{}\mathrel{=}[\mskip1.5mu \mskip1.5mu]{}\<[E]%
\\
\>[B]{}\Varid{reverse}\;\Varid{l}{}\<[13]%
\>[13]{}\mathrel{=}\Varid{squoosh}\;\Varid{l}\;[\mskip1.5mu \mskip1.5mu]{}\<[E]%
\\
\>[B]{}\hsindent{3}{}\<[3]%
\>[3]{}\mathbf{where}\;{}\<[10]%
\>[10]{}\Varid{squoosh}\;(\Varid{x}\mathbin{:}\Varid{xs})\;{}\<[26]%
\>[26]{}\Varid{ans}\mathrel{=}\Varid{squoosh}\;\Varid{xs}\;(\Varid{x}\mathbin{:}\Varid{ans}){}\<[E]%
\\
\>[10]{}\Varid{squoosh}\;[\mskip1.5mu \mskip1.5mu]\;{}\<[26]%
\>[26]{}\Varid{ans}\mathrel{=}\Varid{ans}{}\<[E]%
\ColumnHook
\end{hscode}\resethooks
\end{Answer}

\begin{Exercise} [number=47]
  Write a function \text{\tt splitList} that gives all the ways to split a list
  of at least two elements in two non-empty parts. The type
  declaration is:
\begin{hscode}\SaveRestoreHook
\column{B}{@{}>{\hspre}l<{\hspost}@{}}%
\column{E}{@{}>{\hspre}l<{\hspost}@{}}%
\>[B]{}\Varid{splitList}\mathbin{::}[\mskip1.5mu \Varid{a}\mskip1.5mu]\to [\mskip1.5mu ([\mskip1.5mu \Varid{a}\mskip1.5mu],[\mskip1.5mu \Varid{a}\mskip1.5mu])\mskip1.5mu]{}\<[E]%
\ColumnHook
\end{hscode}\resethooks
  \ExeText the call \text{\tt splitList~\char91{}1\char46{}\char46{}4\char93{}} should give \text{\tt \char91{}\char40{}\char91{}1\char93{}\char44{}~\char91{}2\char44{}3\char44{}4\char93{}\char41{}\char44{}\char40{}\char91{}1\char44{}2\char93{}\char44{}\char91{}3\char44{}4\char93{}\char41{}\char44{}\char40{}\char91{}1\char44{}2\char44{}3\char93{}\char44{}\char91{}4\char93{}\char41{}\char93{}}.
\end{Exercise}

\begin{Answer}
\begin{hscode}\SaveRestoreHook
\column{B}{@{}>{\hspre}l<{\hspost}@{}}%
\column{3}{@{}>{\hspre}l<{\hspost}@{}}%
\column{E}{@{}>{\hspre}l<{\hspost}@{}}%
\>[B]{}\Varid{splitList}\;\Varid{xs}\mathrel{=}[\mskip1.5mu \Varid{splitAt}\;\Varid{i}\;\Varid{xs}\mid \Varid{i}\leftarrow [\mskip1.5mu \mathrm{1}\mathinner{\ldotp\ldotp}\Varid{n}\mskip1.5mu]\mskip1.5mu]{}\<[E]%
\\
\>[B]{}\hsindent{3}{}\<[3]%
\>[3]{}\mathbf{where}\;\Varid{n}\mathrel{=}(\Varid{length}\;\Varid{xs})\mathbin{-}\mathrm{1}{}\<[E]%
\ColumnHook
\end{hscode}\resethooks
\end{Answer}
\section{List Comprehension and Database Query}

\begin{hscode}\SaveRestoreHook
\column{B}{@{}>{\hspre}l<{\hspost}@{}}%
\column{13}{@{}>{\hspre}l<{\hspost}@{}}%
\column{20}{@{}>{\hspre}l<{\hspost}@{}}%
\column{43}{@{}>{\hspre}l<{\hspost}@{}}%
\column{E}{@{}>{\hspre}l<{\hspost}@{}}%
\>[B]{}\mathbf{import}\;\Conid{DB}\;(\Varid{db}){}\<[E]%
\\[\blanklineskip]%
\>[B]{}\Varid{characters}{}\<[13]%
\>[13]{}\mathrel{=}\Varid{nub}\;{}\<[20]%
\>[20]{}[\mskip1.5mu \Varid{x}\mid [\mskip1.5mu \text{\tt \char34 play\char34},\anonymous ,\anonymous ,\Varid{x}\mskip1.5mu]{}\<[43]%
\>[43]{}\leftarrow \Varid{db}\mskip1.5mu]{}\<[E]%
\\
\>[B]{}\Varid{movies}{}\<[13]%
\>[13]{}\mathrel{=}{}\<[20]%
\>[20]{}[\mskip1.5mu \Varid{x}\mid [\mskip1.5mu \text{\tt \char34 release\char34},\Varid{x},\anonymous \mskip1.5mu]{}\<[43]%
\>[43]{}\leftarrow \Varid{db}\mskip1.5mu]{}\<[E]%
\\
\>[B]{}\Varid{actors}{}\<[13]%
\>[13]{}\mathrel{=}\Varid{nub}\;{}\<[20]%
\>[20]{}[\mskip1.5mu \Varid{x}\mid [\mskip1.5mu \text{\tt \char34 play\char34},\Varid{x},\anonymous ,\anonymous \mskip1.5mu]{}\<[43]%
\>[43]{}\leftarrow \Varid{db}\mskip1.5mu]{}\<[E]%
\\
\>[B]{}\Varid{directors}{}\<[13]%
\>[13]{}\mathrel{=}\Varid{nub}\;{}\<[20]%
\>[20]{}[\mskip1.5mu \Varid{x}\mid [\mskip1.5mu \text{\tt \char34 direct\char34},\Varid{x},\anonymous \mskip1.5mu]{}\<[43]%
\>[43]{}\leftarrow \Varid{db}\mskip1.5mu]{}\<[E]%
\\
\>[B]{}\Varid{dates}{}\<[13]%
\>[13]{}\mathrel{=}\Varid{nub}\;{}\<[20]%
\>[20]{}[\mskip1.5mu \Varid{x}\mid [\mskip1.5mu \text{\tt \char34 release\char34},\anonymous ,\Varid{x}\mskip1.5mu]{}\<[43]%
\>[43]{}\leftarrow \Varid{db}\mskip1.5mu]{}\<[E]%
\\
\>[B]{}\Varid{universe}{}\<[13]%
\>[13]{}\mathrel{=}\Varid{nub}\;(\Varid{characters}\plus \Varid{actors}\plus \Varid{directors}\plus \Varid{movies}\plus \Varid{dates}){}\<[E]%
\ColumnHook
\end{hscode}\resethooks

Next, define lists of tuples, again by list comprehension:

(I tried to name the variables mnemonically)
\begin{hscode}\SaveRestoreHook
\column{B}{@{}>{\hspre}l<{\hspost}@{}}%
\column{10}{@{}>{\hspre}l<{\hspost}@{}}%
\column{23}{@{}>{\hspre}l<{\hspost}@{}}%
\column{42}{@{}>{\hspre}l<{\hspost}@{}}%
\column{E}{@{}>{\hspre}l<{\hspost}@{}}%
\>[B]{}\Varid{direct}{}\<[10]%
\>[10]{}\mathrel{=}[\mskip1.5mu (\Varid{d},\Varid{m}){}\<[23]%
\>[23]{}\mid [\mskip1.5mu \text{\tt \char34 direct\char34},\Varid{d},\Varid{m}\mskip1.5mu]{}\<[42]%
\>[42]{}\leftarrow \Varid{db}\mskip1.5mu]{}\<[E]%
\\
\>[B]{}\Varid{act}{}\<[10]%
\>[10]{}\mathrel{=}[\mskip1.5mu (\Varid{a},\Varid{m}){}\<[23]%
\>[23]{}\mid [\mskip1.5mu \text{\tt \char34 play\char34},\Varid{a},\Varid{m},\anonymous \mskip1.5mu]{}\<[42]%
\>[42]{}\leftarrow \Varid{db}\mskip1.5mu]{}\<[E]%
\\
\>[B]{}\Varid{play}{}\<[10]%
\>[10]{}\mathrel{=}[\mskip1.5mu (\Varid{a},\Varid{m},\Varid{c}){}\<[23]%
\>[23]{}\mid [\mskip1.5mu \text{\tt \char34 play\char34},\Varid{a},\Varid{m},\Varid{c}\mskip1.5mu]{}\<[42]%
\>[42]{}\leftarrow \Varid{db}\mskip1.5mu]{}\<[E]%
\\
\>[B]{}\Varid{release}{}\<[10]%
\>[10]{}\mathrel{=}[\mskip1.5mu (\Varid{m},\Varid{y}){}\<[23]%
\>[23]{}\mid [\mskip1.5mu \text{\tt \char34 release\char34},\Varid{m},\Varid{y}\mskip1.5mu]{}\<[42]%
\>[42]{}\leftarrow \Varid{db}\mskip1.5mu]{}\<[E]%
\ColumnHook
\end{hscode}\resethooks

Finally, define one placed, two placed and three placed predicates by
means of lambda abstraction.

\begin{hscode}\SaveRestoreHook
\column{B}{@{}>{\hspre}l<{\hspost}@{}}%
\column{13}{@{}>{\hspre}l<{\hspost}@{}}%
\column{E}{@{}>{\hspre}l<{\hspost}@{}}%
\>[B]{}\Varid{characterP}{}\<[13]%
\>[13]{}\mathrel{=}\lambda \Varid{x}\to \Varid{x}\in \Varid{characters}{}\<[E]%
\\
\>[B]{}\Varid{actorP}{}\<[13]%
\>[13]{}\mathrel{=}\lambda \Varid{x}\to \Varid{x}\in \Varid{actors}{}\<[E]%
\\
\>[B]{}\Varid{movieP}{}\<[13]%
\>[13]{}\mathrel{=}\lambda \Varid{x}\to \Varid{x}\in \Varid{movies}{}\<[E]%
\\
\>[B]{}\Varid{directorP}{}\<[13]%
\>[13]{}\mathrel{=}\lambda \Varid{x}\to \Varid{x}\in \Varid{directors}{}\<[E]%
\\
\>[B]{}\Varid{dateP}{}\<[13]%
\>[13]{}\mathrel{=}\lambda \Varid{x}\to \Varid{x}\in \Varid{dates}{}\<[E]%
\\
\>[B]{}\Varid{actP}{}\<[13]%
\>[13]{}\mathrel{=}\lambda (\Varid{x},\Varid{y})\to (\Varid{x},\Varid{y})\in \Varid{act}{}\<[E]%
\\
\>[B]{}\Varid{releaseP}{}\<[13]%
\>[13]{}\mathrel{=}\lambda (\Varid{x},\Varid{y})\to (\Varid{x},\Varid{y})\in \Varid{release}{}\<[E]%
\\
\>[B]{}\Varid{directP}{}\<[13]%
\>[13]{}\mathrel{=}\lambda (\Varid{x},\Varid{y})\to (\Varid{x},\Varid{y})\in \Varid{direct}{}\<[E]%
\\
\>[B]{}\Varid{playP}{}\<[13]%
\>[13]{}\mathrel{=}\lambda (\Varid{x},\Varid{y},\Varid{z})\to (\Varid{x},\Varid{y},\Varid{z})\in \Varid{play}{}\<[E]%
\ColumnHook
\end{hscode}\resethooks

%unanswered
\begin{ExerciseList}
  Translate the following into queries:
  \Exercise ``Give me the films in which Robert De Niro or Kevin
  Spacey acted.''
  \Exercise ``Give me all films with Quentin Tarantino as actor or
  director that appeared in 1994.''
  \Exercise ``Give me all films released after 1997 in which William
  Hurt did \emph{not} act.'' (emphasis mine)
\end{ExerciseList}
\section{Using Lists to Represent Sets}

\begin{hscode}\SaveRestoreHook
\column{B}{@{}>{\hspre}l<{\hspost}@{}}%
\column{5}{@{}>{\hspre}l<{\hspost}@{}}%
\column{18}{@{}>{\hspre}l<{\hspost}@{}}%
\column{E}{@{}>{\hspre}l<{\hspost}@{}}%
\>[B]{}\Varid{delete}\mathbin{::}\Conid{Eq}\;\Varid{a}\Rightarrow \Varid{a}\to [\mskip1.5mu \Varid{a}\mskip1.5mu]\to [\mskip1.5mu \Varid{a}\mskip1.5mu]{}\<[E]%
\\
\>[B]{}\Varid{delete}\;\Varid{x}\;[\mskip1.5mu \mskip1.5mu]{}\<[18]%
\>[18]{}\mathrel{=}[\mskip1.5mu \mskip1.5mu]{}\<[E]%
\\
\>[B]{}\Varid{delete}\;\Varid{x}\;(\Varid{y}\mathbin{:}\Varid{ys}){}\<[E]%
\\
\>[B]{}\hsindent{5}{}\<[5]%
\>[5]{}\mid \Varid{x}\equiv \Varid{y}{}\<[18]%
\>[18]{}\mathrel{=}\Varid{ys}{}\<[E]%
\\
\>[B]{}\hsindent{5}{}\<[5]%
\>[5]{}\mid \Varid{otherwise}{}\<[18]%
\>[18]{}\mathrel{=}\Varid{y}\mathbin{:}\Varid{delete}\;\Varid{x}\;\Varid{ys}{}\<[E]%
\ColumnHook
\end{hscode}\resethooks

\begin{hscode}\SaveRestoreHook
\column{B}{@{}>{\hspre}l<{\hspost}@{}}%
\column{5}{@{}>{\hspre}l<{\hspost}@{}}%
\column{9}{@{}>{\hspre}l<{\hspost}@{}}%
\column{18}{@{}>{\hspre}l<{\hspost}@{}}%
\column{19}{@{}>{\hspre}l<{\hspost}@{}}%
\column{26}{@{}>{\hspre}l<{\hspost}@{}}%
\column{E}{@{}>{\hspre}l<{\hspost}@{}}%
\>[B]{}\Varid{elem}\mathbin{::}\Conid{Eq}\;\Varid{a}\Rightarrow \Varid{a}\to [\mskip1.5mu \Varid{a}\mskip1.5mu]\to \Conid{Bool}{}\<[E]%
\\
\>[B]{}\Varid{x}\in [\mskip1.5mu \mskip1.5mu]{}\<[18]%
\>[18]{}\mathrel{=}\Conid{False}{}\<[E]%
\\
\>[B]{}\Varid{x}\in (\Varid{y}\mathbin{:}\Varid{ys}){}\<[E]%
\\
\>[B]{}\hsindent{5}{}\<[5]%
\>[5]{}\mid \Varid{x}\equiv \Varid{y}{}\<[18]%
\>[18]{}\mathrel{=}\Conid{True}{}\<[E]%
\\
\>[B]{}\hsindent{5}{}\<[5]%
\>[5]{}\mid \Varid{otherwise}{}\<[18]%
\>[18]{}\mathrel{=}\Varid{elem}\;\Varid{x}\;\Varid{ys}{}\<[E]%
\\[\blanklineskip]%
\>[B]{}\Varid{union}\mathbin{::}\Conid{Eq}\;\Varid{a}\Rightarrow [\mskip1.5mu \Varid{a}\mskip1.5mu]\to [\mskip1.5mu \Varid{a}\mskip1.5mu]\to [\mskip1.5mu \Varid{a}\mskip1.5mu]{}\<[E]%
\\
\>[B]{}[\mskip1.5mu \mskip1.5mu]{}\<[9]%
\>[9]{}\mathbin{`\Varid{union}`}\Varid{ys}\mathrel{=}\Varid{ys}{}\<[E]%
\\
\>[B]{}(\Varid{x}\mathbin{:}\Varid{xs}){}\<[9]%
\>[9]{}\mathbin{`\Varid{union}`}\Varid{ys}\mathrel{=}\Varid{x}\mathbin{:}\Varid{xs}\mathbin{`\Varid{union}`}(\Varid{delete}\;\Varid{x}\;\Varid{ys}){}\<[E]%
\\[\blanklineskip]%
\>[B]{}\Varid{intersect}\mathbin{::}\Conid{Eq}\;\Varid{a}\Rightarrow [\mskip1.5mu \Varid{a}\mskip1.5mu]\to [\mskip1.5mu \Varid{a}\mskip1.5mu]\to [\mskip1.5mu \Varid{a}\mskip1.5mu]{}\<[E]%
\\
\>[B]{}\Varid{intersect}\;[\mskip1.5mu \mskip1.5mu]\;\Varid{s}{}\<[19]%
\>[19]{}\mathrel{=}[\mskip1.5mu \mskip1.5mu]{}\<[E]%
\\
\>[B]{}\Varid{intersect}\;(\Varid{x}\mathbin{:}\Varid{xs})\;\Varid{s}{}\<[E]%
\\
\>[B]{}\hsindent{5}{}\<[5]%
\>[5]{}\mid \Varid{x}\in \Varid{s}{}\<[19]%
\>[19]{}\mathrel{=}\Varid{x}\mathbin{:}{}\<[26]%
\>[26]{}\Varid{xs}\mathbin{`\Varid{intersect}`}\Varid{s}{}\<[E]%
\\
\>[B]{}\hsindent{5}{}\<[5]%
\>[5]{}\mid \Varid{otherwise}{}\<[19]%
\>[19]{}\mathrel{=}{}\<[26]%
\>[26]{}\Varid{xs}\mathbin{`\Varid{intersect}`}\Varid{s}{}\<[E]%
\ColumnHook
\end{hscode}\resethooks

%unanswered
\begin{Exercise} [number=51]
  The Haskell operation for list difference is defined as \ensuremath{\mathbin{\char92 \char92 }} in
  \texttt{List.hs}. Write your own version of this.
\end{Exercise}

%unanswered
\begin{Exercise} [number=53]
  Write functions \ensuremath{\Varid{genUnion}} and \ensuremath{\Varid{genIntersect}} for generalized list
  union and list intersection. The functions should be of type
  \ensuremath{[\mskip1.5mu [\mskip1.5mu \Varid{a}\mskip1.5mu]\mskip1.5mu]\to [\mskip1.5mu \Varid{a}\mskip1.5mu]}. They take a list of lists as input and produce a
  list as output.

  Note that \ensuremath{\Varid{genIntersect}} is undefined on the empty
  list of lists (compare the remark about the presuposition of
  generalized intersection on page ONE HUNDRED THIRTY-FOUR).
\end{Exercise}
\section{A Data Type for Sets}

%unanswered
\begin{Exercise} [number=54]
  Give implementations of the operations % @TODO inline list
\end{Exercise}

%unanswered
\begin{Exercise} [number=55]
  In an implementation of sets as lists without duplicates, the
  implementation of \ensuremath{\Varid{insertSet}} has to be changed. How?
\end{Exercise}

%unanswered
\begin{Exercise} [number=56]
  What would have to change in the module \texttt{SetEq.hs} to get a
  representation of the empty set as $0$?
\end{Exercise}

%unanswered
\begin{Exercise} [number=57,difficulty=1]
  \Question How many pairs of curly braces $\lbrace \rbrace$ occur in
  the expanded notation for $\powerset^5(\emptyset)$, in the
  representation where $\emptyset$ appears as $\{\}$?

  \Question How many copies of $0$ occur in the expanded notation for
  $\powerset^5(\emptyset)$, in the representation where $\emptyset$
  appears as 0?
  % @TODO \ref{...} Exercise #56

  \Question
  How many pairs of curly braces occur in the expanded notation for
  $\powerset^5(\emptyset)$, in the representation where $\emptyset$
  appears as 0?
\end{Exercise}



%%% Local Variables: 
%%% mode: latex
%%% TeX-master: "exercises_base"
%%% End: 

\chapter{Relations}
\section{The Notion of a Relation}

\begin{Exercise} [number=13]
  Show that $\forall{x,y} \colon \exists{R} \suchThat xRy$. (``Between
  every two things there exist some relation''.)
\end{Exercise}

\begin{Answer}
  Well, suppose we have some arbitrary $x$ and $y$---well, there's
  nothing stopping us from creating an ordered pair $(x,y)$. Now
  observe that $\{(x,y)\}$ is a valid set. Thus, there is at least one
  set that contains an ordered pair comprising $x$ and $y$ as
  parts. Finally, note that $\{(x,y)\}$ satisfies the description of a
  relation. It's a set of ordered pairs. (The set's being a \term{singleton}
  is irrelevant in this circumstance.)
\end{Answer}
%%% Local Variables: 
%%% mode: latex
%%% TeX-master: "exercises_base"
%%% End: 
\section{Properties of Relations}

\newcommand{\R}{\mathrel{R}}

%unanswered
\begin{Exercise} [number=17]
  Show that a relation $R$ on $A$ is irreflexive iff $\Delta_A
  \intersect R = \emptyset$.
\end{Exercise}

\begin{Answer}
  \begin{structured_derivation}
    \task{Show that $\Delta_A \intersect R = \emptyset \iff R \textrm{
        is irreflexive}$ when:}
    \assumption{$R \subseteq A^2$}
    \isProvedBy{Derive each side of the equivalence and prove the
      other to prove the equivalence itself.}
    \begin{nested_derivation}
      \task{Show that $\Delta_A \intersect R = \emptyset$ when:}
      \assumption{$R$ is irreflexive}
      \observation{simple substitution based on definition of
        \term{irreflexive}}{$\forall{x \in A} \colon \neg(x \mathbin{R} x)$}
      \observation{reformulation, again based on
        definition}{$\nexists{x \in A} \suchThat (x,x) \in R$}
      
      
    \end{nested_derivation}

    \begin{nested_derivation}
      \task{Show that $R$ is irreflexive when:}
      \assumption{$\Delta_A \intersect R = \emptyset$}
      \observation{??}{
        \begin{nested_derivation}
          \task{$\Delta_A \intersect R = \emptyset$}
          \justification[\equiv]{Substitute by definition}
          \dijk{$\Set{(x,x)}{x \in A} \intersect R = \emptyset$}
          \justification[\equiv]{??}          
        \end{nested_derivation}
}
    \end{nested_derivation}
  \end{structured_derivation}
\end{Answer}

%unanswered
\begin{Exercise} [number=19]
  Show the following:
  \Question A relation $R$ on a set $A$ is symmetric iff $\forall{x,y
    \in A} \colon xRy \iff yRx$.
  \Question A relation $R$ is symmetric iff $R \subseteq R^{-1}$, iff
  $R = R^{-1}$.
\end{Exercise}

%unanswered
\begin{Exercise} [number=20]
  Show that every asymmetric relation is irreflexive.
\end{Exercise}

%unanswered
\begin{Exercise} [number=22]
  Show from the definitions that an asymmetric relation always is
  antisymmetric.
\end{Exercise}

%unanswered
\begin{Exercise} [number=23]
  Show that a relation $R$ on a set $A$ is transitive
  iff \[\forall{x,z \in A} \colon \left( \exists{y \in A} \suchThat
    a \R y \land y \R z \right) \implies x \R z\].
\end{Exercise}

%unanswered
\begin{Exercise} [number=28]
  Show that every strict partial order is asymmetric.
\end{Exercise}

%unanswered
\begin{Exercise} [number=29]
  Show that every relation which is transitive and asymmetric is a
  strict partial order.
\end{Exercise}

%unanswered
\begin{Exercise} [number=30]
  Show that if $R$ is a strict partial order on $A$, then $R \union
  \Delta_A$ is a partial order on $A$. (So every strict partial order
  is contained in a partial order.)
\end{Exercise}

%unanswered
\begin{Exercise} [number=31]
  Show that the inverse of a partial order is again a partial order.
\end{Exercise}

%unanswered
\begin{Exercise} [number=32]
  
  Let $S$ be a reflexive and symmetric relation on a set $A$.

  A path is a finite sequence $a_1, \dots, a_n$ of elements of $A$
  such that for every $i$, $1 \le i < n$, we have that
  $a_{i}\mathrel{S}a_{a+1}$. Such a path connects $a_1$ with $a_n$.
  
  Assume that for all $a$, $b \in A$ there is exactly one path
  connecting $a$ with $b$.

  Fix $r \in A$. Define the relation $\le$ on $A$ by: $a \le b$ iff
  $a$ is one of the elements in the path connecting $r$ with $b$.

  Show the following:
  \Question $\le$ is reflexive
  \Question $\le$ is antisymmetric
  \Question $\le$ is transitive
  \Question for all $a \in A$, $r \le a$
  \Question for every $a \in A$, the set $X_a = \Set{x \in A}{x \le
    a}$ is finite and if $b,c \in X_a$ then $b \le c$ or $c \le b$.

  (A structure $(A,\le,r)$ with these five properties is called a
  tree with root $r$. The directory-structure of a computer account is
  an example of a tree. Another example is the structure tree of a
  formula (see Section 2.3 above). These are example of finite trees,
  but if $A$ is infinite, then the tree $(A,\le,r)$ will have at
  least one infinite branch.)
\end{Exercise}

%unanswered
\begin{Exercise} [number=33]
  Consider the following relations on the natural numbers. Check their
  properties (some answers can be found in the text above). The
  \term{successor} relation is the relation given by
  $\Set{(n,m)}{n+1=m}$. The divisor relation is $\Set{(n,m)}{n
    \textrm{ divides } m}$. The \term{coprime} relation $C$ on
  $\naturals$ is given by $nCm :\equiv \operatorname{GCD}(n,m) = 1$,
  i.e., the only factor of $n$ that divides $m$ is 1, and vice versa.

  \dots (table) \dots
\end{Exercise}

%unanswered
\begin{Exercise} [number=35]
  Suppose that $R$ is a relation on $A$.
  \Question Show that $R \union \Delta_A$ is the reflexive closure of
  $R$.
  \Question Show that $R \union R^{-1}$ is the symmetric closure of $R$.
\end{Exercise}

%unanswered
\begin{Exercise} [number=36]
  Let $R$ be a transitive binary relation on $A$. Does it follow from
  the transitivity of $R$ that its symmetric reflexive closure $R
  \union R^{-1} \union \Delta_A$ is also transitive? Give a proof if
  your answer is yes, a counterexample otherwise.
\end{Exercise}

%unanswered
\begin{Exercise} [number=38]
  Determine the composition of the relation ``father of'' with
  itself. Determine the composition of the relations ``brother of''
  and ``parent of'' Give an example showing that $R \comp S = S \comp
  R$ can be false.
\end{Exercise}

%unanswered
\begin{Exercise} [number=39]
  Consider the relation \[R =
  \{(0,2),(0,3),(1,0),(1,3),(2,0),(2,3)\}\] on the set $A =
  \{0,1,2,3,4\}$.
  \Question Determine $R^2, R^3$ and $R^4$.
  \Question Give a relation $S$ on $A$ such that $R \union (S \comp R)
  = S$.
\end{Exercise}

%unanswered
\begin{Exercise} [number=40]
  Verify:
  \Question A relation $R$ on $A$ is transitive iff $R \comp R
  \subseteq R$.
  \Question Give an example of a transitive relation $R$ for which $R
  \comp R = R$ is false.
\end{Exercise}


%unanswered
\begin{Exercise} [number=41]
  Verify:
  \Question $Q \comp (R \comp S) = (Q \comp R) \comp S$. (Thus, the
  notation $Q \comp R \comp S$ is unambiguous.)
  \Question $(R \comp S)^{-1} = S^{-1} \comp R^{-1}$.
\end{Exercise}


%unanswered
\begin{Exercise} [number=45]
  Show that the relation $<$ on $\naturals$ is the transitive closure
  of the relation $R = \Set{(n,n+1)}{n \in \naturals}$.
\end{Exercise}

%unanswered
\begin{Exercise} [number=46]
  Let $R$ be a relation on $A$. Show that $R^+ \union \Delta_A$ is the
  reflexive transitive closure of $R$.
\end{Exercise}

%unanswered
\begin{Exercise} [number=47]
  Give the reflexive transitive closure of the following relation: \[R
  = \Set{(n,n+1)}{n \in \naturals}\].
\end{Exercise}

%unanswered
\begin{Exercise} [number=48,difficulty=1]
  \Question Show that an intersection of \emph{arbitrarily many} transitive
  relations is transitive.
  \Question Suppose that $R$ is a relation on $A$. Note that $A^2$ is
  one example of a transitive relation on $A$ that extends
  $R$. Conclude that the intersection of \emph{all} transitive relations
  extending $R$ is the \emph{least} transative relation extending $R$. In
  other words, $R^+$ equals the intersection of \emph{all} transitive
  relations extending $R$.
\end{Exercise}

%unanswered
\begin{Exercise} [difficulty=1,number=49]
  \Question Show that $(R^*)^{-1} = (R^{-1})^*$.
  \Question Show by means of a counter-example that $(R \union
  R^{-1})^* = R^* \union R^{-1*}$ may be false.
  \Question Prove: if $S \comp R \subseteq R \comp S$, then $(R \comp
  S)^* \subseteq R^* \comp S^*$.
\end{Exercise}

%unanswered
\begin{Exercise} [number=50]
  Suppose that $R$ and $S$ are reflexive relations on the set
  $A$. Then $\Delta_A \subseteq R$ and $\Delta_A \subseteq S$, so
  $\Delta_A \subseteq R \intersect S$, i.e., $R \intersect S$ is
  reflexive as well. We say: reflexivity is preserved under
  intersection. Similarly, if $R$ and $S$ are reflexive, then
  $\Delta_A \subseteq R \union S$, so $R \union S$ is
  reflexive. Reflexivity is preserved under union. If $R$ is
  reflexive, then $\Delta_A \subseteq R^{-1}$, so $R^{-1}$ is
  reflexive. Reflexivity is preserved under inverse. If $R$ and $S$
  are reflexive, $\Delta_A \subseteq R \comp S$, so $R \comp S$ is
  reflexive. Reflexivity is preserved under composition. Finally, if
  $R$ on $A$ is reflexive, then the complement of $R$, i.e., the
  relation $A^2 \setminus R$, is irreflexive. So reflexivity is not
  preserved under complement. These closure properties of reflexive
  relations are listed in the table below.

  We can ask the same questions for other relational
  properties. Suppose $R$ and $S$ are symmetric relations on $A$. Does
  it follow that $R \intersect S$ is symmetric? That $R \union S$ is
  symmetric? That $R^{-1}$ is symmetric? That $A^2 \setminus R$ is
  symmetric? That $R \comp S$ is symmetric? Similarly for the property
  of transitivity. These questions are summarized in the table
  below. Complete the table by putting `yes' or `no' in the
  appropriate places.

  \dots (table) \dots
\end{Exercise}

%%% Local Variables: 
%%% mode: latex
%%% TeX-master: "exercises_base"
%%% End: 

\end{document}
