\documentclass[leqno]{article}
%% ODER: format ==         = "\mathrel{==}"
%% ODER: format /=         = "\neq "
%
%
\makeatletter
\@ifundefined{lhs2tex.lhs2tex.sty.read}%
  {\@namedef{lhs2tex.lhs2tex.sty.read}{}%
   \newcommand\SkipToFmtEnd{}%
   \newcommand\EndFmtInput{}%
   \long\def\SkipToFmtEnd#1\EndFmtInput{}%
  }\SkipToFmtEnd

\newcommand\ReadOnlyOnce[1]{\@ifundefined{#1}{\@namedef{#1}{}}\SkipToFmtEnd}
\usepackage{amstext}
\usepackage{amssymb}
\usepackage{stmaryrd}
\DeclareFontFamily{OT1}{cmtex}{}
\DeclareFontShape{OT1}{cmtex}{m}{n}
  {<5><6><7><8>cmtex8
   <9>cmtex9
   <10><10.95><12><14.4><17.28><20.74><24.88>cmtex10}{}
\DeclareFontShape{OT1}{cmtex}{m}{it}
  {<-> ssub * cmtt/m/it}{}
\newcommand{\texfamily}{\fontfamily{cmtex}\selectfont}
\DeclareFontShape{OT1}{cmtt}{bx}{n}
  {<5><6><7><8>cmtt8
   <9>cmbtt9
   <10><10.95><12><14.4><17.28><20.74><24.88>cmbtt10}{}
\DeclareFontShape{OT1}{cmtex}{bx}{n}
  {<-> ssub * cmtt/bx/n}{}
\newcommand{\tex}[1]{\text{\texfamily#1}}	% NEU

\newcommand{\Sp}{\hskip.33334em\relax}


\newcommand{\Conid}[1]{\mathit{#1}}
\newcommand{\Varid}[1]{\mathit{#1}}
\newcommand{\anonymous}{\kern0.06em \vbox{\hrule\@width.5em}}
\newcommand{\plus}{\mathbin{+\!\!\!+}}
\newcommand{\bind}{\mathbin{>\!\!\!>\mkern-6.7mu=}}
\newcommand{\rbind}{\mathbin{=\mkern-6.7mu<\!\!\!<}}% suggested by Neil Mitchell
\newcommand{\sequ}{\mathbin{>\!\!\!>}}
\renewcommand{\leq}{\leqslant}
\renewcommand{\geq}{\geqslant}
\usepackage{polytable}

%mathindent has to be defined
\@ifundefined{mathindent}%
  {\newdimen\mathindent\mathindent\leftmargini}%
  {}%

\def\resethooks{%
  \global\let\SaveRestoreHook\empty
  \global\let\ColumnHook\empty}
\newcommand*{\savecolumns}[1][default]%
  {\g@addto@macro\SaveRestoreHook{\savecolumns[#1]}}
\newcommand*{\restorecolumns}[1][default]%
  {\g@addto@macro\SaveRestoreHook{\restorecolumns[#1]}}
\newcommand*{\aligncolumn}[2]%
  {\g@addto@macro\ColumnHook{\column{#1}{#2}}}

\resethooks

\newcommand{\onelinecommentchars}{\quad-{}- }
\newcommand{\commentbeginchars}{\enskip\{-}
\newcommand{\commentendchars}{-\}\enskip}

\newcommand{\visiblecomments}{%
  \let\onelinecomment=\onelinecommentchars
  \let\commentbegin=\commentbeginchars
  \let\commentend=\commentendchars}

\newcommand{\invisiblecomments}{%
  \let\onelinecomment=\empty
  \let\commentbegin=\empty
  \let\commentend=\empty}

\visiblecomments

\newlength{\blanklineskip}
\setlength{\blanklineskip}{0.66084ex}

\newcommand{\hsindent}[1]{\quad}% default is fixed indentation
\let\hspre\empty
\let\hspost\empty
\newcommand{\NB}{\textbf{NB}}
\newcommand{\Todo}[1]{$\langle$\textbf{To do:}~#1$\rangle$}

\EndFmtInput
\makeatother
%
%
%
%
%
%
% This package provides two environments suitable to take the place
% of hscode, called "plainhscode" and "arrayhscode". 
%
% The plain environment surrounds each code block by vertical space,
% and it uses \abovedisplayskip and \belowdisplayskip to get spacing
% similar to formulas. Note that if these dimensions are changed,
% the spacing around displayed math formulas changes as well.
% All code is indented using \leftskip.
%
% Changed 19.08.2004 to reflect changes in colorcode. Should work with
% CodeGroup.sty.
%
\ReadOnlyOnce{polycode.fmt}%
\makeatletter

\newcommand{\hsnewpar}[1]%
  {{\parskip=0pt\parindent=0pt\par\vskip #1\noindent}}

% can be used, for instance, to redefine the code size, by setting the
% command to \small or something alike
\newcommand{\hscodestyle}{}

% The command \sethscode can be used to switch the code formatting
% behaviour by mapping the hscode environment in the subst directive
% to a new LaTeX environment.

\newcommand{\sethscode}[1]%
  {\expandafter\let\expandafter\hscode\csname #1\endcsname
   \expandafter\let\expandafter\endhscode\csname end#1\endcsname}

% "compatibility" mode restores the non-polycode.fmt layout.

\newenvironment{compathscode}%
  {\par\noindent
   \advance\leftskip\mathindent
   \hscodestyle
   \let\\=\@normalcr
   \let\hspre\(\let\hspost\)%
   \pboxed}%
  {\endpboxed\)%
   \par\noindent
   \ignorespacesafterend}

\newcommand{\compaths}{\sethscode{compathscode}}

% "plain" mode is the proposed default.
% It should now work with \centering.
% This required some changes. The old version
% is still available for reference as oldplainhscode.

\newenvironment{plainhscode}%
  {\hsnewpar\abovedisplayskip
   \advance\leftskip\mathindent
   \hscodestyle
   \let\hspre\(\let\hspost\)%
   \pboxed}%
  {\endpboxed%
   \hsnewpar\belowdisplayskip
   \ignorespacesafterend}

\newenvironment{oldplainhscode}%
  {\hsnewpar\abovedisplayskip
   \advance\leftskip\mathindent
   \hscodestyle
   \let\\=\@normalcr
   \(\pboxed}%
  {\endpboxed\)%
   \hsnewpar\belowdisplayskip
   \ignorespacesafterend}

% Here, we make plainhscode the default environment.

\newcommand{\plainhs}{\sethscode{plainhscode}}
\newcommand{\oldplainhs}{\sethscode{oldplainhscode}}
\plainhs

% The arrayhscode is like plain, but makes use of polytable's
% parray environment which disallows page breaks in code blocks.

\newenvironment{arrayhscode}%
  {\hsnewpar\abovedisplayskip
   \advance\leftskip\mathindent
   \hscodestyle
   \let\\=\@normalcr
   \(\parray}%
  {\endparray\)%
   \hsnewpar\belowdisplayskip
   \ignorespacesafterend}

\newcommand{\arrayhs}{\sethscode{arrayhscode}}

% The mathhscode environment also makes use of polytable's parray 
% environment. It is supposed to be used only inside math mode 
% (I used it to typeset the type rules in my thesis).

\newenvironment{mathhscode}%
  {\parray}{\endparray}

\newcommand{\mathhs}{\sethscode{mathhscode}}

% texths is similar to mathhs, but works in text mode.

\newenvironment{texthscode}%
  {\(\parray}{\endparray\)}

\newcommand{\texths}{\sethscode{texthscode}}

% The framed environment places code in a framed box.

\def\codeframewidth{\arrayrulewidth}
\RequirePackage{calc}

\newenvironment{framedhscode}%
  {\parskip=\abovedisplayskip\par\noindent
   \hscodestyle
   \arrayrulewidth=\codeframewidth
   \tabular{@{}|p{\linewidth-2\arraycolsep-2\arrayrulewidth-2pt}|@{}}%
   \hline\framedhslinecorrect\\{-1.5ex}%
   \let\endoflinesave=\\
   \let\\=\@normalcr
   \(\pboxed}%
  {\endpboxed\)%
   \framedhslinecorrect\endoflinesave{.5ex}\hline
   \endtabular
   \parskip=\belowdisplayskip\par\noindent
   \ignorespacesafterend}

\newcommand{\framedhslinecorrect}[2]%
  {#1[#2]}

\newcommand{\framedhs}{\sethscode{framedhscode}}

% The inlinehscode environment is an experimental environment
% that can be used to typeset displayed code inline.

\newenvironment{inlinehscode}%
  {\(\def\column##1##2{}%
   \let\>\undefined\let\<\undefined\let\\\undefined
   \newcommand\>[1][]{}\newcommand\<[1][]{}\newcommand\\[1][]{}%
   \def\fromto##1##2##3{##3}%
   \def\nextline{}}{\) }%

\newcommand{\inlinehs}{\sethscode{inlinehscode}}

% The joincode environment is a separate environment that
% can be used to surround and thereby connect multiple code
% blocks.

\newenvironment{joincode}%
  {\let\orighscode=\hscode
   \let\origendhscode=\endhscode
   \def\endhscode{\def\hscode{\endgroup\def\@currenvir{hscode}\\}\begingroup}
   %\let\SaveRestoreHook=\empty
   %\let\ColumnHook=\empty
   %\let\resethooks=\empty
   \orighscode\def\hscode{\endgroup\def\@currenvir{hscode}}}%
  {\origendhscode
   \global\let\hscode=\orighscode
   \global\let\endhscode=\origendhscode}%

\makeatother
\EndFmtInput
%

\usepackage{amsmath}

\long\def\ignore#1{}

\renewcommand{\iff}{\Leftrightarrow}
\newcommand{\xor}{\oplus}
\renewcommand{\implies}{\Rightarrow}

\setlength{\parindent}{0pt}
\setlength{\parskip}{1ex plus 0.5ex minus 0.2ex}

\begin{document}

\section*{Exercises from Chapter 2}

\subsection*{Exercise 2.2}
The truth table for the \emph{exclusive} version of \emph{or}:

\begin{tabular}{ccc}
  \hline
  $P$ & $Q$ & $P \xor Q$ \\
  \hline
   t  &  t  &      f \\
   t  &  f  &      t \\
   f  &  t  &      t \\
   f  &  f  &      f \\
  \hline
\end{tabular}

\subsection*{Exercise 2.4}
The truth table for $\neg (P \iff Q)$:

\begin{tabular}{cccc}
  \hline
  $P$ & $Q$ & $P \iff Q$ & $\neg (P \iff Q)$ \\
  \hline
  t & t &    t    &       f \\
  t & f &    f    &       t \\
  f & t &    f    &       t \\
  f & f &    t    &       f \\
  \hline
\end{tabular}

As we can see, the the column for $\neg (P \iff Q)$ is identical to the column for $P \xor Q$ in exercise 2.2. Then
\begin{hscode}\SaveRestoreHook
\column{B}{@{}>{\hspre}l<{\hspost}@{}}%
\column{3}{@{}>{\hspre}l<{\hspost}@{}}%
\column{E}{@{}>{\hspre}l<{\hspost}@{}}%
\>[3]{}\Varid{x}\xor\Varid{y}\mathrel{=}\neg \;(\Varid{x}\iff\Varid{y})\mathrel{=}\neg \;(\Varid{x}\equiv \Varid{y})\mathrel{=}\Varid{x}\not\equiv \Varid{y}{}\<[E]%
\ColumnHook
\end{hscode}\resethooks
Thus the implementation of \ensuremath{(\xor)} is correct.

\subsection*{Exercise 2.9}
\begin{tabular}{ccccccc}
  \hline
  $(P$ & $\iff$ & $Q)$ & $\xor$ & $Q$ & $\iff$ & $P$ \\
  \hline
   t  &  f  & t  &  t  & t &  t  & t \\
   t  &  t  & f  &  t  & f &  t  & t \\
   f  &  t  & t  &  f  & t &  t  & f \\
   f  &  f  & f  &  f  & f &  t  & f \\
  \hline
\end{tabular}

$(P \xor Q) \xor Q \iff P$ is a logical validity, thus $(P \xor Q) \xor Q$ is equivalent to $P$.

\subsection*{Ecercise 2.11}
\begin{enumerate}
\item Law of double negation:

\begin{tabular}{ccccc}
  \hline
  $P$ & $\iff$ & $\neg$ & $\neg$ & $P$ \\
  \hline
   t  &    t   &    t   &    f   &  t  \\
   f  &    t   &    f   &    t   &  f  \\
  \hline
\end{tabular}

\item Laws of idempotence:

\begin{tabular}{ccccc}
  \hline
  $P$ & $\land$ & $P$ & $\iff$ & $P$ \\
  \hline
   t  &    t    &  t  &    t   &  t  \\
   f  &    f    &  f  &    t   &  f  \\
  \hline
\end{tabular}
\quad
\begin{tabular}{ccccc}
  \hline
  $P$ & $\lor$ & $P$ & $\iff$ & $P$ \\
  \hline
   t  &    t    &  t  &    t   &  t  \\
   f  &    f    &  f  &    t   &  f  \\
  \hline
\end{tabular}

\item
\begin{tabular}{cccccccc}
  \hline
  $P$ & $\implies$ & $Q$ & $\iff$ & $\neg$ & $P$ & $\lor$ & $Q$ \\
  \hline
   t  &      t     &  t  &    t   &    f   &  t  &    t   &  t  \\
   t  &      f     &  f  &    t   &    f   &  t  &    f   &  f  \\
   f  &      t     &  t  &    t   &    t   &  f  &    t   &  t  \\
   f  &      t     &  f  &    t   &    t   &  f  &    t   &  f  \\
  \hline
\end{tabular}

\begin{tabular}{ccccccccc}
  \hline
  $\neg$ & $(P$ & $\implies$ & $Q)$ & $\iff$ & $P$ & $\land$ & $\neg$ & $Q$ \\
  \hline
     f   &   t  &      t     &  t   &    t   &  t  &    f    &    f   &  t  \\
     t   &   t  &      f     &  f   &    t   &  t  &    t    &    t   &  f  \\
     f   &   f  &      t     &  t   &    t   &  f  &    f    &    f   &  t  \\
     f   &   f  &      t     &  f   &    t   &  f  &    f    &    t   &  f  \\
  \hline
\end{tabular}

\item Laws of contraposition:

\begin{tabular}{ccccccccc}
  \hline
  $(\neg$ & $P$ & $\implies$ & $\neg$ & $Q)$ & $\iff$ & $(Q$ & $\implies$ & $P)$ \\
  \hline
      f   &  t  &      t     &    f   &  t   &    t   &   t  &      t     &  t   \\
      f   &  t  &      t     &    t   &  f   &    t   &   f  &      t     &  t   \\
      t   &  f  &      f     &    f   &  t   &    t   &   t  &      f     &  f   \\
      t   &  f  &      t     &    t   &  f   &    t   &   f  &      t     &  f   \\
  \hline
\end{tabular}

\begin{tabular}{ccccccccc}
  \hline
  $(P$ & $\implies$ & $\neg$ & $Q)$ & $\iff$ & $(Q$ & $\implies$ & $\neg$ & $P)$ \\
  \hline
    t  &      f     &    f   &  t   &    t   &   t  &      f     &    f   &  t   \\
    t  &      t     &    t   &  f   &    t   &   f  &      t     &    f   &  t   \\
    f  &      t     &    f   &  t   &    t   &   t  &      t     &    t   &  f   \\
    f  &      t     &    t   &  f   &    t   &   f  &      t     &    t   &  f   \\
  \hline
\end{tabular}

\begin{tabular}{ccccccccc}
  \hline
  $(\neg$ & $P$ & $\implies$ & $Q)$ & $\iff$ & $(\neg$ & $Q$ & $\implies$ & $P)$ \\
  \hline
      f   &  t  &      t     &  t   &    t   &     f   &  t  &      t     &  t   \\
      f   &  t  &      t     &  f   &    t   &     t   &  f  &      t     &  t   \\
      t   &  f  &      t     &  t   &    t   &     f   &  t  &      t     &  f   \\
      t   &  f  &      f     &  f   &    t   &     t   &  f  &      f     &  f   \\
  \hline
\end{tabular}

\item
\renewcommand{\tabcolsep}{3pt}
\begin{tabular}{ccccccccccccccccccccc}
  \hline
  $(P$ & $\iff$ & $Q)$ & $\iff$ & $((P$ & $\implies$ & $Q)$ & $\land$ & $(Q$ & $\implies$ & $P))$
                & $\iff$ & $((P$ & $\land$ & $Q)$ & $\lor$ & $(\neg$ & $P$ & $\land$ & $\neg$ & $Q))$ \\
  \hline
 %%(P \iff Q) \iff ((P \implies Q) \land (Q \implies P)) \iff ((P \land Q) \lor (\neg P \land \neg Q))
   t & t & t &  t & t &   t   & t &  t  & t &   t  & t  &  t &  t & t & t &  t &  f & t & f  & f & t  \\ 
   t & f & f &  t & t &   f   & f &  f  & f &   t  & t  &  t &  t & f & f &  f &  f & t & f  & t & f  \\ 
   f & f & t &  t & f &   t   & t &  f  & t &   f  & f  &  t &  f & f & t &  f &  t & f & f  & f & t  \\ 
   f & t & f &  t & f &   t   & f &  t  & f &   t  & f  &  t &  f & f & f &  t &  t & f & t  & t & f  \\ 
  \hline
\end{tabular}
%% Return to default table column seperator
\renewcommand{\tabcolsep}{6pt}

\item Laws of commutativity:

\begin{tabular}{ccccccc}
  \hline
  $P$ & $\land$ & $Q$ & $\iff$ & $Q$ & $\land$ & $P$ \\
  \hline
   t  &    t    &  t  &   t    &  t  &    t    &  t  \\
   t  &    f    &  f  &   t    &  f  &    f    &  t  \\
   f  &    f    &  t  &   t    &  t  &    f    &  f  \\
   f  &    f    &  f  &   t    &  f  &    f    &  f  \\
  \hline
\end{tabular}
\quad
\begin{tabular}{ccccccc}
  \hline
  $P$ & $\lor$ & $Q$ & $\iff$ & $Q$ & $\lor$ & $P$ \\
  \hline
   t  &    t   &  t  &   t    &  t  &    t   &  t  \\
   t  &    t   &  f  &   t    &  f  &    t   &  t  \\
   f  &    t   &  t  &   t    &  t  &    t   &  f  \\
   f  &    f   &  f  &   t    &  f  &    f   &  f  \\
  \hline
\end{tabular}

\item DeMorgan laws:

\begin{tabular}{cccccccccc}
  \hline
  $\neg$ & $(P$ & $\land$ & $Q)$ & $\iff$ & $\neg$ & $P$ & $\lor$ & $\neg$ & $Q$ \\
  \hline
     f   &   t  &    t    &  t   &    t   &    f   &  t  &    f   &    f   &  t  \\
     t   &   t  &    f    &  f   &    t   &    f   &  t  &    t   &    t   &  f  \\
     t   &   f  &    f    &  t   &    t   &    t   &  f  &    t   &    f   &  t  \\
     t   &   f  &    f    &  f   &    t   &    t   &  f  &    t   &    t   &  f  \\
  \hline
\end{tabular}

\begin{tabular}{cccccccccc}
  \hline
  $\neg$ & $(P$ & $\lor$ & $Q)$ & $\iff$ & $\neg$ & $P$ & $\land$ & $\neg$ & $Q$ \\
  \hline
     f   &   t  &   t    &  t   &    t   &    f   &  t  &    f    &    f   &  t  \\
     f   &   t  &   t    &  f   &    t   &    f   &  t  &    f    &    t   &  f  \\
     f   &   f  &   t    &  t   &    t   &    t   &  f  &    f    &    f   &  t  \\
     t   &   f  &   f    &  f   &    t   &    t   &  f  &    t    &    t   &  f  \\
  \hline
\end{tabular}

\item Laws of associativity:

\begin{tabular}{ccccccccccc}
  \hline
  $P$ & $\land$ & $(Q$ & $\land$ & $R)$ & $\iff$ & $(P$ & $\land$ & $Q)$ & $\land$ & $R$ \\
  \hline
   t  &    t    &   t  &    t    &  t   &   t    &   t  &    t    &  t   &    t    &  t  \\
   t  &    f    &   t  &    f    &  f   &   t    &   t  &    t    &  t   &    f    &  f  \\
   t  &    f    &   f  &    f    &  t   &   t    &   t  &    f    &  f   &    f    &  t  \\
   t  &    f    &   f  &    f    &  f   &   t    &   t  &    f    &  f   &    f    &  f  \\
   f  &    f    &   t  &    t    &  t   &   t    &   f  &    f    &  t   &    f    &  t  \\
   f  &    f    &   t  &    f    &  f   &   t    &   f  &    f    &  t   &    f    &  f  \\
   f  &    f    &   f  &    f    &  t   &   t    &   f  &    f    &  f   &    f    &  t  \\
   f  &    f    &   f  &    f    &  f   &   t    &   f  &    f    &  f   &    f    &  f  \\
  \hline
\end{tabular}

\begin{tabular}{ccccccccccc}
  \hline
  $P$ & $\lor$ & $(Q$ & $\lor$ & $R)$ & $\iff$ & $(P$ & $\lor$ & $Q)$ & $\lor$ & $R$ \\
  \hline
   t  &    t   &   t  &    t   &  t   &   t    &   t  &   t    &  t   &   t    &  t  \\
   t  &    t   &   t  &    t   &  f   &   t    &   t  &   t    &  t   &   t    &  f  \\
   t  &    t   &   f  &    t   &  t   &   t    &   t  &   t    &  f   &   t    &  t  \\
   t  &    t   &   f  &    f   &  f   &   t    &   t  &   t    &  f   &   t    &  f  \\
   f  &    t   &   t  &    t   &  t   &   t    &   f  &   t    &  t   &   t    &  t  \\
   f  &    t   &   t  &    t   &  f   &   t    &   f  &   t    &  t   &   t    &  f  \\
   f  &    t   &   f  &    t   &  t   &   t    &   f  &   f    &  f   &   t    &  t  \\
   f  &    f   &   f  &    f   &  f   &   t    &   f  &   f    &  f   &   f    &  f  \\
  \hline
\end{tabular}

\item Distribution laws:

\begin{tabular}{ccccccccccccc}
  \hline
  $P$ & $\land$ & $(Q$ & $\lor$ & $R)$ & $\iff$ & $(P$ & $\land$ & $Q)$ & $\lor$ & $(P$ & $\land$ & $R)$ \\
  \hline
   t  &    t    &   t  &    t   &  t   &    t   &   t  &    t    &  t   &    t   &   t  &    t    &  t   \\
   t  &    t    &   t  &    t   &  f   &    t   &   t  &    t    &  t   &    t   &   t  &    f    &  f   \\
   t  &    t    &   f  &    t   &  t   &    t   &   t  &    f    &  f   &    t   &   t  &    t    &  t   \\
   t  &    f    &   f  &    f   &  f   &    t   &   t  &    f    &  f   &    f   &   t  &    f    &  f   \\
   f  &    f    &   t  &    t   &  t   &    t   &   f  &    f    &  t   &    f   &   f  &    f    &  t   \\
   f  &    f    &   t  &    t   &  f   &    t   &   f  &    f    &  t   &    f   &   f  &    f    &  f   \\
   f  &    f    &   f  &    t   &  t   &    t   &   f  &    f    &  f   &    f   &   f  &    f    &  t   \\
   f  &    f    &   f  &    f   &  f   &    t   &   f  &    f    &  f   &    f   &   f  &    f    &  f   \\
  \hline
\end{tabular}

\begin{tabular}{ccccccccccccc}
  \hline
  $P$ & $\lor$ & $(Q$ & $\land$ & $R)$ & $\iff$ & $(P$ & $\lor$ & $Q)$ & $\land$ & $(P$ & $\lor$ & $R)$ \\
  \hline
   t  &    t   &   t  &    t    &  t   &    t   &   t  &    t   &  t   &    t    &   t  &    t   &  t   \\
   t  &    t   &   t  &    f    &  f   &    t   &   t  &    t   &  t   &    t    &   t  &    t   &  f   \\
   t  &    t   &   f  &    f    &  t   &    t   &   t  &    t   &  f   &    t    &   t  &    t   &  t   \\
   t  &    t   &   f  &    f    &  f   &    t   &   t  &    t   &  f   &    t    &   t  &    t   &  f   \\
   f  &    t   &   t  &    t    &  t   &    t   &   f  &    t   &  t   &    t    &   f  &    t   &  t   \\
   f  &    f   &   t  &    f    &  f   &    t   &   f  &    t   &  t   &    f    &   f  &    f   &  f   \\
   f  &    f   &   f  &    f    &  t   &    t   &   f  &    f   &  f   &    f    &   f  &    t   &  t   \\
   f  &    f   &   f  &    f    &  f   &    t   &   f  &    f   &  f   &    f    &   f  &    f   &  f   \\
  \hline
\end{tabular}
\end{enumerate}

\subsection*{Exercise 2.13}
Checks for the principles from Theorem 2.12:
\begin{hscode}\SaveRestoreHook
\column{B}{@{}>{\hspre}l<{\hspost}@{}}%
\column{3}{@{}>{\hspre}l<{\hspost}@{}}%
\column{12}{@{}>{\hspre}c<{\hspost}@{}}%
\column{12E}{@{}l@{}}%
\column{15}{@{}>{\hspre}l<{\hspost}@{}}%
\column{E}{@{}>{\hspre}l<{\hspost}@{}}%
\>[3]{}\mathbf{import}\;\Conid{TAMO}{}\<[E]%
\\[\blanklineskip]%
\>[3]{}\Varid{test'1a}{}\<[12]%
\>[12]{}\mathrel{=}{}\<[12E]%
\>[15]{}\neg \;\Conid{True}\equiv \Conid{False}{}\<[E]%
\\
\>[3]{}\Varid{test'1b}{}\<[12]%
\>[12]{}\mathrel{=}{}\<[12E]%
\>[15]{}\neg \;\Conid{False}\equiv \Conid{True}{}\<[E]%
\\
\>[3]{}\Varid{test'2}{}\<[12]%
\>[12]{}\mathrel{=}{}\<[12E]%
\>[15]{}\Varid{logEquiv1}\;(\lambda \Varid{p}\to \Varid{p}\Longrightarrow\Conid{False})\;(\lambda \Varid{p}\to \neg \;\Varid{p}){}\<[E]%
\\
\>[3]{}\Varid{test'3a}{}\<[12]%
\>[12]{}\mathrel{=}{}\<[12E]%
\>[15]{}\Varid{logEquiv1}\;(\lambda \Varid{p}\to \Varid{p}\mathrel{\vee}\Conid{True})\;(\lambda \Varid{p}\to \Conid{True}){}\<[E]%
\\
\>[3]{}\Varid{test'3b}{}\<[12]%
\>[12]{}\mathrel{=}{}\<[12E]%
\>[15]{}\Varid{logEquiv1}\;(\lambda \Varid{p}\to \Varid{p}\mathrel{\wedge}\Conid{False})\;(\lambda \Varid{p}\to \Conid{False}){}\<[E]%
\\
\>[3]{}\Varid{test'4a}{}\<[12]%
\>[12]{}\mathrel{=}{}\<[12E]%
\>[15]{}\Varid{logEquiv1}\;(\lambda \Varid{p}\to \Varid{p}\mathrel{\vee}\Conid{False})\;\Varid{id}{}\<[E]%
\\
\>[3]{}\Varid{test'4b}{}\<[12]%
\>[12]{}\mathrel{=}{}\<[12E]%
\>[15]{}\Varid{logEquiv1}\;(\lambda \Varid{p}\to \Varid{p}\mathrel{\wedge}\Conid{True})\;\Varid{id}{}\<[E]%
\\
\>[3]{}\Varid{test'5}{}\<[12]%
\>[12]{}\mathrel{=}{}\<[12E]%
\>[15]{}\Varid{logEquiv1}\;(\lambda \Varid{p}\to \Varid{p}\mathrel{\vee}\neg \;\Varid{p})\;(\lambda \Varid{p}\to \Conid{True}){}\<[E]%
\\
\>[3]{}\Varid{test'6}{}\<[12]%
\>[12]{}\mathrel{=}{}\<[12E]%
\>[15]{}\Varid{logEquiv1}\;(\lambda \Varid{p}\to \Varid{p}\mathrel{\wedge}\neg \;\Varid{p})\;(\lambda \Varid{p}\to \Conid{False}){}\<[E]%
\ColumnHook
\end{hscode}\resethooks

\subsection*{Exercise 2.15}
Contradiction tests for propositional functions with one, two and three variables:
\begin{hscode}\SaveRestoreHook
\column{B}{@{}>{\hspre}l<{\hspost}@{}}%
\column{3}{@{}>{\hspre}l<{\hspost}@{}}%
\column{15}{@{}>{\hspre}c<{\hspost}@{}}%
\column{15E}{@{}l@{}}%
\column{19}{@{}>{\hspre}l<{\hspost}@{}}%
\column{40}{@{}>{\hspre}l<{\hspost}@{}}%
\column{42}{@{}>{\hspre}l<{\hspost}@{}}%
\column{E}{@{}>{\hspre}l<{\hspost}@{}}%
\>[3]{}\Varid{bools}\mathrel{=}[\mskip1.5mu \Conid{True},\Conid{False}\mskip1.5mu]{}\<[E]%
\\[\blanklineskip]%
\>[3]{}\Varid{contra1}{}\<[15]%
\>[15]{}\mathbin{::}{}\<[15E]%
\>[19]{}(\Conid{Bool}\to \Conid{Bool})\to \Conid{Bool}{}\<[E]%
\\
\>[3]{}\Varid{contra1}\;\Varid{bf}{}\<[15]%
\>[15]{}\mathrel{=}{}\<[15E]%
\>[19]{}\Varid{and}\;[\mskip1.5mu \neg \;(\Varid{bf}\;\Varid{p})\mid \Varid{p}\leftarrow \Varid{bools}\mskip1.5mu]{}\<[E]%
\\[\blanklineskip]%
\>[3]{}\Varid{contra2}{}\<[15]%
\>[15]{}\mathbin{::}{}\<[15E]%
\>[19]{}(\Conid{Bool}\to \Conid{Bool}\to \Conid{Bool})\to \Conid{Bool}{}\<[E]%
\\
\>[3]{}\Varid{contra2}\;\Varid{bf}{}\<[15]%
\>[15]{}\mathrel{=}{}\<[15E]%
\>[19]{}\Varid{and}\;[\mskip1.5mu \neg \;(\Varid{bf}\;\Varid{p}\;\Varid{q})\mid {}\<[40]%
\>[40]{}\Varid{p}\leftarrow \Varid{bools},\Varid{q}\leftarrow \Varid{bools}\mskip1.5mu]{}\<[E]%
\\[\blanklineskip]%
\>[3]{}\Varid{contra3}{}\<[15]%
\>[15]{}\mathbin{::}{}\<[15E]%
\>[19]{}(\Conid{Bool}\to \Conid{Bool}\to \Conid{Bool}\to \Conid{Bool})\to \Conid{Bool}{}\<[E]%
\\
\>[3]{}\Varid{contra3}\;\Varid{bf}{}\<[15]%
\>[15]{}\mathrel{=}{}\<[15E]%
\>[19]{}\Varid{and}\;[\mskip1.5mu \neg \;(\Varid{bf}\;\Varid{p}\;\Varid{q}\;\Varid{r})\mid {}\<[42]%
\>[42]{}\Varid{p}\leftarrow \Varid{bools},\Varid{q}\leftarrow \Varid{bools},\Varid{r}\leftarrow \Varid{bools}\mskip1.5mu]{}\<[E]%
\ColumnHook
\end{hscode}\resethooks

\subsection*{Exercise 2.16}
Useful denials for every sentence of Exercise 2.31:
\begin{enumerate}
  \item The equation $x^2 + 1 = 0$ does not have a solution.
  \item A largest natural number exists.
  \item The number 13 is not prime.
  \item The number $n$ is not prime.
  \item There are a finite number of primes.
\end{enumerate}

\subsection*{Exercise 2.17}
A denial for the statement that $x < y < z$ (where $x,y,z \in \mathbb{R}$):

$x < y < z \equiv x < y \land y < z$. Thus we can use the First Law of DeMorgan. 
\[ \neg (x < y < z) \equiv \neg (x < y \land y < z) \equiv x \geq y \lor y \geq z \]

\subsection*{Exercise 2.18}
\begin{equation*}
  \tag*{1.}
  \begin{aligned}
    (\Phi \iff \Psi) & \equiv ((\Phi \implies \Psi) \land (\Psi \implies \Phi))
                       && \text{by Theorem 2.10, 5} \\
                     & \equiv ((\neg \Psi \implies \neg \Phi) \land (\neg \Phi \implies \neg \Psi))
                       && \text{by contraposition} \\
                     & \equiv ((\neg \Phi \implies \neg \Psi) \land (\neg \Psi \implies \neg \Phi))
                       && \text{by commutativity of } \land \\
                     & \equiv (\neg \Phi \iff \neg \Psi)
                       && \text{by Theorem 2.10, 5}
  \end{aligned}
\end{equation*}

\begin{equation*}
  \tag*{2.}
  \begin{aligned}
    (\neg \Phi \iff \Psi) & \equiv ((\neg \Phi \implies \Psi) \land (\Psi \implies \neg \Phi))
                            && \text{by Theorem 2.10, 5} \\
                          & \equiv ((\neg \Psi \implies \Phi) \land (\Phi \implies \neg \Psi))
                            && \text{by contraposition} \\
                          & \equiv ((\Phi \implies \neg \Psi) \land (\neg \Psi \implies \Phi))
                            && \text{by commutativity of } \land \\
                          & \equiv (\Phi \iff \neg \Psi)
                            && \text{by Theorem 2.10, 5}
  \end{aligned}
\end{equation*}

\subsection*{Exercise 2.19}
$\Phi \equiv \Psi$ means that, no matter the truth values of $P, Q, \dots$ occuring in the formulas, the formulas $\Phi$ and $\Psi$ produce the same truth values: either both are true, or both are false. In both situations, $\Phi \iff \Psi$ is true.
Thus $(\Phi \equiv \Psi) \implies (\Phi \iff \Psi)$.

Since $\Phi \iff \Psi$ is only true when $\Phi$ and $\Psi$ have the same truth values, we also have $(\Phi \iff \Psi) \implies (\Phi \equiv \Psi)$. By part 5 of Theorem 2.10 we conclude that $(\Phi \equiv \Psi) \iff (\Phi \iff \Psi)$.

\subsection*{Exercise 2.20}
\begin{enumerate}
  \item $\neg P \implies Q$ and $P \implies \neg Q$ are not equivalent. The truth table below shows that $(\neg P \implies Q) \iff (P \implies \neg Q)$ is not a logical validity:
        \begin{center}
        \begin{tabular}{ccccccccc}
          \hline
          $(\neg$ & $P$ & $\implies$ & $Q)$ & $\iff$ & $(P$ & $\implies$ & $\neg$ & $Q)$ \\
          \hline
              f   &  t  &      t     &  t   &    f   &   t  &      f     &    f   &  t   \\
              f   &  t  &      t     &  f   &    t   &   t  &      t     &    t   &  f   \\
              t   &  f  &      t     &  t   &    t   &   f  &      t     &    f   &  t   \\
              t   &  f  &      f     &  f   &    f   &   f  &      t     &    t   &  f   \\
          \hline
        \end{tabular}
        \end{center}
        Verifying with Haskell:
        \begin{hscode}\SaveRestoreHook
\column{B}{@{}>{\hspre}l<{\hspost}@{}}%
\column{11}{@{}>{\hspre}l<{\hspost}@{}}%
\column{E}{@{}>{\hspre}l<{\hspost}@{}}%
\>[11]{}\Varid{logEquiv2}\;(\lambda \Varid{p}\;\Varid{q}\to \neg \;\Varid{p}\Longrightarrow\Varid{q})\;(\lambda \Varid{p}\;\Varid{q}\to \Varid{p}\Longrightarrow\neg \;\Varid{q}){}\<[E]%
\\
\>[11]{}\Conid{False}{}\<[E]%
\ColumnHook
\end{hscode}\resethooks

  \item $\neg P \implies Q$ and $Q \implies \neg P$ are not equivalent. By part 4 of Theorem 2.10 we see that $Q \implies \neg P$ is the contrapositive of $P \implies \neg Q$, which we have already shown to be non-equivalent to $\neg P \implies Q$.

        Verifying with Haskell:
        \begin{hscode}\SaveRestoreHook
\column{B}{@{}>{\hspre}l<{\hspost}@{}}%
\column{11}{@{}>{\hspre}l<{\hspost}@{}}%
\column{E}{@{}>{\hspre}l<{\hspost}@{}}%
\>[11]{}\Varid{logEquiv2}\;(\lambda \Varid{p}\;\Varid{q}\to \neg \;\Varid{p}\Longrightarrow\Varid{q})\;(\lambda \Varid{p}\;\Varid{q}\to \Varid{q}\Longrightarrow\neg \;\Varid{p}){}\<[E]%
\\
\>[11]{}\Conid{False}{}\<[E]%
\ColumnHook
\end{hscode}\resethooks

  \item $\neg P \implies Q$ and $\neg Q \implies P$ are equivalent, by part 4 of Theorem 2.10.

        Verifying with Haskell:
        \begin{hscode}\SaveRestoreHook
\column{B}{@{}>{\hspre}l<{\hspost}@{}}%
\column{11}{@{}>{\hspre}l<{\hspost}@{}}%
\column{E}{@{}>{\hspre}l<{\hspost}@{}}%
\>[11]{}\Varid{logEquiv2}\;(\lambda \Varid{p}\;\Varid{q}\to \neg \;\Varid{p}\Longrightarrow\Varid{q})\;(\lambda \Varid{p}\;\Varid{q}\to \neg \;\Varid{q}\Longrightarrow\Varid{p}){}\<[E]%
\\
\>[11]{}\Conid{True}{}\<[E]%
\ColumnHook
\end{hscode}\resethooks

  \item $P \implies (Q \implies R)$ and $Q \implies (P \implies R)$ are equivalent:
        \begin{align*}
          (P \implies (Q \implies R))   &\equiv \neg P \lor (\neg Q \lor R)
            && \text{by Theorem 2.10, part 3} \\
                                        &\equiv \neg P \lor \neg Q \lor R
            && \text{by associativity of } \lor \\
                                        &\equiv \neg Q \lor \neg P \lor R
            && \text{by commutativity of } \lor \\
                                        &\equiv \neg Q \lor (\neg P \lor R)
            && \text{by associativity of } \lor \\
                                        &\equiv (Q \implies (P \implies R))
            && \text{by Theorem 2.10, part 3}
        \end{align*}
        Verifying with Haskell:
        \begin{hscode}\SaveRestoreHook
\column{B}{@{}>{\hspre}l<{\hspost}@{}}%
\column{11}{@{}>{\hspre}l<{\hspost}@{}}%
\column{E}{@{}>{\hspre}l<{\hspost}@{}}%
\>[11]{}\Varid{logEquiv3}\;(\lambda \Varid{p}\;\Varid{q}\;\Varid{r}\to \Varid{p}\Longrightarrow(\Varid{q}\Longrightarrow\Varid{r}))\;(\lambda \Varid{p}\;\Varid{q}\;\Varid{r}\to \Varid{q}\Longrightarrow(\Varid{p}\Longrightarrow\Varid{r})){}\<[E]%
\\
\>[11]{}\Conid{True}{}\<[E]%
\ColumnHook
\end{hscode}\resethooks

  \item $P \implies (Q \implies R)$ and $(P \implies Q) \implies R$ are not equivalent:
        \begin{align*}
          (P \implies (Q \implies R))   &\equiv \neg P \lor (\neg Q \lor R)
            && \text{by Theorem 2.10, part 3} \\
                                        &\equiv \neg P \lor \neg Q \lor R
            && \text{by associativity of } \lor \\
                                        &\equiv \neg (P \land Q) \lor R
            && \text{by DeMorgan} \\
                                        &\equiv (P \land Q) \implies R
            && \text{by Theorem 2.10, part 3} \\
        \end{align*}
        Now, if $(P \land Q) \implies R$ and $(P \implies Q) \implies R$ are equivalent,
        then $P \land Q$ and $P \implies Q$ must also be equivalent. We check using a truth table:
        \begin{center}
        \begin{tabular}{ccccccc}
          \hline
          $P$ & $\land$ & $Q$ & $\iff$ & $(P$ & $\implies$ & $Q)$ \\
          \hline
           t  &    t    &  t  &    t   &   t  &      t     &  t   \\
           t  &    f    &  f  &    t   &   t  &      f     &  f   \\
           f  &    f    &  t  &    f   &   f  &      t     &  t   \\
           f  &    f    &  f  &    f   &   f  &      t     &  f   \\
          \hline
        \end{tabular}
        \end{center}
        As we can see from the $\iff$ column, the two are not equivalent. Thus $P \implies (Q \implies R)$ and $(P \implies Q) \implies R$ are not equivalent.

        Verifying with Haskell:
        \begin{hscode}\SaveRestoreHook
\column{B}{@{}>{\hspre}l<{\hspost}@{}}%
\column{11}{@{}>{\hspre}l<{\hspost}@{}}%
\column{E}{@{}>{\hspre}l<{\hspost}@{}}%
\>[11]{}\Varid{logEquiv3}\;(\lambda \Varid{p}\;\Varid{q}\;\Varid{r}\to \Varid{p}\Longrightarrow(\Varid{q}\Longrightarrow\Varid{r}))\;(\lambda \Varid{p}\;\Varid{q}\;\Varid{r}\to (\Varid{p}\Longrightarrow\Varid{q})\Longrightarrow\Varid{r}){}\<[E]%
\\
\>[11]{}\Conid{False}{}\<[E]%
\ColumnHook
\end{hscode}\resethooks

  \item $(P \implies Q) \implies P$ and $P$ are equivalent:
        \begin{center}
        \begin{tabular}{ccccccc}
          \hline
          $(P$ & $\implies$ & $Q)$ & $\implies$ & $P$ & $\iff$ & $P$ \\
          \hline
            t  &      t     &  t   &     t      &  t  &    t   &  t  \\
            t  &      f     &  f   &     t      &  t  &    t   &  t  \\
            f  &      t     &  t   &     f      &  f  &    t   &  f  \\
            f  &      t     &  f   &     f      &  f  &    t   &  f  \\
          \hline
        \end{tabular}
        \end{center}
        Verifying with Haskell:
        \begin{hscode}\SaveRestoreHook
\column{B}{@{}>{\hspre}l<{\hspost}@{}}%
\column{11}{@{}>{\hspre}l<{\hspost}@{}}%
\column{E}{@{}>{\hspre}l<{\hspost}@{}}%
\>[11]{}\Varid{logEquiv2}\;(\lambda \Varid{p}\;\Varid{q}\to (\Varid{p}\Longrightarrow\Varid{q})\Longrightarrow\Varid{p})\;(\lambda \Varid{p}\;\Varid{q}\to \Varid{p}){}\<[E]%
\\
\>[11]{}\Conid{True}{}\<[E]%
\ColumnHook
\end{hscode}\resethooks
  
  \item $P \lor Q \implies R$ and $(P \implies R) \land (Q \implies R)$ are equivalent:
        \begin{align*}
          (P \lor Q \implies R) &\equiv \neg (P \lor Q) \lor R
              && \text{by Theorem 2.10, part 3} \\
                                &\equiv (\neg P \land \neg Q) \lor R
              && \text{by DeMorgan} \\
                                &\equiv (\neg P \lor R) \land (\neg Q \lor R)
              && \text{by distribution} \\
                                &\equiv (P \implies R) \land (Q \implies R)
              && \text{by Theorem 2.10, part 3}
        \end{align*}
        Verifying with Haskell:
        \begin{hscode}\SaveRestoreHook
\column{B}{@{}>{\hspre}l<{\hspost}@{}}%
\column{11}{@{}>{\hspre}l<{\hspost}@{}}%
\column{E}{@{}>{\hspre}l<{\hspost}@{}}%
\>[11]{}\Varid{logEquiv3}\;(\lambda \Varid{p}\;\Varid{q}\;\Varid{r}\to \Varid{p}\mathrel{\vee}\Varid{q}\Longrightarrow\Varid{r})\;(\lambda \Varid{p}\;\Varid{q}\;\Varid{r}\to (\Varid{p}\Longrightarrow\Varid{r})\mathrel{\wedge}(\Varid{q}\Longrightarrow\Varid{r})){}\<[E]%
\\
\>[11]{}\Conid{True}{}\<[E]%
\ColumnHook
\end{hscode}\resethooks
\end{enumerate}

\subsection*{Exercise 2.21}
\begin{enumerate}
  \item
  Let $\Phi$ be defined as $P \lor \neg Q$. Now $\Phi$ has the desired truth table:
  \begin{center}
    \begin{tabular}{cccc|c}
      \hline
      $P$ & $\lor$ & $\neg$ & $Q$ & $\Phi$ \\
      \hline
       t  &    t   &    f   &  t  &    t   \\
       t  &    t   &    t   &  f  &    t   \\
       f  &    f   &    f   &  t  &    f   \\
       f  &    t   &    t   &  f  &    t   \\
      \hline
    \end{tabular}
  \end{center}

  \item
  There are a total of $2^4 = 16$ truth tables for 2-letter formulas.

  \item
  All 16 truth tables, with formulas, are listed in table~\ref{tab:2letterformulas} on page~\pageref{tab:2letterformulas}.

  \begin{table}
  \begin{tabular}{ll}
      \begin{tabular}{|cc|c|}
        \hline
        $P$ & $Q$ & $\Phi$ \\
        \hline
         t  &  t  &    f   \\
         t  &  f  &    f   \\
         f  &  t  &    f   \\
         f  &  f  &    f   \\
        \hline
      \end{tabular}
      \quad $\Phi := P \land \neg P$
      &
      \begin{tabular}{|cc|c|}
        \hline
        $P$ & $Q$ & $\Phi$ \\
        \hline
         t  &  t  &    f   \\
         t  &  f  &    f   \\
         f  &  t  &    f   \\
         f  &  f  &    t   \\
        \hline
      \end{tabular}
      \quad $\Phi := \neg (P \lor Q)$
      \\
      \begin{tabular}{|cc|c|}
        \hline
        $P$ & $Q$ & $\Phi$ \\
        \hline
         t  &  t  &    f   \\
         t  &  f  &    f   \\
         f  &  t  &    t   \\
         f  &  f  &    f   \\
        \hline
      \end{tabular}
      \quad $\Phi := \neg P \land Q$
      &
      \begin{tabular}{|cc|c|}
        \hline
        $P$ & $Q$ & $\Phi$ \\
        \hline
         t  &  t  &    f   \\
         t  &  f  &    f   \\
         f  &  t  &    t   \\
         f  &  f  &    t   \\
        \hline
      \end{tabular}
      \quad $\Phi := \neg P$
      \\
      \begin{tabular}{|cc|c|}
        \hline
        $P$ & $Q$ & $\Phi$ \\
        \hline
         t  &  t  &    f   \\
         t  &  f  &    t   \\
         f  &  t  &    f   \\
         f  &  f  &    f   \\
        \hline
      \end{tabular}
      \quad $\Phi := P \land \neg Q$
      &
      \begin{tabular}{|cc|c|}
        \hline
        $P$ & $Q$ & $\Phi$ \\
        \hline
         t  &  t  &    f   \\
         t  &  f  &    t   \\
         f  &  t  &    f   \\
         f  &  f  &    t   \\
        \hline
      \end{tabular}
      \quad $\Phi := \neg Q$
      \\
      \begin{tabular}{|cc|c|}
        \hline
        $P$ & $Q$ & $\Phi$ \\
        \hline
         t  &  t  &    f   \\
         t  &  f  &    t   \\
         f  &  t  &    t   \\
         f  &  f  &    f   \\
        \hline
      \end{tabular}
      \quad $\Phi := (P \land \neg Q) \lor (\neg P \land Q)$
      &
      \begin{tabular}{|cc|c|}
        \hline
        $P$ & $Q$ & $\Phi$ \\
        \hline
         t  &  t  &    f   \\
         t  &  f  &    t   \\
         f  &  t  &    t   \\
         f  &  f  &    t   \\
        \hline
      \end{tabular}
      \quad $\Phi := \neg (P \land Q)$
      \\
      \begin{tabular}{|cc|c|}
        \hline
        $P$ & $Q$ & $\Phi$ \\
        \hline
         t  &  t  &    t   \\
         t  &  f  &    f   \\
         f  &  t  &    f   \\
         f  &  f  &    f   \\
        \hline
      \end{tabular}
      \quad $\Phi := P \land Q$
      &
      \begin{tabular}{|cc|c|}
        \hline
        $P$ & $Q$ & $\Phi$ \\
        \hline
         t  &  t  &    t   \\
         t  &  f  &    f   \\
         f  &  t  &    f   \\
         f  &  f  &    t   \\
        \hline
      \end{tabular}
      \quad $\Phi := (P \land Q) \lor \neg (P \lor Q)$
      \\
      \begin{tabular}{|cc|c|}
        \hline
        $P$ & $Q$ & $\Phi$ \\
        \hline
         t  &  t  &    t   \\
         t  &  f  &    f   \\
         f  &  t  &    t   \\
         f  &  f  &    f   \\
        \hline
      \end{tabular}
      \quad $\Phi := Q$
      &
      \begin{tabular}{|cc|c|}
        \hline
        $P$ & $Q$ & $\Phi$ \\
        \hline
         t  &  t  &    t   \\
         t  &  f  &    f   \\
         f  &  t  &    t   \\
         f  &  f  &    t   \\
        \hline
      \end{tabular}
      \quad $\Phi := \neg P \lor Q$
      \\
      \begin{tabular}{|cc|c|}
        \hline
        $P$ & $Q$ & $\Phi$ \\
        \hline
         t  &  t  &    t   \\
         t  &  f  &    t   \\
         f  &  t  &    f   \\
         f  &  f  &    f   \\
        \hline
      \end{tabular}
      \quad $\Phi := P$
      &
      \begin{tabular}{|cc|c|}
        \hline
        $P$ & $Q$ & $\Phi$ \\
        \hline
         t  &  t  &    t   \\
         t  &  f  &    t   \\
         f  &  t  &    f   \\
         f  &  f  &    t   \\
        \hline
      \end{tabular}
      \quad $\Phi := P \lor \neg Q$
      \\
      \begin{tabular}{|cc|c|}
        \hline
        $P$ & $Q$ & $\Phi$ \\
        \hline
         t  &  t  &    t   \\
         t  &  f  &    t   \\
         f  &  t  &    t   \\
         f  &  f  &    f   \\
        \hline
      \end{tabular}
      \quad $\Phi := P \lor Q$
      &
      \begin{tabular}{|cc|c|}
        \hline
        $P$ & $Q$ & $\Phi$ \\
        \hline
         t  &  t  &    t   \\
         t  &  f  &    t   \\
         f  &  t  &    t   \\
         f  &  f  &    t   \\
        \hline
      \end{tabular}
      \quad $\Phi := P \lor \neg P$
  \end{tabular}
  \caption{The 16 truth tables for 2-letter formulas}
  \label{tab:2letterformulas}
  \end{table}

  \item
  I do not know if there is a general method for finding these formulas.
  There probably is, seeing as constructing the above formulas was very easy,
  but I am unable to precisely define the process.

  \item
  There would be a total of $2^8 = 256$ truth tables for 3-letter formulas.
  As for the question of a general method of finding the formulas, see the previous
  answer.
\end{enumerate}

\subsection*{Exercise 2.26}
\setcounter{equation}{0}
\begin{gather}
  \exists x,y \in \mathbb{Q} ~ (x < y)
  \\
  \forall x \in \mathbb{R} ~ \exists y \in {R} ~ (x < y)
  \\
  \forall x \in \mathbb{Z} ~ \exists m,n \in \mathbb{N} ~ (x = m - n)
\end{gather}

\subsection*{Exercise 2.27}
\setcounter{equation}{0}
\begin{gather}
  \forall x ~ (x \in \mathbb{Q} \implies
  \exists m,n ~ (m \in \mathbb{Z} \land n \in \mathbb{Z} \implies
  \neq 0 \land x = m/n))
  \\
  \forall x,y ~ (x \in F \land y \in D \implies (Oxy \implies Bxy))
\end{gather}

\subsection*{Exercise 2.31}
\begin{enumerate}
  \item
  The equation $x^2 + 1 = 0$ has a solution:
  \[ \exists x~(x^2 + 1 = 0) \]

  \item
  A largest natural number does not exist:
  \[ \forall x \in \mathbb{N}~\exists y \in \mathbb{N}~(x < y) \]

  \item
  The number 13 is prime ($d|n$ means `$d$ divides $n$'):
  \[ \forall d~(d|13 \implies d = 1 \lor d = 13) \]

  \item
  The number $n$ is prime:
  \[ \forall d~(n \neq 1 \land (d|n \implies d = 1 \lor d = n)) \]

  \item
  There are infinitely many primes:
  \[ \forall x~(\exists n \forall d~
   (n \neq 1 \land (d|n \implies d = 1 \lor d = n) \land x < n)) \]
\end{enumerate}

\subsection*{Exercise 2.32}
\begin{enumerate}
  \item
  Everyone loved Diana ($L(x,y)$ means `$x$ loved $y$', $d$ is Diana):
  \[ \forall x ~ L(x,d) \]

  \item
  Diana loved everyone: \\
  \[ \forall x ~ L(d,x) \]

  \item
  Man is mortal ($M(x)$ means `$x$ is a man', $M'(x)$ means `$x$ is mortal'):
  \[ \forall x ~ (M(x) \implies M'(x)) \]

  \item
  Some birds do not fly ($B(x)$ means `$x$ is a bird', $F(x)$ means `$x$ can fly'):
  \[ \exists x ~ (B(x) \land \neg F(x)) \]
\end{enumerate}

\subsection*{Exercise 2.33}
\begin{enumerate}
  \item
  Dogs that bark do not bite:
  \[ \forall x ~ (\text{Dog}(x) \land \text{Bark}(x) \implies \neg \text{Bite}(x)) \]

  \item
  All that glitters is not gold:
  \[ \exists x ~ (\text{Glitters}(x) \land \neg \text{Gold}(x)) \]

  \item
  Friends of Diana's friends are her friends:
  \[ \forall x,y ~ (\text{Friend}(d,x) \land \text{Friend}(x,y) \implies \text{Friend}(d,y)) \]

  \item
  The limit of $\frac{1}{n}$ as $n$ approaches infinity is zero:
  \[ \lim_{n \to \infty} \frac{1}{n} = 0 \]
\end{enumerate}

\subsection*{Exercise 2.34}
\begin{enumerate}
  \item
  Everyone loved Diana except Charles:
\end{enumerate}

\end{document}
