\documentclass{article}
%% ODER: format ==         = "\mathrel{==}"
%% ODER: format /=         = "\neq "
%
%
\makeatletter
\@ifundefined{lhs2tex.lhs2tex.sty.read}%
  {\@namedef{lhs2tex.lhs2tex.sty.read}{}%
   \newcommand\SkipToFmtEnd{}%
   \newcommand\EndFmtInput{}%
   \long\def\SkipToFmtEnd#1\EndFmtInput{}%
  }\SkipToFmtEnd

\newcommand\ReadOnlyOnce[1]{\@ifundefined{#1}{\@namedef{#1}{}}\SkipToFmtEnd}
\usepackage{amstext}
\usepackage{amssymb}
\usepackage{stmaryrd}
\DeclareFontFamily{OT1}{cmtex}{}
\DeclareFontShape{OT1}{cmtex}{m}{n}
  {<5><6><7><8>cmtex8
   <9>cmtex9
   <10><10.95><12><14.4><17.28><20.74><24.88>cmtex10}{}
\DeclareFontShape{OT1}{cmtex}{m}{it}
  {<-> ssub * cmtt/m/it}{}
\newcommand{\texfamily}{\fontfamily{cmtex}\selectfont}
\DeclareFontShape{OT1}{cmtt}{bx}{n}
  {<5><6><7><8>cmtt8
   <9>cmbtt9
   <10><10.95><12><14.4><17.28><20.74><24.88>cmbtt10}{}
\DeclareFontShape{OT1}{cmtex}{bx}{n}
  {<-> ssub * cmtt/bx/n}{}
\newcommand{\tex}[1]{\text{\texfamily#1}}	% NEU

\newcommand{\Sp}{\hskip.33334em\relax}


\newcommand{\Conid}[1]{\mathit{#1}}
\newcommand{\Varid}[1]{\mathit{#1}}
\newcommand{\anonymous}{\kern0.06em \vbox{\hrule\@width.5em}}
\newcommand{\plus}{\mathbin{+\!\!\!+}}
\newcommand{\bind}{\mathbin{>\!\!\!>\mkern-6.7mu=}}
\newcommand{\rbind}{\mathbin{=\mkern-6.7mu<\!\!\!<}}% suggested by Neil Mitchell
\newcommand{\sequ}{\mathbin{>\!\!\!>}}
\renewcommand{\leq}{\leqslant}
\renewcommand{\geq}{\geqslant}
\usepackage{polytable}

%mathindent has to be defined
\@ifundefined{mathindent}%
  {\newdimen\mathindent\mathindent\leftmargini}%
  {}%

\def\resethooks{%
  \global\let\SaveRestoreHook\empty
  \global\let\ColumnHook\empty}
\newcommand*{\savecolumns}[1][default]%
  {\g@addto@macro\SaveRestoreHook{\savecolumns[#1]}}
\newcommand*{\restorecolumns}[1][default]%
  {\g@addto@macro\SaveRestoreHook{\restorecolumns[#1]}}
\newcommand*{\aligncolumn}[2]%
  {\g@addto@macro\ColumnHook{\column{#1}{#2}}}

\resethooks

\newcommand{\onelinecommentchars}{\quad-{}- }
\newcommand{\commentbeginchars}{\enskip\{-}
\newcommand{\commentendchars}{-\}\enskip}

\newcommand{\visiblecomments}{%
  \let\onelinecomment=\onelinecommentchars
  \let\commentbegin=\commentbeginchars
  \let\commentend=\commentendchars}

\newcommand{\invisiblecomments}{%
  \let\onelinecomment=\empty
  \let\commentbegin=\empty
  \let\commentend=\empty}

\visiblecomments

\newlength{\blanklineskip}
\setlength{\blanklineskip}{0.66084ex}

\newcommand{\hsindent}[1]{\quad}% default is fixed indentation
\let\hspre\empty
\let\hspost\empty
\newcommand{\NB}{\textbf{NB}}
\newcommand{\Todo}[1]{$\langle$\textbf{To do:}~#1$\rangle$}

\EndFmtInput
\makeatother
%
%
%
%
%
%
% This package provides two environments suitable to take the place
% of hscode, called "plainhscode" and "arrayhscode". 
%
% The plain environment surrounds each code block by vertical space,
% and it uses \abovedisplayskip and \belowdisplayskip to get spacing
% similar to formulas. Note that if these dimensions are changed,
% the spacing around displayed math formulas changes as well.
% All code is indented using \leftskip.
%
% Changed 19.08.2004 to reflect changes in colorcode. Should work with
% CodeGroup.sty.
%
\ReadOnlyOnce{polycode.fmt}%
\makeatletter

\newcommand{\hsnewpar}[1]%
  {{\parskip=0pt\parindent=0pt\par\vskip #1\noindent}}

% can be used, for instance, to redefine the code size, by setting the
% command to \small or something alike
\newcommand{\hscodestyle}{}

% The command \sethscode can be used to switch the code formatting
% behaviour by mapping the hscode environment in the subst directive
% to a new LaTeX environment.

\newcommand{\sethscode}[1]%
  {\expandafter\let\expandafter\hscode\csname #1\endcsname
   \expandafter\let\expandafter\endhscode\csname end#1\endcsname}

% "compatibility" mode restores the non-polycode.fmt layout.

\newenvironment{compathscode}%
  {\par\noindent
   \advance\leftskip\mathindent
   \hscodestyle
   \let\\=\@normalcr
   \let\hspre\(\let\hspost\)%
   \pboxed}%
  {\endpboxed\)%
   \par\noindent
   \ignorespacesafterend}

\newcommand{\compaths}{\sethscode{compathscode}}

% "plain" mode is the proposed default.
% It should now work with \centering.
% This required some changes. The old version
% is still available for reference as oldplainhscode.

\newenvironment{plainhscode}%
  {\hsnewpar\abovedisplayskip
   \advance\leftskip\mathindent
   \hscodestyle
   \let\hspre\(\let\hspost\)%
   \pboxed}%
  {\endpboxed%
   \hsnewpar\belowdisplayskip
   \ignorespacesafterend}

\newenvironment{oldplainhscode}%
  {\hsnewpar\abovedisplayskip
   \advance\leftskip\mathindent
   \hscodestyle
   \let\\=\@normalcr
   \(\pboxed}%
  {\endpboxed\)%
   \hsnewpar\belowdisplayskip
   \ignorespacesafterend}

% Here, we make plainhscode the default environment.

\newcommand{\plainhs}{\sethscode{plainhscode}}
\newcommand{\oldplainhs}{\sethscode{oldplainhscode}}
\plainhs

% The arrayhscode is like plain, but makes use of polytable's
% parray environment which disallows page breaks in code blocks.

\newenvironment{arrayhscode}%
  {\hsnewpar\abovedisplayskip
   \advance\leftskip\mathindent
   \hscodestyle
   \let\\=\@normalcr
   \(\parray}%
  {\endparray\)%
   \hsnewpar\belowdisplayskip
   \ignorespacesafterend}

\newcommand{\arrayhs}{\sethscode{arrayhscode}}

% The mathhscode environment also makes use of polytable's parray 
% environment. It is supposed to be used only inside math mode 
% (I used it to typeset the type rules in my thesis).

\newenvironment{mathhscode}%
  {\parray}{\endparray}

\newcommand{\mathhs}{\sethscode{mathhscode}}

% texths is similar to mathhs, but works in text mode.

\newenvironment{texthscode}%
  {\(\parray}{\endparray\)}

\newcommand{\texths}{\sethscode{texthscode}}

% The framed environment places code in a framed box.

\def\codeframewidth{\arrayrulewidth}
\RequirePackage{calc}

\newenvironment{framedhscode}%
  {\parskip=\abovedisplayskip\par\noindent
   \hscodestyle
   \arrayrulewidth=\codeframewidth
   \tabular{@{}|p{\linewidth-2\arraycolsep-2\arrayrulewidth-2pt}|@{}}%
   \hline\framedhslinecorrect\\{-1.5ex}%
   \let\endoflinesave=\\
   \let\\=\@normalcr
   \(\pboxed}%
  {\endpboxed\)%
   \framedhslinecorrect\endoflinesave{.5ex}\hline
   \endtabular
   \parskip=\belowdisplayskip\par\noindent
   \ignorespacesafterend}

\newcommand{\framedhslinecorrect}[2]%
  {#1[#2]}

\newcommand{\framedhs}{\sethscode{framedhscode}}

% The inlinehscode environment is an experimental environment
% that can be used to typeset displayed code inline.

\newenvironment{inlinehscode}%
  {\(\def\column##1##2{}%
   \let\>\undefined\let\<\undefined\let\\\undefined
   \newcommand\>[1][]{}\newcommand\<[1][]{}\newcommand\\[1][]{}%
   \def\fromto##1##2##3{##3}%
   \def\nextline{}}{\) }%

\newcommand{\inlinehs}{\sethscode{inlinehscode}}

% The joincode environment is a separate environment that
% can be used to surround and thereby connect multiple code
% blocks.

\newenvironment{joincode}%
  {\let\orighscode=\hscode
   \let\origendhscode=\endhscode
   \def\endhscode{\def\hscode{\endgroup\def\@currenvir{hscode}\\}\begingroup}
   %\let\SaveRestoreHook=\empty
   %\let\ColumnHook=\empty
   %\let\resethooks=\empty
   \orighscode\def\hscode{\endgroup\def\@currenvir{hscode}}}%
  {\origendhscode
   \global\let\hscode=\orighscode
   \global\let\endhscode=\origendhscode}%

\makeatother
\EndFmtInput
%
\setlength{\parindent}{0pt}
\setlength{\parskip}{1ex plus 0.5ex minus 0.2ex}
\begin{document}

\section*{Exercises from Chapter 1}

\subsection*{Exercise 1.4}
The replacement of the condition \ensuremath{\Varid{k}\mathbin{\uparrow}\mathrm{2}\mathbin{>}\Varid{n}} by \ensuremath{\Varid{k}\mathbin{\uparrow}\mathrm{2}\geq \Varid{n}} in the definition of \ensuremath{\Varid{ldf}} makes no difference. This is because, in the case where \ensuremath{\Varid{k}} is such that \ensuremath{\Varid{k}\mathbin{\uparrow}\mathrm{2}\mathrel{=}\Varid{n}}, the condition \ensuremath{\Varid{divides}\;\Varid{k}\;\Varid{n}} is true. Thus the case is handled by the first guarded equation.

\subsection*{Exercise 1.6}
Based only on \ensuremath{\Varid{divides}}, I would guess:  
\begin{hscode}\SaveRestoreHook
\column{B}{@{}>{\hspre}l<{\hspost}@{}}%
\column{3}{@{}>{\hspre}l<{\hspost}@{}}%
\column{E}{@{}>{\hspre}l<{\hspost}@{}}%
\>[3]{}\Varid{rem}\mathbin{::}\Conid{Integer}\to \Conid{Integer}\to \Conid{Integer}{}\<[E]%
\ColumnHook
\end{hscode}\resethooks

\subsection*{Exercise 1.7}
\ensuremath{\Varid{divides}\;\mathrm{5}} is a procedure that takes an integer as a parameter and returns a boolean value. Thus it is an expression of type \ensuremath{\Conid{Integer}\to \Conid{Bool}}.

\ensuremath{\Varid{divides}\;\mathrm{5}\;\mathrm{7}} evaluates to a boolean value, thus it is an expression of type \ensuremath{\Conid{Bool}}.

\subsection*{Exercise 1.9}
A function that gives the maximum of a list of integers, using the predefined function \ensuremath{\Varid{max}}.

\begin{hscode}\SaveRestoreHook
\column{B}{@{}>{\hspre}l<{\hspost}@{}}%
\column{3}{@{}>{\hspre}l<{\hspost}@{}}%
\column{18}{@{}>{\hspre}c<{\hspost}@{}}%
\column{18E}{@{}l@{}}%
\column{22}{@{}>{\hspre}l<{\hspost}@{}}%
\column{E}{@{}>{\hspre}l<{\hspost}@{}}%
\>[3]{}\Varid{maxInt}{}\<[18]%
\>[18]{}\mathbin{::}{}\<[18E]%
\>[22]{}[\mskip1.5mu \Conid{Int}\mskip1.5mu]\to \Conid{Int}{}\<[E]%
\\
\>[3]{}\Varid{maxInt}\;[\mskip1.5mu \mskip1.5mu]{}\<[18]%
\>[18]{}\mathrel{=}{}\<[18E]%
\>[22]{}\Varid{error}\;\text{\tt \char34 empty~list\char34}{}\<[E]%
\\
\>[3]{}\Varid{maxInt}\;[\mskip1.5mu \Varid{x}\mskip1.5mu]{}\<[18]%
\>[18]{}\mathrel{=}{}\<[18E]%
\>[22]{}\Varid{x}{}\<[E]%
\\
\>[3]{}\Varid{maxInt}\;(\Varid{x}\mathbin{:}\Varid{xs}){}\<[18]%
\>[18]{}\mathrel{=}{}\<[18E]%
\>[22]{}\Varid{max}\;\Varid{x}\;(\Varid{maxInt}\;\Varid{xs}){}\<[E]%
\ColumnHook
\end{hscode}\resethooks

\subsection*{Exercise 1.10}
A function \ensuremath{\Varid{removeFst}} that removes the first occurrence of an integer \ensuremath{\Varid{m}} from a list of integers. If \ensuremath{\Varid{m}} does not occur in the list, the list remains unchanged.

\begin{hscode}\SaveRestoreHook
\column{B}{@{}>{\hspre}l<{\hspost}@{}}%
\column{3}{@{}>{\hspre}l<{\hspost}@{}}%
\column{16}{@{}>{\hspre}l<{\hspost}@{}}%
\column{24}{@{}>{\hspre}c<{\hspost}@{}}%
\column{24E}{@{}l@{}}%
\column{28}{@{}>{\hspre}l<{\hspost}@{}}%
\column{39}{@{}>{\hspre}c<{\hspost}@{}}%
\column{39E}{@{}l@{}}%
\column{42}{@{}>{\hspre}l<{\hspost}@{}}%
\column{E}{@{}>{\hspre}l<{\hspost}@{}}%
\>[3]{}\Varid{removeFst}{}\<[24]%
\>[24]{}\mathbin{::}{}\<[24E]%
\>[28]{}\Conid{Int}\to [\mskip1.5mu \Conid{Int}\mskip1.5mu]\to [\mskip1.5mu \Conid{Int}\mskip1.5mu]{}\<[E]%
\\
\>[3]{}\Varid{removeFst}\;\anonymous \;{}\<[16]%
\>[16]{}[\mskip1.5mu \mskip1.5mu]{}\<[24]%
\>[24]{}\mathrel{=}{}\<[24E]%
\>[28]{}[\mskip1.5mu \mskip1.5mu]{}\<[E]%
\\
\>[3]{}\Varid{removeFst}\;\Varid{m}\;{}\<[16]%
\>[16]{}(\Varid{x}\mathbin{:}\Varid{xs}){}\<[24]%
\>[24]{}\mid {}\<[24E]%
\>[28]{}\Varid{m}\equiv \Varid{x}{}\<[39]%
\>[39]{}\mathrel{=}{}\<[39E]%
\>[42]{}\Varid{xs}{}\<[E]%
\\
\>[24]{}\mid {}\<[24E]%
\>[28]{}\Varid{otherwise}{}\<[39]%
\>[39]{}\mathrel{=}{}\<[39E]%
\>[42]{}\Varid{x}\mathbin{:}\Varid{removeFst}\;\Varid{m}\;\Varid{xs}{}\<[E]%
\ColumnHook
\end{hscode}\resethooks

\subsection*{Exercise 1.13}
A function \ensuremath{\Varid{count}} for counting the number of occurences of a character in a string.

\begin{hscode}\SaveRestoreHook
\column{B}{@{}>{\hspre}l<{\hspost}@{}}%
\column{3}{@{}>{\hspre}l<{\hspost}@{}}%
\column{12}{@{}>{\hspre}l<{\hspost}@{}}%
\column{20}{@{}>{\hspre}c<{\hspost}@{}}%
\column{20E}{@{}l@{}}%
\column{24}{@{}>{\hspre}l<{\hspost}@{}}%
\column{35}{@{}>{\hspre}c<{\hspost}@{}}%
\column{35E}{@{}l@{}}%
\column{38}{@{}>{\hspre}l<{\hspost}@{}}%
\column{E}{@{}>{\hspre}l<{\hspost}@{}}%
\>[3]{}\Varid{count}{}\<[20]%
\>[20]{}\mathbin{::}{}\<[20E]%
\>[24]{}\Conid{Char}\to \Conid{String}\to \Conid{Int}{}\<[E]%
\\
\>[3]{}\Varid{count}\;\anonymous \;{}\<[12]%
\>[12]{}[\mskip1.5mu \mskip1.5mu]{}\<[20]%
\>[20]{}\mathrel{=}{}\<[20E]%
\>[24]{}\mathrm{0}{}\<[E]%
\\
\>[3]{}\Varid{count}\;\Varid{c}\;{}\<[12]%
\>[12]{}(\Varid{x}\mathbin{:}\Varid{xs}){}\<[20]%
\>[20]{}\mid {}\<[20E]%
\>[24]{}\Varid{c}\equiv \Varid{x}{}\<[35]%
\>[35]{}\mathrel{=}{}\<[35E]%
\>[38]{}\mathrm{1}\mathbin{+}\Varid{count}\;\Varid{c}\;\Varid{xs}{}\<[E]%
\\
\>[20]{}\mid {}\<[20E]%
\>[24]{}\Varid{otherwise}{}\<[35]%
\>[35]{}\mathrel{=}{}\<[35E]%
\>[38]{}\Varid{count}\;\Varid{c}\;\Varid{xs}{}\<[E]%
\ColumnHook
\end{hscode}\resethooks

\subsection*{Exercise 1.14}
A function \ensuremath{\Varid{blowup}} that converts a string $a_1, a_2, a_3, \dots$ to $a_1, a_2, a_2, a_3, a_3, a_3 \dots$

Using explicit recursion:
\begin{hscode}\SaveRestoreHook
\column{B}{@{}>{\hspre}l<{\hspost}@{}}%
\column{3}{@{}>{\hspre}l<{\hspost}@{}}%
\column{5}{@{}>{\hspre}l<{\hspost}@{}}%
\column{7}{@{}>{\hspre}l<{\hspost}@{}}%
\column{11}{@{}>{\hspre}c<{\hspost}@{}}%
\column{11E}{@{}l@{}}%
\column{15}{@{}>{\hspre}l<{\hspost}@{}}%
\column{21}{@{}>{\hspre}l<{\hspost}@{}}%
\column{29}{@{}>{\hspre}c<{\hspost}@{}}%
\column{29E}{@{}l@{}}%
\column{33}{@{}>{\hspre}l<{\hspost}@{}}%
\column{E}{@{}>{\hspre}l<{\hspost}@{}}%
\>[3]{}\Varid{blowup}{}\<[11]%
\>[11]{}\mathbin{::}{}\<[11E]%
\>[15]{}\Conid{String}\to \Conid{String}{}\<[E]%
\\
\>[3]{}\Varid{blowup}{}\<[11]%
\>[11]{}\mathrel{=}{}\<[11E]%
\>[15]{}\Varid{blowHelper}\;\mathrm{1}{}\<[E]%
\\
\>[3]{}\hsindent{2}{}\<[5]%
\>[5]{}\mathbf{where}{}\<[E]%
\\
\>[5]{}\hsindent{2}{}\<[7]%
\>[7]{}\Varid{blowHelper}{}\<[29]%
\>[29]{}\mathbin{::}{}\<[29E]%
\>[33]{}\Conid{Int}\to \Conid{String}\to \Conid{String}{}\<[E]%
\\
\>[5]{}\hsindent{2}{}\<[7]%
\>[7]{}\Varid{blowHelper}\;\anonymous \;{}\<[21]%
\>[21]{}[\mskip1.5mu \mskip1.5mu]{}\<[29]%
\>[29]{}\mathrel{=}{}\<[29E]%
\>[33]{}[\mskip1.5mu \mskip1.5mu]{}\<[E]%
\\
\>[5]{}\hsindent{2}{}\<[7]%
\>[7]{}\Varid{blowHelper}\;\Varid{n}\;{}\<[21]%
\>[21]{}(\Varid{x}\mathbin{:}\Varid{xs}){}\<[29]%
\>[29]{}\mathrel{=}{}\<[29E]%
\>[33]{}\Varid{replicate}\;\Varid{n}\;\Varid{x}\plus \Varid{blowHelper}\;(\Varid{n}\mathbin{+}\mathrm{1})\;\Varid{xs}{}\<[E]%
\ColumnHook
\end{hscode}\resethooks

More elegant solution:
\begin{hscode}\SaveRestoreHook
\column{B}{@{}>{\hspre}l<{\hspost}@{}}%
\column{3}{@{}>{\hspre}l<{\hspost}@{}}%
\column{12}{@{}>{\hspre}c<{\hspost}@{}}%
\column{12E}{@{}l@{}}%
\column{16}{@{}>{\hspre}l<{\hspost}@{}}%
\column{E}{@{}>{\hspre}l<{\hspost}@{}}%
\>[3]{}\Varid{blowup'}{}\<[12]%
\>[12]{}\mathbin{::}{}\<[12E]%
\>[16]{}\Conid{String}\to \Conid{String}{}\<[E]%
\\
\>[3]{}\Varid{blowup'}{}\<[12]%
\>[12]{}\mathrel{=}{}\<[12E]%
\>[16]{}\Varid{concat}\mathbin{\circ}(\Varid{zipWith}\;\Varid{replicate}\;[\mskip1.5mu \mathrm{1}\mathinner{\ldotp\ldotp}\mskip1.5mu]){}\<[E]%
\ColumnHook
\end{hscode}\resethooks

\subsection*{Exercise 1.15}
A function \ensuremath{\Varid{srtString}\mathbin{::}[\mskip1.5mu \Conid{String}\mskip1.5mu]\to [\mskip1.5mu \Conid{String}\mskip1.5mu]} that sorts a list of strings in alphabetical order

\begin{hscode}\SaveRestoreHook
\column{B}{@{}>{\hspre}l<{\hspost}@{}}%
\column{3}{@{}>{\hspre}l<{\hspost}@{}}%
\column{5}{@{}>{\hspre}l<{\hspost}@{}}%
\column{17}{@{}>{\hspre}l<{\hspost}@{}}%
\column{20}{@{}>{\hspre}l<{\hspost}@{}}%
\column{28}{@{}>{\hspre}c<{\hspost}@{}}%
\column{28E}{@{}l@{}}%
\column{32}{@{}>{\hspre}l<{\hspost}@{}}%
\column{43}{@{}>{\hspre}c<{\hspost}@{}}%
\column{43E}{@{}l@{}}%
\column{46}{@{}>{\hspre}l<{\hspost}@{}}%
\column{E}{@{}>{\hspre}l<{\hspost}@{}}%
\>[3]{}\Varid{srtString}{}\<[28]%
\>[28]{}\mathbin{::}{}\<[28E]%
\>[32]{}[\mskip1.5mu \Conid{String}\mskip1.5mu]\to [\mskip1.5mu \Conid{String}\mskip1.5mu]{}\<[E]%
\\
\>[3]{}\Varid{srtString}\;{}\<[17]%
\>[17]{}[\mskip1.5mu \mskip1.5mu]{}\<[28]%
\>[28]{}\mathrel{=}{}\<[28E]%
\>[32]{}[\mskip1.5mu \mskip1.5mu]{}\<[E]%
\\
\>[3]{}\Varid{srtString}\;{}\<[17]%
\>[17]{}\Varid{xs}{}\<[28]%
\>[28]{}\mathrel{=}{}\<[28E]%
\>[32]{}\Varid{m}\mathbin{:}(\Varid{srtString}\;(\Varid{removeString}\;\Varid{m}\;\Varid{xs})){}\<[E]%
\\
\>[3]{}\hsindent{2}{}\<[5]%
\>[5]{}\mathbf{where}\;\Varid{m}\mathrel{=}\Varid{mnmString}\;\Varid{xs}{}\<[E]%
\\[\blanklineskip]%
\>[3]{}\Varid{removeString}{}\<[28]%
\>[28]{}\mathbin{::}{}\<[28E]%
\>[32]{}\Conid{String}\to [\mskip1.5mu \Conid{String}\mskip1.5mu]\to [\mskip1.5mu \Conid{String}\mskip1.5mu]{}\<[E]%
\\
\>[3]{}\Varid{removeString}\;{}\<[17]%
\>[17]{}\anonymous \;{}\<[20]%
\>[20]{}[\mskip1.5mu \mskip1.5mu]{}\<[28]%
\>[28]{}\mathrel{=}{}\<[28E]%
\>[32]{}[\mskip1.5mu \mskip1.5mu]{}\<[E]%
\\
\>[3]{}\Varid{removeString}\;{}\<[17]%
\>[17]{}\Varid{m}\;{}\<[20]%
\>[20]{}(\Varid{x}\mathbin{:}\Varid{xs}){}\<[28]%
\>[28]{}\mid {}\<[28E]%
\>[32]{}\Varid{m}\equiv \Varid{x}{}\<[43]%
\>[43]{}\mathrel{=}{}\<[43E]%
\>[46]{}\Varid{xs}{}\<[E]%
\\
\>[28]{}\mid {}\<[28E]%
\>[32]{}\Varid{otherwise}{}\<[43]%
\>[43]{}\mathrel{=}{}\<[43E]%
\>[46]{}\Varid{x}\mathbin{:}\Varid{removeString}\;\Varid{m}\;\Varid{xs}{}\<[E]%
\\[\blanklineskip]%
\>[3]{}\Varid{mnmString}{}\<[28]%
\>[28]{}\mathbin{::}{}\<[28E]%
\>[32]{}[\mskip1.5mu \Conid{String}\mskip1.5mu]\to \Conid{String}{}\<[E]%
\\
\>[3]{}\Varid{mnmString}\;{}\<[17]%
\>[17]{}[\mskip1.5mu \mskip1.5mu]{}\<[28]%
\>[28]{}\mathrel{=}{}\<[28E]%
\>[32]{}[\mskip1.5mu \mskip1.5mu]{}\<[E]%
\\
\>[3]{}\Varid{mnmString}\;{}\<[17]%
\>[17]{}[\mskip1.5mu \Varid{x}\mskip1.5mu]{}\<[28]%
\>[28]{}\mathrel{=}{}\<[28E]%
\>[32]{}\Varid{x}{}\<[E]%
\\
\>[3]{}\Varid{mnmString}\;{}\<[17]%
\>[17]{}(\Varid{x}\mathbin{:}\Varid{xs}){}\<[28]%
\>[28]{}\mathrel{=}{}\<[28E]%
\>[32]{}\Varid{min}\;\Varid{x}\;(\Varid{mnmString}\;\Varid{xs}){}\<[E]%
\ColumnHook
\end{hscode}\resethooks

\subsection*{Exercise 1.17}
A function \ensuremath{\Varid{substring}\mathbin{::}\Conid{String}\to \Conid{String}\to \Conid{Bool}} that checks whether \ensuremath{\Varid{str1}} is a substring of \ensuremath{\Varid{str2}}.

\begin{hscode}\SaveRestoreHook
\column{B}{@{}>{\hspre}l<{\hspost}@{}}%
\column{3}{@{}>{\hspre}l<{\hspost}@{}}%
\column{11}{@{}>{\hspre}l<{\hspost}@{}}%
\column{14}{@{}>{\hspre}l<{\hspost}@{}}%
\column{19}{@{}>{\hspre}l<{\hspost}@{}}%
\column{27}{@{}>{\hspre}c<{\hspost}@{}}%
\column{27E}{@{}l@{}}%
\column{31}{@{}>{\hspre}l<{\hspost}@{}}%
\column{49}{@{}>{\hspre}c<{\hspost}@{}}%
\column{49E}{@{}l@{}}%
\column{52}{@{}>{\hspre}l<{\hspost}@{}}%
\column{E}{@{}>{\hspre}l<{\hspost}@{}}%
\>[3]{}\Varid{prefix}{}\<[27]%
\>[27]{}\mathbin{::}{}\<[27E]%
\>[31]{}\Conid{String}\to \Conid{String}\to \Conid{Bool}{}\<[E]%
\\
\>[3]{}\Varid{prefix}\;{}\<[11]%
\>[11]{}[\mskip1.5mu \mskip1.5mu]\;{}\<[19]%
\>[19]{}\Varid{ys}{}\<[27]%
\>[27]{}\mathrel{=}{}\<[27E]%
\>[31]{}\Conid{True}{}\<[E]%
\\
\>[3]{}\Varid{prefix}\;{}\<[11]%
\>[11]{}(\Varid{x}\mathbin{:}\Varid{xs})\;{}\<[19]%
\>[19]{}[\mskip1.5mu \mskip1.5mu]{}\<[27]%
\>[27]{}\mathrel{=}{}\<[27E]%
\>[31]{}\Conid{False}{}\<[E]%
\\
\>[3]{}\Varid{prefix}\;{}\<[11]%
\>[11]{}(\Varid{x}\mathbin{:}\Varid{xs})\;{}\<[19]%
\>[19]{}(\Varid{y}\mathbin{:}\Varid{ys}){}\<[27]%
\>[27]{}\mathrel{=}{}\<[27E]%
\>[31]{}(\Varid{x}\equiv \Varid{y})\mathrel{\wedge}\Varid{prefix}\;\Varid{xs}\;\Varid{ys}{}\<[E]%
\\[\blanklineskip]%
\>[3]{}\Varid{substring}{}\<[27]%
\>[27]{}\mathbin{::}{}\<[27E]%
\>[31]{}\Conid{String}\to \Conid{String}\to \Conid{Bool}{}\<[E]%
\\
\>[3]{}\Varid{substring}\;{}\<[14]%
\>[14]{}[\mskip1.5mu \mskip1.5mu]\;{}\<[19]%
\>[19]{}\anonymous {}\<[27]%
\>[27]{}\mathrel{=}{}\<[27E]%
\>[31]{}\Conid{True}{}\<[E]%
\\
\>[3]{}\Varid{substring}\;{}\<[14]%
\>[14]{}\anonymous \;{}\<[19]%
\>[19]{}[\mskip1.5mu \mskip1.5mu]{}\<[27]%
\>[27]{}\mathrel{=}{}\<[27E]%
\>[31]{}\Conid{False}{}\<[E]%
\\
\>[3]{}\Varid{substring}\;{}\<[14]%
\>[14]{}\Varid{xs}\;{}\<[19]%
\>[19]{}(\Varid{y}\mathbin{:}\Varid{ys}){}\<[27]%
\>[27]{}\mid {}\<[27E]%
\>[31]{}\Varid{prefix}\;\Varid{xs}\;(\Varid{y}\mathbin{:}\Varid{ys}){}\<[49]%
\>[49]{}\mathrel{=}{}\<[49E]%
\>[52]{}\Conid{True}{}\<[E]%
\\
\>[27]{}\mid {}\<[27E]%
\>[31]{}\Varid{substring}\;\Varid{xs}\;\Varid{ys}{}\<[49]%
\>[49]{}\mathrel{=}{}\<[49E]%
\>[52]{}\Conid{True}{}\<[E]%
\\
\>[27]{}\mid {}\<[27E]%
\>[31]{}\Varid{otherwise}{}\<[49]%
\>[49]{}\mathrel{=}{}\<[49E]%
\>[52]{}\Conid{False}{}\<[E]%
\ColumnHook
\end{hscode}\resethooks

\subsection*{Exercise 1.18}
Find expressions with the following types:
\begin{enumerate}
  \item \ensuremath{[\mskip1.5mu \Conid{String}\mskip1.5mu]}\\
     answer: \ensuremath{[\mskip1.5mu \text{\tt \char34 Text\char34},\text{\tt \char34 More~text\char34}\mskip1.5mu]}

  \item \ensuremath{(\Conid{Bool},\Conid{String})}\\
     answer: \ensuremath{(\Conid{True},\text{\tt \char34 Indeed\char34})}

  \item \ensuremath{[\mskip1.5mu (\Conid{Bool},\Conid{String})\mskip1.5mu]}\\
     answer: \ensuremath{[\mskip1.5mu (\Conid{True},\text{\tt \char34 Yes\char34}),(\Conid{False},\text{\tt \char34 No\char34})\mskip1.5mu]}

  \item \ensuremath{([\mskip1.5mu \Conid{Bool}\mskip1.5mu],\Conid{String})}\\
     answer: \ensuremath{([\mskip1.5mu \Conid{True},\Conid{False}\mskip1.5mu],\text{\tt \char34 What?\char34})}

  \item \ensuremath{\Conid{Bool}\to \Conid{Bool}}\\
     answer: \ensuremath{\neg }
\end{enumerate}

\subsection*{Exercise 1.19}
Find the types of the following predefined functions, supply them with arguments of the expected types, and try to guess what they do.

\begin{enumerate}
  \item \ensuremath{\Varid{head}\mathbin{::}[\mskip1.5mu \Varid{a}\mskip1.5mu]\to \Varid{a}}\\
     returns the first element from a list

  \item \ensuremath{\Varid{last}\mathbin{::}[\mskip1.5mu \Varid{a}\mskip1.5mu]\to \Varid{a}}\\
     returns the last element from a list

  \item \ensuremath{\Varid{init}\mathbin{::}[\mskip1.5mu \Varid{a}\mskip1.5mu]\to [\mskip1.5mu \Varid{a}\mskip1.5mu]}\\
     returns a new list with the last element from the old list removed

  \item \ensuremath{\Varid{fst}\mathbin{::}(\Varid{a},\Varid{b})\to \Varid{a}}\\
     returns the first element of a pair

  \item \ensuremath{(\plus )\mathbin{::}[\mskip1.5mu \Varid{a}\mskip1.5mu]\to [\mskip1.5mu \Varid{a}\mskip1.5mu]\to [\mskip1.5mu \Varid{a}\mskip1.5mu]}\\
     returns a list with the contents of the second list appended to the end of the first list

  \item \ensuremath{\Varid{flip}\mathbin{::}(\Varid{a}\to \Varid{b}\to \Varid{c})\to \Varid{b}\to \Varid{a}\to \Varid{c}}\\
     given a function of 2 arguments, returns a function with the order of the two arguments interchanged

  \item \ensuremath{\Varid{flip}\;(\plus )\mathbin{::}[\mskip1.5mu \Varid{a}\mskip1.5mu]\to [\mskip1.5mu \Varid{a}\mskip1.5mu]\to [\mskip1.5mu \Varid{a}\mskip1.5mu]}\\
     returns a list with the contents of the first list appended to the end of the second list
\end{enumerate}

\subsection*{Exercise 1.20}
Use \ensuremath{\Varid{map}} to write a function \ensuremath{\Varid{lengths}} that takes a list of lists and returns a list of the corresponding list lengths.

\begin{hscode}\SaveRestoreHook
\column{B}{@{}>{\hspre}l<{\hspost}@{}}%
\column{3}{@{}>{\hspre}l<{\hspost}@{}}%
\column{12}{@{}>{\hspre}c<{\hspost}@{}}%
\column{12E}{@{}l@{}}%
\column{16}{@{}>{\hspre}l<{\hspost}@{}}%
\column{E}{@{}>{\hspre}l<{\hspost}@{}}%
\>[3]{}\Varid{lengths}{}\<[12]%
\>[12]{}\mathbin{::}{}\<[12E]%
\>[16]{}[\mskip1.5mu [\mskip1.5mu \Varid{a}\mskip1.5mu]\mskip1.5mu]\to [\mskip1.5mu \Conid{Int}\mskip1.5mu]{}\<[E]%
\\
\>[3]{}\Varid{lengths}{}\<[12]%
\>[12]{}\mathrel{=}{}\<[12E]%
\>[16]{}\Varid{map}\;\Varid{length}{}\<[E]%
\ColumnHook
\end{hscode}\resethooks

\subsection*{Exercise 1.21}
Use \ensuremath{\Varid{map}} to write a function \ensuremath{\Varid{sumLengths}} that takes a list of lists and returns the sum of their lenghts.

\begin{hscode}\SaveRestoreHook
\column{B}{@{}>{\hspre}l<{\hspost}@{}}%
\column{3}{@{}>{\hspre}l<{\hspost}@{}}%
\column{15}{@{}>{\hspre}c<{\hspost}@{}}%
\column{15E}{@{}l@{}}%
\column{19}{@{}>{\hspre}l<{\hspost}@{}}%
\column{E}{@{}>{\hspre}l<{\hspost}@{}}%
\>[3]{}\Varid{sumLengths}{}\<[15]%
\>[15]{}\mathbin{::}{}\<[15E]%
\>[19]{}[\mskip1.5mu [\mskip1.5mu \Varid{a}\mskip1.5mu]\mskip1.5mu]\to \Conid{Int}{}\<[E]%
\\
\>[3]{}\Varid{sumLengths}{}\<[15]%
\>[15]{}\mathrel{=}{}\<[15E]%
\>[19]{}\Varid{sum}\mathbin{\circ}\Varid{map}\;\Varid{length}{}\<[E]%
\ColumnHook
\end{hscode}\resethooks

\subsection*{Exercise 1.22}
We modify the defining equation of \ensuremath{\Varid{ldp}} as follows:
\begin{hscode}\SaveRestoreHook
\column{B}{@{}>{\hspre}l<{\hspost}@{}}%
\column{3}{@{}>{\hspre}l<{\hspost}@{}}%
\column{8}{@{}>{\hspre}c<{\hspost}@{}}%
\column{8E}{@{}l@{}}%
\column{12}{@{}>{\hspre}l<{\hspost}@{}}%
\column{E}{@{}>{\hspre}l<{\hspost}@{}}%
\>[3]{}\Varid{ldp}{}\<[8]%
\>[8]{}\mathbin{::}{}\<[8E]%
\>[12]{}\Conid{Integer}\to \Conid{Integer}{}\<[E]%
\\
\>[3]{}\Varid{ldp}{}\<[8]%
\>[8]{}\mathrel{=}{}\<[8E]%
\>[12]{}\Varid{ldpf}\;\Varid{primes1}{}\<[E]%
\ColumnHook
\end{hscode}\resethooks

Now, \ensuremath{\Varid{ldp}} works exactly as though we had written \ensuremath{\Varid{ldp}\;\Varid{n}\mathrel{=}\Varid{ldpf}\;\Varid{primes1}\;\Varid{n}}.

\ensuremath{\Varid{ldpf}} is a function of type \ensuremath{[\mskip1.5mu \Conid{Integer}\mskip1.5mu]\to \Conid{Integer}\to \Conid{Integer}}. What we did above was to supply it with only a first argument, causing it to return a function of type \ensuremath{\Conid{Integer}\to \Conid{Integer}} -- which is what we wanted for \ensuremath{\Varid{ldp}}.

\end{document}
