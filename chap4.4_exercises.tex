\section{Algebra of Sets}

\begin{Exercise} [number=13]
  What are the types of the set difference operator $\setminus$ and of the inclusion operator $\subseteq$?
\end{Exercise}

\begin{Answer}
  $\subseteq$ :: $s \rightarrow s \rightarrow t$\\
  $\setminus$ :: $s \rightarrow s \rightarrow s$
\end{Answer}


\begin{Exercise} [number=14]
  Give the types of the following expressions:
  \Question $x \in \Set{x}{E(x)}$.
  \Question $\Set{x}{E(x)}$.
  \Question $(A \intersect B) \subseteq C$.
  \Question $(A \union B) \intersect C$.
  \Question $\forall{x} \colon x \in A \implies x \in B$.
  \Question $A=B$.
  \Question $a \in A \iff a \in B$.
\end{Exercise}

\begin{Answer}
$t$, $s$, $t$, $s$, $t$, $t$, $t$
\end{Answer}

%unfinished
%TODO leave the set notation and dive into logical quantifiers and
%then do Dijkstra calculation
\begin{Exercise} [number=17]
  $A$, $B$ and $C$ are sets. Show:
  \Question $A \nsubseteq B \iff A - B \neq \emptyset$.
  \Question $A \intersect B = A - (A - B)$.
\end{Exercise}

\begin{Answer} [number=17.1]
  \begin{structured_derivation}
    \task{Show that $A \nsubseteq B$ when:}
    \assumption{$A - B \neq \emptyset$}
    \task{$A - B \neq \emptyset$}
    \justification[\iff]{Rewrite the set difference notation (?)}
    \dijk{$\Set{x}{x \in A \land x \notin B} \neq \emptyset$}
    \justification[\iff]{Rewrite the \dots}
    \dijk{$\exists{x \in A} \suchThat x \notin B$}
    \observation{$A-B$ can be rewritten as:}{}    
  \end{structured_derivation}
\end{Answer}


\begin{Answer} [number=17.2]
  \begin{structured_derivation}
    \task{Prove that $A \intersect B = A - (A-B)$}
    \isProvedBy{Demonstrate that the RHS is a subset of the LHS and vice-versa}
    \begin{nested_derivation}
      \task{Prove that $A \intersect B \subseteq A - (A-B)$}
      \isProvedBy{??}
      \begin{nested_derivation}
        \task{Prove that $x = y$ when:}
        \assumption{$\exists{x} \suchThat x \in (A \intersect B)$}
        \observation{definition of subset \& intersection of
          sets}{$\exists{x} \suchThat x \in A \land x \in B$}
        \assumption{$\exists{y} \suchThat y \in (A - (A-B))$}
        \observation{??}{$\exists{y \in A} \suchThat y \notin A-B$}
        
        
      \end{nested_derivation}
    \end{nested_derivation}
    \begin{nested_derivation}
      \task{Prove that $A - (A - B) \subseteq A \intersect B$}    
    \end{nested_derivation}
  \end{structured_derivation}
\end{Answer}

%TODO missed steps on commutativity
\begin{Exercise} [number=19]
  Express $(A \union B) \intersect (C \union D)$ as the union of four
  intersections.
\end{Exercise}

\begin{Answer}
  \begin{structured_derivation}
    \task{$(A \union B) \intersect (C \union D)$}
    \justification[\equiv]{distributivity}
    \dijk{$\bigl(A \intersect (C \union D)\bigr) \union \bigl(B
      \intersect (C \union D)\bigr)$}
    \justification[\equiv]{distributivity}
    \dijk{$\bigl( (A \intersect C) \union (A \intersect D) \bigr)
      \union \bigl(B \intersect (C \union D)\bigr)$}
    \justification[\equiv]{distributivity}
    \dijk{$\bigl( (A \intersect C) \union (A \intersect D) \bigr)
      \union \bigl( (B \intersect C) \union (B \intersect D) \bigr)$}
    \justification[\equiv]{trivial rewrite}
    \dijk{$\cumulUnion \bigl( (A \intersect C), (A \intersect D), (B
        \intersect C), (B \intersect D) \bigr)$}
  \end{structured_derivation}
\end{Answer}

%unanswered
\begin{Exercise} [number=21]
  Show that $A \xor B = (A - B) \union (B - A) = (A \union B) - (A
  \intersect B)$.
  
  ($A \xor B$ is the set of all objects that are in $A$ or $B$ but
  \emph{not both}.)
\end{Exercise}

\begin{Answer}
  \begin{structured_derivation}
    \task{see above (I'm getting lazy)}
    \observation{??}{$A \xor B \equiv \Set{x}{x \in A \xor x \in B}$}
  \end{structured_derivation}
\end{Answer}

%unfinished
%TODO typeset a way to say X := {1, 2, 3}, y := {1}, z := {2}. Neither
%  y nor z is a subset of the other. (it's a counterexample)
\begin{Exercise} [number=23] 
  Let $X$ be a set with at least two elements. Then by Theorem \ref{THM:sets:RAT},
  the relation $\subseteq$ on $\powerset(X)$ has the properties of reflexivity,
  antisymmetry, and transitivity. The relation $\leq$ on $\reals$ also has these
  properties. The relation $\leq$ on $\reals$ has the further property of 
  \term{linearity}: for all $x, y \in \reals$, either $x \leq y$ or $y \leq x$.
  Show that $\subseteq$ on $\powerset(X)$ lacks this property.
\end{Exercise}

%unfinished
%TODO typeset from notebook
\begin{Exercise} [number=26]
  Give a logical translation of $\cumulIntersect \mathcal{F} \subseteq
  \cumulUnion \mathcal{G}$ using only the relation $\in$.
\end{Exercise}

%unfinished
%TODO typeset from notebook
\begin{Exercise} [number=27]
  Let $\mathcal{F}$ be a family of sets. Show that there is a set $A$ with the 
  following properties:
  \Question $\mathcal{F} \subseteq \powerset(A)$.
  \Question For all sets $B$: if $\mathcal{F} \subseteq \powerset(B)$ then $A \subseteq B$.
\end{Exercise}

%unanswered
%@tedious
\begin{Exercise} [number=29]
  fill in and prove Theorem 4.20
\end{Exercise}

%unfinished: not transferred from notebook.
%TODO: make the powersets cascadingly smaller
\begin{Exercise} [number=30]
Answer as many of the following questions as you can.
\Question Determine: $\powerset(\emptyset)$, $\powerset(\powerset(\emptyset))$, $\powerset(\powerset(\powerset(\emptyset)))$
\Question How many elements has $\powerset^{5}(\emptyset)$?
\Question How many elements has $\powerset(A)$, given that $A$ as $n$ elements?
\end{Exercise}

%unfinished: need to LaTeX-ize and formalize/arrange argument from notebook.
\begin{Exercise} [number=31]
Check whether the following is true: if two sets have the same subsets, then they are equal. I.e.: if $\powerset(A) = \powerset(B)$, then $A = B$. Give a proof or a refutation by means of a counterexample.
\end{Exercise}

%unanswered
\begin{Exercise} [number=32]
Is it true that for all sets $A$ and $B$:
\Question $\powerset(A \intersect B) = \powerset(A) \intersect \powerset(B)$?
\Question $\powerset(A \union B) = \powerset(A) \union \powerset(B)$?
\ExeText Provide either a proof or a refutation by counter-example.
\end{Exercise}

%unanswered
%@tedious
%@numerous
\begin{Exercise} [number=33, difficulty=1]
  Show: 

  \Question $B \intersect \bigl( \cumulUnion_{i \in I} A_i
  \bigr) = \cumulUnion_{i \in I}(B \intersect A_i)$
  
  \Question $B \union \bigl( \cumulIntersect_{i \in I} A_i \bigr) =
  \cumulIntersect_{i \in I}(B \union A_i)$
  
  \Question $\bigl(\cumulUnion_{i \in I} A_i \bigr)^c =
  \cumulIntersect_{i \in I} A_{i}^{c}$, assuming that $\forall{i} \in
  I \colon A_i \subseteq X$

  \Question $\bigl(\cumulIntersect_{i \in I} A_i \bigr)^c =
  \cumulUnion_{i \in I} A_{i}^{c}$, assuming that $\forall \in I
  \colon A_i \subseteq X$
\end{Exercise}

%unanswered
\begin{Exercise} [number=34, difficulty=1]
  Assume that you are given a certain set $A_0$. Suppose you are
  assigned the task of finding sets $A_1$, $A_2$, $A_3$, \dots, such
  that $\powerset(A_1) \subseteq A_0$, $\powerset(A_2) \subseteq A_1$,
  $\powerset(A_3) \subseteq A_2$, \dots Show that no matter how hard
  you try, you will eventually fail, that is: hit a set $A_n$ for
  which no $A_{n+1}$ exists such that $\powerset(A_{n+1}) \subseteq
  A_n$. (I.e., $\emptyset \notin A_n$.)

  \ExeText Suppose you can go on forever. Show this would entail
  $\powerset\bigl(\cumulIntersect_{i \in \naturals} A_i \bigr)
  \subseteq \cumulIntersect_{i \in \naturals} A_i$. Apply exercise 4.7.
\end{Exercise}

%unanswered
%@next
\begin{Exercise} [number=35, difficulty=1]
  Suppose that the collection $\mathcal{K}$ of sets satisfies the
  following condition:
  \begin{displaymath}
    \forall{A} \in \mathcal{K} \colon A = \emptyset \lor \exists{B \in
    \mathcal{K}} \suchthat A = \powerset(B)
  \end{displaymath}
  \ExeText Show that every element of $\mathcal{K}$ has the form
  $\powerset^n(\emptyset)$ for some $n \in \naturals$. (N.B.:
  $\powerset^0(\emptyset) = \emptyset$)
\end{Exercise}

%%% Local Variables:
%%% TeX-master: "exercises_base"
%%% End: