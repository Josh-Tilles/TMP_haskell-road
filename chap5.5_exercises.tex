\begin{Exercise} [number=66]
  Show that for ever $n \in \intgers$ with $n \neg 0$ it holds that
  $\equiv_n$ is an equivalence on $\integers$.
\end{Exercise}

Here is a Haskell implementation of the modulo relation:
\begin{code}

module SampleRelations

modulo :: Integer -> Integer -> Integer -> Bool
modulo n = \ x y -> n `divides` (x-y)
\end{code}

The relation that applies to two finite sets in case they have the
same number of elements is an equivalence on the collection of all
finite sets.

The corresponding equivalence on finite lists is given by the
following piece of Haskell code:
\begin{code}
equalSize :: [a] -> [b] -> Bool
equalSize list1 list2 = (length list1) == (length list2)
\end{code}

\begin{Exercise} [number=69]
  Determine whether the following relations on $\naturals$ are \dots % @TODO inline list
  \Question $\lbrack (2;3), (3;5), (5;2) \rbrack$
  \Question $\bigl{\lbrace} (n;\!m) \suchThat \abs{n - m} \geq 3 \bigr{\rbrace}$
\end{Exercise}

\begin{Exercise} [number=70]
  $A = \{ 1, 2, 3 \}$. Can you guess how many relations there are on
  this small set? Indeed, there must be sufficiently many to provide
  for the following questions.
  \Question Give an example of a relation on $A$ that is reflexive,
  but not symmetric and not transitive.
  \Question Give an example of a relation on $A$ that is symmetric,
  but not reflexive and not transitive.
  \Question Give examples (if any) of relations on $A$ that satisfy
  each of the six remaining possibilities \wrt reflexivity, symmetry
  and transitivity.
\end{Exercise}

\begin{Exercise} [number=71]
  For finite sets $A$ (0, 1, 2, 3, 4, and 5 elements and $n$ elements
  generally) the following table has entries for: the number of
  elements in $A^2$, the number of elements in $\powerset(A^2)$ (that
  is: the number of relations on $A$), the number of relations on $A$
  that are %@TODO inline list
  , and the number of equivalences on A.
  % @TODO fill in table
  Give all reflexive, symmetric, transitive relations and equivalences
  for the cases that $A = \emptyset$ (0 elements) and $A = \{0\}$ (1
  element). SHow there are exactly 13 transitive relations on
  $\{0,1\}$, and give the 3 that are not transitive. Put numbers on
  the places with question marks. (You are not requested to fill in
  the ~---)
\end{Exercise}

\begin{Exercise} [number=73]
  Suppose that $R$ is a symmetric and transitive relation on the set
  % @TODO prevent the line break on the math below.
  $A$ such that $\forall{x \in A} \colon \exists{y \in A} \suchThat
  x\mathrel{R}y$. Show that $R$ is reflexive on $A$.
\end{Exercise}

\begin{Exercise} [number=74]
  Let $R$ be a relation on $A$. Show that $R$ is an equivalence iff
  \dots % @TODO inline
\end{Exercise}
%%% Local Variables: 
%%% mode: latex
%%% TeX-master: "exercises_base"
%%% End: 
